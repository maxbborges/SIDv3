%% abtex2-modelo-relatorio-tecnico.tex, v<VERSION> laurocesar
%% Copyright 2012-2015 by abnTeX2 group at http://www.abntex.net.br/ 
%%
%% This work may be distributed and/or modified under the
%% conditions of the LaTeX Project Public License, either version 1.3
%% of this license or (at your option) any later version.
%% The latest version of this license is in
%%   http://www.latex-project.org/lppl.txt
%% and version 1.3 or later is part of all distributions of LaTeX
%% version 2005/12/01 or later.
%%
%% This work has the LPPL maintenance status `maintained'.
%% 
%% The Current Maintainer of this work is the abnTeX2 team, led
%% by Lauro César Araujo. Further information are available on 
%% http://www.abntex.net.br/
%%
%% This work consists of the files abntex2-modelo-relatorio-tecnico.tex,
%% abntex2-modelo-include-comandos and abntex2-modelo-references.bib
%%

% ------------------------------------------------------------------------
% ------------------------------------------------------------------------
% abnTeX2: Modelo de Relatório Técnico/Acadêmico em conformidade com 
% ABNT NBR 10719:2011 Informação e documentação - Relatório técnico e/ou
% científico - Apresentação
% Adaptado por Daniel Saad Nogueira Nunes para uso no IFG Formosa.
% ------------------------------------------------------------------------ 
% ------------------------------------------------------------------------

\documentclass[
	% -- opções da classe memoir --
	12pt,				% tamanho da fonte
	openright,			% capítulos começam em pág ímpar (insere página vazia caso preciso)
	oneside,			% para impressão em recto e verso. Oposto a oneside
	a4paper,			% tamanho do papel. 
	% -- opções da classe abntex2 --
	%chapter=TITLE,		% títulos de capítulos convertidos em letras maiúsculas
	%section=TITLE,		% títulos de seções convertidos em letras maiúsculas
	%subsection=TITLE,	% títulos de subseções convertidos em letras maiúsculas
	%subsubsection=TITLE,% títulos de subsubseções convertidos em letras maiúsculas
	% -- opções do pacote babel --
	english,			% idioma adicional para hifenização
	french,				% idioma adicional para hifenização
	spanish,			% idioma adicional para hifenização
	brazil,				% o último idioma é o principal do documento
	]{abntex2}

\usepackage{anteprojeto-tads}

% ---
% Informações de dados para CAPA e FOLHA DE ROSTO
% ---
\titulo{Título do Anteprojeto}
\autor{Maxwell Borges Bezerra}
\orientador{Daniel Saad Nogueira Nunes}
\linhadepesquisa{Cangaço}
\local{Brasilia, DF}
\data{2017}

\begin{document}

\selectlanguage{brazil}
\frenchspacing 
\imprimircapa
\imprimirfolhaderosto*

\section*{Introdução}
	A comunicação é responsável por introduzir mudanças comportamentais e comerciais nas mais diferentes sociedades, seja ela por informações presentes em canais de televisão, outdoors ou até mesmo panfletos, em ambientes públicos ou privados.
	
	A divulgação de forma estática e de formas mais tradicionais (revistas e jornais) já não são as formas mais eficientes de se expor um conteúdo ou propaganda, para \cite{escobar2007} novas tecnologias de comunicação colocaram a interatividade em evidencia, então com o uso de novas ferramentas mais interativas com seu receptor, torna a leitura menos monótona e se consegue uma maior atenção do espectador, além de se conseguir atingir um maior numero de pessoas, de forma mais consistente.
	
	Para \cite{machado2010}, o rápido crescimento das organizações juntamente com a internet obrigou elas a aderir novos conceitos de gestão. Pensando na maior abrangência, surge o conceito de sinalização digital, que para \cite{machado2010}, consiste na transmissão de conteúdo via internet,  onde essa mesma informação pode se ter receptores no mais diversos locais, independente de cidade, estado ou país com o uso de painéis e televisores apresentando informações e propagandas de forma dinâmica, podendo gerencia-las remotamente de acordo com a necessidade.
	
	Segundo a \cite{forbes2016}, pesquisas feitas pela agencia eMarketer afirmavam que até o final de 2016, 42\% da população da América latina iriam acessar regularmente as redes sociais. O uso das redes para disseminação de uma informação ou conteúdo vem se tornando uma das ferramentas mais atraentes para divulgações. Não apenas por ser um dos meios mais acessados atualmente, mas também por conta da maior facilidade de interações dos espectadores, usuários e empresas.
	
	Pensando não so na maior abrangência, mas também na interatividade, a expansão do conceito de sinalização digital com a união das mídias sociais permite não somente que as informações circulem fora de ambientes específicos, mas também que os receptores das informações transmitidas possam interagir quase que em tempo real com o conteúdo que é apresentado. Para \cite{santos2014}, no contexto do novo cenário da web é necessário um marketing em ambiente digital. 
	
\section*{Revisão da Literatura}
	

	\subsection*{Sinalização Digital}
	Para \cite{munari2006} a sinalização pode ser feita das mais diversas formas, desde códigos, sinais, luzes, canetas, lápis ou até mesmo por imagem. Ela deve ser melhorada de modo a  atingir o objetivo proposto pelo criador, merchandising, informativo ou entretenimento, atraindo a atenção do receptor. 
	
	De acordo com \cite{mishima2016} sistemas de sinalização digital que possuem algum dispositivo eletrônico são amplamente usados para exibir informações dinâmicas, ao contrários de anúncios estáticos, que ficam por determinado tempo na mesma informação. 
	
	Com o uso de um software para gerenciamento do conteúdo, \cite{machado2010} explica que a a sinalização digital consiste na transmissão dos conteúdos, via internet, à televisões ou painéis instalado em pontos estratégicos, pontos onde as pessoas costumam ficar algum tempo como bancos, ônibus, elevadores.
	
	\subsection*{Marketing Digital}
	\cite{santos2014} comenta que em um primeiro momento as estrategias de mercado era voltada para meios tradicionais como TV, rádio, jornais, revistas e outros. Com a popularização da internet foi necessário buscar outros meios de marketing, chegando ao marketing digital que é o uso das mídias sociais para exposição dos produtos.
	
	Na opinião de \cite{torres2000}, "As mídias sociais são sites na internet que permitem a criação e o compartilhamento de informações e conteúdos pelas pessoas e para as pessoas, nas quais o consumidor é ao mesmo tempo produtor e consumidor da informação. Elas recebem esse nome porque são sociais, ou seja, são livres e abertas à colaboração e interação de todos, e porque são mídias, ou seja, meios de transmissão de informações e conteúdo.". 
	
	\subsection*{Trabalhos Relacionados}
	Atualmente, uma solução encontrada que faz integração entre as redes sociais e ao conceito de sinalização digital é o software da OOZO, usando o kit Raspberry Pi, na sua versão mais completa é possível espelhar comentários das redes sociais e criar um filtro de hastag das redes sociais. Entretanto, todo o conteúdo é exibido apenas em tela publica, sendo possível apenas a pré-visualização em browsers \cite{oozo2017}.
	
	Outra solução encontrada é a MangoSigns, possui integração com as redes sociais, atualidades, informações do tempo e uma interface amigável. Entretanto, além de não fornecer planos com telas publicas é necessário um dispositivo próprio chamado Mango Sing Box para que as modificações feitas na interface do programa sejam transmitidas para a TV \cite{mango2017}.
	
	\begin{table}[]
		\caption{Comparativo}
		\begin{tabular}{|c|c|c|c|c|c|}
			\hline
			Quesito/Solução & OOZO & MangoSigns & TerceitaOPC & Quarta OPC & Quinta OPC \\ \hline
			I &  & & & & \\ \hline
			II &  & & & & \\ \hline
			III &  & & & & \\ \hline
			IV &  & & & & \\ \hline
			V &  & & & & \\ \hline
		\end{tabular}
	\end{table}
	
\section*{Justificativa}
	Atualmente, o IFB utiliza excepcionalmente o seu perfil do facebook e sua página oficial para realização das postagens referentes a informações da instituição, sendo necessário o administrador acessar cada página e realizar uma postagem independente em cada uma delas. Além de ser uma tarefa não trivial para se realizar todos os dias, a interatividade com e entre os usuários de cada página é algo um tanto quanto ruim, não somente pela falta de praticidade mas também pela falta de moderação das postagens e comentários.
	
\section*{Proposta}
	Com o uso da estrutura cliente-servidor e usando o conceito de sinalização digital, a proposta é fazer a integração do cliente com as redes sociais, apresentando as postagens e os comentários em tempo real, devidamente moderados, em telas espalhadas pelos campus do Instituto Federal de Brasília
	
\subsection*{Objetivos Gerais}
\subsection*{Objetivos Específicos}
\section*{Metodologia}
\section*{Cronograma e Descrição Orçamentária do Projeto}
\subsection*{Cronograma}

	\begin{table}[]
		\centering
		\caption{Cronograma}
		\label{my-label}
		\begin{tabular}{|c|c|c|c|c|c|c|c|}
			\hline
			OBJETIVO/MES & JUN & JUL & AGO & SET & OUT & NOV & DEZ \\ \hline
			I &  & & & & & & \\ \hline
			II &  & & & & & & \\ \hline
			III &  & & & & & & \\ \hline
			IV &  & & & & & & \\ \hline
			V &  & & & & & & \\ \hline
		\end{tabular}
	\end{table}

\bibliography{bibliografia}
\end{document}
