%% abtex2-modelo-relatorio-tecnico.tex, v<VERSION> laurocesar
%% Copyright 2012-2015 by abnTeX2 group at http://www.abntex.net.br/ 
%%
%% This work may be distributed and/or modified under the
%% conditions of the LaTeX Project Public License, either version 1.3
%% of this license or (at your option) any later version.
%% The latest version of this license is in
%%   http://www.latex-project.org/lppl.txt
%% and version 1.3 or later is part of all distributions of LaTeX
%% version 2005/12/01 or later.
%%
%% This work has the LPPL maintenance status `maintained'.
%% 
%% The Current Maintainer of this work is the abnTeX2 team, led
%% by Lauro César Araujo. Further information are available on 
%% http://www.abntex.net.br/
%%
%% This work consists of the files abntex2-modelo-relatorio-tecnico.tex,
%% abntex2-modelo-include-comandos and abntex2-modelo-references.bib
%%

% ------------------------------------------------------------------------
% ------------------------------------------------------------------------
% abnTeX2: Modelo de Relatório Técnico/Acadêmico em conformidade com 
% ABNT NBR 10719:2011 Informação e documentação - Relatório técnico e/ou
% científico - Apresentação
% Adaptado por Daniel Saad Nogueira Nunes para uso no IFG Formosa.
% ------------------------------------------------------------------------ 
% ------------------------------------------------------------------------

\documentclass[
	% -- opções da classe memoir --
	12pt,				% tamanho da fonte
	openright,			% capítulos começam em pág ímpar (insere página vazia caso preciso)
	oneside,			% para impressão em recto e verso. Oposto a oneside
	a4paper,			% tamanho do papel. 
	% -- opções da classe abntex2 --
	%chapter=TITLE,		% títulos de capítulos convertidos em letras maiúsculas
	%section=TITLE,		% títulos de seções convertidos em letras maiúsculas
	%subsection=TITLE,	% títulos de subseções convertidos em letras maiúsculas
	%subsubsection=TITLE,% títulos de subsubseções convertidos em letras maiúsculas
	% -- opções do pacote babel --
	english,			% idioma adicional para hifenização
	french,				% idioma adicional para hifenização
	spanish,			% idioma adicional para hifenização
	brazil,				% o último idioma é o principal do documento
	]{abntex2}

\usepackage{anteprojeto-tads}

% ---
% Informações de dados para CAPA e FOLHA DE ROSTO
% ---
\titulo{Título do Anteprojeto}
\autor{Maxwell Borges Bezerra}
\orientador{Daniel Saad Nogueira Nunes}
\linhadepesquisa{Cangaço}
\local{Brasilia, DF}
\data{2017}

\begin{document}

\selectlanguage{brazil}
\frenchspacing 
\imprimircapa
\imprimirfolhaderosto*




\section*{Introdução}
	A divulgação de forma estática e de formas mais tradicionais (Revistas e jornais) já não são as formas mais eficientes de se expor um conteúdo ou propaganda, para \cite{escobar2007} novas tecnologias de comunicação colocaram a interatividade em evidencia, então com o uso de novas ferramentas mais interativas com seu receptor, torna a leitura menos monótona e se consegue uma maior atenção do espectador, além de se conseguir atingir um maior numero de pessoas, de forma mais consistente.
	
	Pensando na maior abrangência, surge o conceito de sinalização digital, onde um mesmo conteúdo pode se ter receptores no mais diversos locais, independente de cidade, estado ou país com o uso de painéis e televisores apresentando informações dinâmicas alterando-as de acordo com a necessidade.
	
	Segundo a revista forbes, pesquisas feitas pela agencia eMarketer afirmavam que até o final de 2016, 42 da população da América latina iriam acessar regularmente as redes sociais \cite{forbes2016}. O uso das redes para disseminação de uma informação ou conteúdo vem se tornando uma das ferramentas mais atraentes para divulgações. Não apenas por ser um dos meios mais acessados atualmente, mas também por conta da maior facilidade de interações dos espectadores, usuários e empresas.
	
	Pensando não so na maior abrangência, mas também na interatividade, a expansão do conceito de sinalização digital com a união das mídias sociais
	
\section*{Revisão da Literatura}



\section*{Justificativa}

\section*{Proposta}

\subsection*{Objetivos Gerais}
\subsection*{Objetivos Específicos}
\section*{Metodologia}
\section*{Cronograma e Descrição Orçamentária do Projeto}
\bibliography{bibliografia}
\end{document}
