%% abtex2-modelo-relatorio-tecnico.tex, v<VERSION> laurocesar
%% Copyright 2012-2015 by abnTeX2 group at http://www.abntex.net.br/ 
%%
%% This work may be distributed and/or modified under the
%% conditions of the LaTeX Project Public License, either version 1.3
%% of this license or (at your option) any later version.
%% The latest version of this license is in
%%   http://www.latex-project.org/lppl.txt
%% and version 1.3 or later is part of all distributions of LaTeX
%% version 2005/12/01 or later.
%%
%% This work has the LPPL maintenance status `maintained'.
%% 
%% The Current Maintainer of this work is the abnTeX2 team, led
%% by Lauro César Araujo. Further information are available on 
%% http://www.abntex.net.br/
%%
%% This work consists of the files abntex2-modelo-relatorio-tecnico.tex,
%% abntex2-modelo-include-comandos and abntex2-modelo-references.bib
%%

% ------------------------------------------------------------------------
% ------------------------------------------------------------------------
% abnTeX2: Modelo de Relatório Técnico/Acadêmico em conformidade com 
% ABNT NBR 10719:2011 Informação e documentação - Relatório técnico e/ou
% científico - Apresentação
% Adaptado por Daniel Saad Nogueira Nunes para uso no IFG Formosa.
% ------------------------------------------------------------------------ 
% ------------------------------------------------------------------------

\documentclass[
	% -- opções da classe memoir --
	12pt,				% tamanho da fonte
	openright,			% capítulos começam em pág ímpar (insere página vazia caso preciso)
	oneside,			% para impressão em recto e verso. Oposto a oneside
	a4paper,			% tamanho do papel. 
	% -- opções da classe abntex2 --
	%chapter=TITLE,		% títulos de capítulos convertidos em letras maiúsculas
	%section=TITLE,		% títulos de seções convertidos em letras maiúsculas
	%subsection=TITLE,	% títulos de subseções convertidos em letras maiúsculas
	%subsubsection=TITLE,% títulos de subsubseções convertidos em letras maiúsculas
	% -- opções do pacote babel --
	english,			% idioma adicional para hifenização
	french,				% idioma adicional para hifenização
	spanish,			% idioma adicional para hifenização
	brazil,				% o último idioma é o principal do documento
	]{abntex2}

\usepackage{anteprojeto-tads}
\usepackage[normalem]{ulem}

% ---
% Informações de dados para CAPA e FOLHA DE ROSTO
% ---
\titulo{Título do Anteprojeto}
\autor{Maxwell Borges Bezerra}
\orientador{Daniel Saad Nogueira Nunes}
\linhadepesquisa{Sistema de Informação}
\local{Brasilia, DF}
\data{2017}


\newcommand{\daniel}[2]{\sout{#1} {\color{red} #2}}
\newcommand{\danielobs}[1]{{\color{red} $\diamond$ #1}}
\begin{document}

\selectlanguage{brazil}
\frenchspacing 
\imprimircapa
\imprimirfolhaderosto*



\section*{Introdução}
	A comunicação é responsável por introduzir mudanças comportamentais e comerciais nas mais diferentes sociedades, seja ela por informações presentes em canais de televisão, outdoors ou até mesmo panfletos, em ambientes públicos ou privados \cite{silva2007}.
	
	A divulgação de forma estática e de formas mais tradicionais (revistas e jornais) já não são as formas mais eficientes de se expor um conteúdo ou propaganda. Para \cite{escobar2007} novas tecnologias de comunicação colocaram a interatividade em evidencia, então,  a utilização de novas ferramentas mais interativas com seu receptor, torna a leitura menos monótona e é possível conseguir uma maior atenção do espectador, além de atingir um maior numero de pessoas de forma mais consistente.
	
	Para \cite{machado2010}, o rápido crescimento das organizações juntamente com a \textit{Internet} obrigou elas a aderir novos conceitos de gestão. Pensando na maior abrangência, surge o conceito de sinalização digital, que para \cite{machado2010}, consiste na transmissão de conteúdo via \textit{Internet},  onde essa mesma informação pode se ter receptores no mais diversos locais, independente de cidade, estado ou país com o uso de painéis e televisores apresentando informações e propagandas de forma dinâmica, podendo gerencia-las remotamente de acordo com a necessidade.
	
	Segundo a \cite{forbes2016}, pesquisas feitas pela agencia eMarketer afirmavam que até o final de 2016, 42\% da população da América latina iriam acessar regularmente as redes sociais. O uso das redes para disseminação de uma informação ou conteúdo vem se tornando uma das ferramentas mais atraentes para divulgações. Não apenas por ser um dos meios mais acessados atualmente, mas também por conta da maior facilidade de interações dos espectadores, usuários e empresas.
	
	Pensando não só na maior abrangência, mas também na interatividade, a expansão do conceito de sinalização digital com a união das mídias sociais permite não somente que as informações circulem fora de ambientes específicos, mas também que os receptores das informações transmitidas possam interagir quase que em tempo real com o conteúdo que é apresentado. Para \cite{santos2014}, no contexto do novo cenário da web é necessário um \textit{marketing} em ambiente digital. 
	

\section*{Revisão da Literatura}
	
	\subsection*{Sinalização Digital}
	Para \cite{munari2006} a sinalização pode ser feita das mais diversas formas, desde códigos, sinais, luzes, canetas, lápis ou até mesmo por imagem. Ela deve ser melhorada de modo a  atingir o objetivo proposto pelo criador, seja ela para propaganda, informativo ou entretenimento, devendo ela atrair a atenção do receptor. 
	
	De acordo com \cite{mishima2016} sistemas de sinalização digital que possuem algum dispositivo eletrônico são amplamente usados para exibir informações dinâmicas, ao contrários de anúncios estáticos, que ficam por determinado tempo na mesma informação. 
	
	Com o uso de um programas para gerenciamento do conteúdo, \cite{machado2010} explica que a a sinalização digital consiste na transmissão dos conteúdos, via \textit{Internet}, à televisões ou painéis instalado em pontos estratégicos onde as pessoas costumam ficar algum tempo como bancos, ônibus, elevadores.
	
	\subsection*{Marketing Digital}
	\cite{santos2014} comenta que em um primeiro momento as estratégias de mercado eram voltadas para meios tradicionais como TVs, rádios, jornais, revistas e outros. Com a popularização da \textit{Internet} foi necessário a busca por outros meios de para se realizar o \textit{marketing}, chegando ao \textit{marketing} digital que é o uso das mídias sociais para exposição dos produtos.
	
	Na opinião de \cite{torres2000}, as mídias sociais são paginas de Internet onde os usuários são ao mesmo tempo produtor e consumidor das informações que nelas são criadas, sendo possível a criação e compartilhamento dessas informações. Ainda para \cite{torres2000}, essas mídias receberam esse nome por serem livres e terem a possibilidade de colaboração e interação de todos que nelas estão, além de ser um meio de transmissão das informações e conteúdos.
	
	\subsection*{Trabalhos Relacionados}
	Atualmente, uma solução encontrada na literatura que faz integração entre \textit{marketing} digital e o conceito de sinalização digital é o software da OOZO, usando o kit \textit{Raspberry Pi}. Na sua versão mais completa é possível espelhar comentários e até mesmo criar um filtro de \textit{hashtag}, além de possuir em cada tela um \textit{QR code} de redirecionamento para a página da noticia completa. Entretanto, todo o conteúdo é exibido apenas em uma tela pública, painéis colocada em locais estratégicos de grande acesso público e com programação especificas, designadas pelo desenvolvedor com a assinatura de um serviço de exibição. Já em navegadores é possível apenas a pré-visualização das informações. Em sua versão gratuita, metade do conteúdo exibido será o seu e a outra parte será outras propagandas escolhidas pelos criados do software. \cite{oozo2017}.
	
	Outra solução encontrada é a MangoSigns, possui integração com as redes sociais, atualidades, informações do tempo e uma interface amigável. Entretanto, é necessário um dispositivo próprio chamado Mango Sing Box para que as modificações feitas na interface do programa sejam transmitidas para a TV, além de não possuir um \textit{QR code} para que o telespectador possa acessar a noticia completa. Em sua versão gratuita, a compatibilidade com as mídias sociais serão desabilitadas \cite{mango2017}.
	
	O SID (Sistema Inteligente de Divulgação de Informações do IFG-Formosa), é outra solução, entretanto a integração com as redes sociais é pouca, limitando-se ao Facebook com postagens no perfil do usuário do sistema. Usando o \textit{Raspberry Pi}, o SID apresenta as informações que lhe são configuradas, conta com um \textit{QR Code} apresentado na tela para caso o telespectador tenha interesse em acessar a noticia completa \cite{sobrinho2017}.
	
	O Screenly usa o \textit{Raspberry Pi} e um programa próprio que deve ser instalado no equipamento para seu funcionamento. Além de não possuir nenhum tipo de integração social ou \textit{QR code}, sua versão gratuita oferece apenas uma tela pública e somente 2 propagandas. Na sua versão mais completa oferece até 130 telas públicas, havendo a possibilidade de incluir telas extras \cite{screenly2017}.
	
	Outra solução encontrada na literatura corresponde ao \textit{software} Xibo, que trata-se de um sistema baseado em arquitetura cliente-servidor completo e flexível de sinalização digital que permite diversas customizações, na qual cada divulgação tem opção de estruturação das suas informações. Também suporta diferentes tipos de mídias como vídeos, imagens, texto, relógios, dados tabulares, etc. Possui gerenciador de conteúdo incluso (CMS) e possibilita que o servidor CMS e o módulo cliente estejam em dispositivos separados. No entanto, não possui integração com as redes sociais além de ser necessário que a cada divulgação a ser inserida tenha de ser estruturada como será a sua forma de apresentação \cite{xibo2017}.
	
	A Tabela 1 faz um comparativo entre os sistemas citados acima, comparando algumas das funcionalidades consideradas importantes para sistemas que trabalham com a implantação de sinalização digital e \textit{maketing} digital, na qual seus elementos comparativos são descritos a seguir:
	\begin{enumerate}[label=\Roman*)]
	\item Comprometimento com o propósito: o sistema em questão possibilita a veiculação de informações através de mecanismos de sinalização digital?
	\item Criação simples de divulgações: o operador possui facilidade de incluir novas divulgações com aspecto atrativo?
	\item Portabilidade: é possível visualizar a divulgação em diferentes dispositivos?
	\item Integração com redes sociais: o sistema integra-se nativamente de alguma forma com redes sociais?
	\item O conteúdo pode ser gerenciado em um dispositivo diferente ao que é criado, fortalecendo a descentralização e manutenção?
	\item A versão gratuita explora toda a capacidade do programa?
	\item Usa um sistema/dispositivo de fácil obtenção (Aplicativo próprio ou de uso comum)?
	\end{enumerate}
	
	\begin{table}[h!]
		\caption{Comparativo}
		\centering
		\begin{tabular}{|c|c|c|c|c|c|}
			\hline
			Quesito/Sistema & OOZO & MangoSigns & SID & Screenly & XIBO \\ \hline
			I & SIM & SIM & SIM & SIM & SIM \\ \hline
			II & SIM  & SIM & SIM & SIM & SIM \\ \hline
			III & NÃO & SIM & NÃO & NÃO & NÃO\\ \hline
			IV & SIM & NÃO & SIM & NÃO & NÃO\\ \hline
			V & NÃO & NÃO & NÃO & NÃO & NÃO\\ \hline
			VI & NÃO & NÃO & SIM & NÃO & SIM  \\ \hline
			VII & SIM & NÃO & SIM & SIM & SIM \\ \hline
		\end{tabular}
	\end{table}
		
\section*{Justificativa}
	Atualmente, o IFB utiliza excepcionalmente o seu perfil do facebook e sua página oficial para realização das postagens referentes a informações da instituição, sendo necessário o administrador acessar cada página e realizar uma postagem independente em cada uma delas. Além de ser uma tarefa não trivial para se realizar todos os dias, a interatividade com e entre os usuários de cada página é algo um tanto quanto ruim, não somente pela falta de praticidade mas também pela falta de moderação das postagens e comentários.
	
\section*{Proposta}
	Usando uma estrutura cliente-servidor, utilizando o sistema SID como base e com a união do conceito de sinalização e marketing digital, a proposta é fazer com que o sistema apresente conteúdos referentes ao IFB e essas informações tenha integração com as redes sociais, incluindo o Facebook, Realizando uma mineração de dados para fazer uma filtragem dos comentários e apresentar postagens e comentários em tempo real nas, devidamente moderados, em telas ou dispositivos móveis espalhadas pelos Câmpus Taguatinga do Instituto Federal de Brasília. Na versão para dispositivos móveis, o servidor irá enviar informações e avisos distintos para cada aluno ou turma, através de um login com a matricula cadastrada no SGA (Sistema de Gestão Academica) do Câmpus.
	
\subsection*{Objetivos Gerais}
	Com objetivo de diminuição da carência e aumento da facilidade de disseminação das informações e propagandas pertinentes ao IFB - Câmpus Taguatinga, o sistema deverá ser capaz de proporcionar objetividade e simplicidade nas informações a serem repassadas. Além de painéis instalados pelo Câmpus, ele deve ter a integração com as mídias sociais como Facebook e Twitter, unificando os sistemas de comunicação do IFB. Além das otimizações necessárias no sistema, será usada também técnicas de mineração de dados, para que seja possível selecionar conteúdos apropriados para inserção e publicação no sistema, filtrando informações e comentários que sejam mais propícios a ter reações positivas por partes dos telespectadores. Com a versão mobile do sistema, o aluno poderá não só ter acesso as propagandas que são publicadas de forma geral para o Câmpus, mas também a conteúdos específicos através da matricula do SGA, informações como mensagens encaminhada do professor para uma turma ou para um aluno especifico.


\subsection*{Objetivos Específicos}
	 \begin{itemize}
	\item Implementar um sistema para um âmbito mais acadêmico, para melhorar a disseminação de informações dentro do Câmpus.
	 	
	\item Melhorias do sistema usado como base, o SID \cite{sobrinho2017}.
	
	\item Usar a ferramenta Graph API para melhoria na integração do sistema com o Facebook.
	
	\item Integrar o sistema com outras mídias sociais como o twitter.
	
	\item Implementação de uma versão mobile do sistema, para possíveis consultas ou exibição do conteúdo, tornando a exibição das informações multiplataforma, exibindo-a em painéis, TVs, paginas de Internet ou celulares.
	
	\item  Integração da versão mobile como o sistema SGA.
	\end{itemize}
\section*{Metodologia}
	Partindo da pesquisa descritiva, será descrito os procedimentos e passos que foram seguidos e usados para obtenção do resultado desejado.
	
	A revisão de bibliografia é usada como meio de direcionamento do trabalho, usando comparações entre ferramentas desenvolvidas com o proposito principais de sinalização e marketing digital, partindo de tais soluções com o objetivo de avaliar os pontos negativos tendo como base as necessidades do Câmpus e então juntar ao processo de desenvolvimento os elementos que forem selecionados como principais e que são responsáveis por efetivar a disseminação da informação ao sistema de forma descentralizada e com o auxílio de ferramentas utilizadas no contexto WEB
	 
	Usando o SID como sistema base, uma estrutura cliente-servidor e conexão a \textit{Internet} , será implementado no sistema as interações com as redes sociais. As informações serão apresentadas em multiplataforma que podem ser televisores, painéis, paginas web ou celulares, essas informações podem ser alteradas acessando o servidor, um \textit{Raspberry Pi}, com o sistema instalado e conectado a Internet. Após serem criadas ou modificadas, as publicadas poderão ser transmitidas e acessadas pelos clientes em distintas plataformas ao mesmo tempo.
	
	A metodologia presente neste trabalho está direcionada aos aspectos específicos	do desenvolvimento de ferramentas computacionais com o intuito de melhoria nos processo de comunicação e veiculação de informações.
	
\section*{Cronograma e Descrição Orçamentária do Projeto}
	 O cronograma do projeto subdivide-se em diversas etapas, levando em conta os meses de agosto a dezembro de 2017 e fevereiro a junho de 2018, as etapas estão dispostas nesses meses da seguinte maneira:

	\begin{enumerate}[label=\Roman*)]
	\item Levantamento Bibliográfico
	
	\item Configuração do SID, que é usado como base para implementação da melhoria no projeto.
	
	\item Implementação de melhorias no código e no banco de dados do sistema base.
	
	\item Melhorias da interação do sistema com Facebook.
	
	\item Mineração de dados para recuperação de e moderação dos comentários que serão publicados.
	
	\item Porte do sistema para a versão mobile.
	
	\item Desenvolvimento da interface do sistema para a versão mobile.
	
	\item Implementação de um banco de dados para teste de integração com o SGA.
	
	\item Implementação da integração da versão mobile com o SGA.
	
	\item Implementação de integração com outras possíveis redes sociais.
	
	\item Revisão do sistema.
	
	\item Escrita da Monografia.
	
	\end{enumerate}

\begin{table}[h!]
	\centering
	\caption{Cronograma}
	\label{my-label}
	\begin{tabular}{c|c|c|c|c|c|c|c|c|c|c|}
		\cline{2-11}
		& \multicolumn{5}{c|}{2017}   & \multicolumn{5}{c|}{2018}   \\ \hline
		\multicolumn{1}{|c|}{OBJ/MÊS} & AGO & SET & OUT & NOV & DEZ & FEV & MAR & ABR & MAI & JUN \\ \hline
		\multicolumn{1}{|c|}{I}		&X&X& & & & & & & &\\ \hline
		\multicolumn{1}{|c|}{II}	& &X&X& & & & & & &\\ \hline
		\multicolumn{1}{|c|}{III}	& & &X&X&X& & & & &\\ \hline
		\multicolumn{1}{|c|}{IV}	& & & &X&X&X& & & &\\ \hline
		\multicolumn{1}{|c|}{V}		& & & & &X&X&X& & &\\ \hline
		\multicolumn{1}{|c|}{VI}	& & & & & &X&X& & &\\ \hline
		\multicolumn{1}{|c|}{VII}	& & & & & &X&X& & &\\ \hline
		\multicolumn{1}{|c|}{VIII}	& & & & & & &X&X& &\\ \hline
		\multicolumn{1}{|c|}{IX}	& & & & & & & &X&X&X\\ \hline
		\multicolumn{1}{|c|}{X}		& & & & & & & &X&X&X\\ \hline
		\multicolumn{1}{|c|}{XI}	& & & & & & & & &X&X\\ \hline
		\multicolumn{1}{|c|}{XII}	& & & & & & & &X&X&X\\ \hline
	\end{tabular}
\end{table}

\bibliography{bibliografia}
\end{document}
