\chapter[Introdução]{Introdução}

A comunicação é responsável por introduzir mudanças comportamentais e comerciais nas mais diferentes sociedades, seja ela por informações presentes em canais de televisão, outdoors ou até mesmo panfletos, em ambientes públicos ou privados \cite{silva2007}.

A divulgação de forma estática e de formas mais tradicionais (revistas e jornais) já não são as formas mais eficientes de se expor um conteúdo ou propaganda. Para \cite{escobar2007} novas tecnologias de comunicação colocaram a interatividade em evidencia, então,  a utilização de novas ferramentas mais interativas com seu receptor, torna a leitura menos monótona e é possível conseguir uma maior atenção do espectador, além de atingir um maior numero de pessoas de forma mais consistente.

Para \cite{machado2010}, o rápido crescimento das organizações juntamente com a \textit{Internet} obrigou elas a aderir novos conceitos de gestão. Pensando na maior abrangência, surge o conceito de sinalização digital, que para \cite{machado2010}, consiste na transmissão de conteúdo via \textit{Internet},  onde essa mesma informação pode se ter receptores no mais diversos locais, independente de cidade, estado ou país com o uso de painéis e televisores apresentando informações e propagandas de forma dinâmica, podendo gerencia-las remotamente de acordo com a necessidade.

Segundo a \cite{forbes2016}, pesquisas feitas pela agencia eMarketer afirmavam que até o final de 2016, 42\% da população da América latina iriam acessar regularmente as redes sociais. O uso das redes para disseminação de uma informação ou conteúdo vem se tornando uma das ferramentas mais atraentes para divulgações. Não apenas por ser um dos meios mais acessados atualmente, mas também por conta da maior facilidade de interações dos espectadores, usuários e empresas.

Pensando não só na maior abrangência, mas também na interatividade, a expansão do conceito de sinalização digital com a união das mídias sociais permite não somente que as informações circulem fora de ambientes específicos, mas também que os receptores das informações transmitidas possam interagir quase que em tempo real com o conteúdo que é apresentado. Para \cite{santos2014}, no contexto do novo cenário da web é necessário um \textit{marketing} em ambiente digital. 
