\chapter[Referencial Teórico]{Referencial Teórico}

\section{Sinalização digital}
Para \cite{munari2006} a sinalização pode ser feita das mais diversas formas, desde códigos, sinais, luzes, canetas, lápis ou até mesmo por imagem. Ela deve ser sensibilizada e melhorada de modo a  atingir o seu objetivo, seja ele para realização de uma venda ou até mesmo para exibição de um produto, atraindo a atenção do receptor para realizar o objetivo proposto pelo criador. Ainda para \cite{munari2006}, é possível unir vários sinais para formar apenas um, melhor estruturado.

Segundo \cite{redig2004} a sinalização de transito foi uma das primeiras manifestações do Design Informação, onde apenas com placas ou cores é possível mostrar algo para o receptor. \cite{redig2004} usa o metro como exemplo de uma forma errônea de sinalização, onde as informações são voltadas para o interesse dos emissores ao invés dos receptores. 

De acordo com \cite{mishima2016} sistemas de sinalização digital que possuem algum dispositivo eletrônico são amplamente usados para exibir informações dinâmicas, ao contrários de anúncios estáticos, que ficam por determinado tempo na mesma informação. Buscando o melhor custo beneficio e a maior facilidade de desenvolvimento \cite{mishima2016} escolheu usar o \lq\lq \textit{Raspaberry Pi}\rq\rq\ pela sua facilidade de compra, desempenho razoável e o custo do equipamento.

A Elemida é a líder na América Latina no seguimento de mídia digital, usa-se de telas com anúncios dinâmicos para atrair a atenção e audiência dos seus receptores. Presentes em mais de 70 cidades e 17 estados do pais a Elemidia é referencia e conta com mais de 1,7 mil anunciantes. \cite{elemidia2017}

A \cite{screenly2017} usando um sistema configurado no \lq\lq\textit{Raspaberry Pi}\rq\rq\ oferece serviço de \textit{Digital signage}, entretanto a versão gratuita conta apenas com uma tela inclusa, pequeno espaço de armazenamento, numero limitado de informações e nenhum suporte.

Nas pesquisas de \cite{brambilla2017}, quanto maior e melhor o investimento e desenvolvimento da sinalização no \textit{website}, melhor foi a percepção e avaliação do publico.