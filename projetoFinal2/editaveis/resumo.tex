\begin{resumo}
Este trabalho apresenta o Sistema Integrado de Divulgação de Informações do IFB Câmpus Taguatinga - SID que, por meio de uso dos conceitos de sinalização digital e marketing digital, visa veicular informações a serem repassadas no ambiente do Câmpus de forma simples, efetiva, interativa e dinâmica.

Para alcançar as  características desejadas, uma série de decisões e modificações foram realizadas no sistema SIDv2 proposto por \cite{sobrinho2017}. Foram realizadas modificações na arquitetura existente e implementado novas funcionalidades, de modo a flexibilizar a implantação de novos dispositivos que pudessem fazer uso desse sistema. 

O SID também possui a integração completa com a rede social Facebook, disponibilizando a possibilidade de realização de publicações em páginas do Facebook e apresentação dos conteúdos referentes a essas publicações. Com isto é realizada a unificação de veículos distintos de disseminação de informações, um baseado em sinalização digital e outro em redes sociais.

É proposto um aplicativo \textit{mobile} de comunicação que visa repassar as divulgações criadas, além de realizar o consumo de uma API fictícia para interação entre alunos e professores com a troca de mensagens. Esse consumo de API possibilita uma futura integração com o Sistema de Gestão Acadêmica - SGA.
 \vspace{\onelineskip}
    
 \noindent
 %\textbf{Palavras-chaves}: latex. abntex. editoração de texto.
\end{resumo}
