\chapter[Resultados]{Resultados}
\maxwell{COLOCAR MAIS TEXTO ANTES}

O módulo cliente do SID pode ser implementado em diversos dispositivos, bastando o mesmo possuir um navegador de paginas \textit{WEB} e conexão constante a internet para que seja realizado as consultas HTTP. Essa facilidade é possível graças a implementação em modelo cliente-servidor.

A implementação nesse modelo foi pensada para que o módulo consumisse poucos recursos, permitindo o uso de dispositivos que dispõem de uma grande eficiência energética, como: \textit{Raspberry} PI, \textit{BoxTVs} ou até mesmo em sistemas integrados as próprias \textit{smartTvs}.

Na busca de soluções que seguem a ideia de criação de publicação para apresentação em dispositivos externos com uso de dispositivos pequenos e com grande eficiência energética, foram encontradas diversas ferramentas com essa finalidade, as que apresentaram melhores funcionalidades foram a OOZO, a MangoSings, o SID versão 2, a Screenly e a XIBO.

Comparando cada uma das ferramentas encontradas com o novo SID é possível notar diversas funcionalidades que não estão presentes, mas que o SID as oferecem. 

O OOZO, apresentado na sessão \ref{sec:oozo}, é um sistema pago que possui  uma versão gratuita com acesso limitado. Ele oferece suporte a multiplataformas, controle WEB, integração a redes sociais, acesso via a publicação via QRCode e arquitetura cliente-servidor.

A ferramenta MangoSings, apresentado na sessão \ref{sec:mango}, é um sistema pago que possui uma versão gratuita com acesso limitado. A ferramenta possui acesso multiplataformas, inclusive móvel, diversos temas para apresentação e integração com as redes sociais para recuperação de postagens, entretanto somente para usuários que fazem assinatura do plano.

O SID versão 2, apresentado na sessão \ref{sec:sid}, é um sistema gratuito, que possui acesso somente \textit{WEB}, Aceso a publicação via QRCode e acesso limitado ao Facebook, sendo possível apenas realizar publicação da página de perfil do usuário.

\maxwell{falar sobre o Screenly e o XIBO} 

Diversos pontos que são considerados como essenciais no novo SID, as ferramentas apresentadas não oferecem ou oferecem de forma limitada em sua versão gratuita e paga. 

Entre esses pontos, está a de visualização das publicações por meio de um dispositivo móvel, onde com exceção do MangoSings, nenhum dos sistemas outros oferecem, necessitando de um dispositivo que possua um navegador WEB para acesso.

Outro ponto é o de integração com as redes sociais, nenhum das solução realizam uma integração tão completa quanto ao do SID, realizando publicações detalhadas, recuperando comentários, curtidas e fotos de perfil, além do filtro de comentários. 

A Tabela 1 faz um comparativo entre os sistemas citados acima, comparando algumas das funcionalidades consideradas importantes para sistemas que trabalham com a implantação de sinalização digital e \textit{maketing} digital, na qual seus elementos comparativos são descritos a seguir:
\begin{enumerate}[label=\Roman*)]
	\item Comprometimento com o propósito: o sistema em questão possibilita a veiculação de informações através de mecanismos de sinalização digital?
	\item Criação simples de divulgações: o operador possui facilidade de incluir novas divulgações com aspecto atrativo?
	\item Portabilidade: é possível visualizar a divulgação em diferentes dispositivos?
	\item Integração com redes sociais: o sistema integra-se nativamente de alguma forma com redes sociais, mesmo que de forma limitada?
	\item O conteúdo pode ser gerenciado em um dispositivo diferente ao que é criado, fortalecendo a descentralização e manutenção?
	\item A versão gratuita explora toda a capacidade do programa?
	\item Usa um sistema/dispositivo de fácil obtenção (Aplicativo próprio ou de uso comum)?
\end{enumerate}



\begin{table}[h!]
	\caption{Comparativo}
	\centering
	\begin{tabular}{|c|c|c|c|c|c|}
		\hline
		Quesito/Sistema & OOZO & MangoSigns & SID v2 & Screenly & XIBO \\ \hline
		I 				& SIM  & SIM		& SIM & SIM 	 & SIM	\\ \hline
		II 				& SIM  & SIM 		& SIM & SIM 	 & SIM	\\ \hline
		III				& NÃO  & SIM 		& NÃO & NÃO 	 & NÃO	\\ \hline
		IV 				& SIM  & NÃO 		& SIM & NÃO 	 & NÃO	\\ \hline
		V 				& SIM  & SIM 		& SIM & SIM 	 & SIM	\\ \hline
		VI 				& NÃO  & NÃO 		& SIM & NÃO 	 & SIM	\\ \hline
		VII 			& SIM  & SIM 		& SIM & SIM 	 & SIM	\\ \hline
	\end{tabular}
\end{table}
  




\maxwell{Após todos os testes, nenhum das opções de software citados atendia aos objetivos desejados, que é a integração completa e em tempo real com as redes sociais, a gratuidade e o sistema de gestão acadêmica do IFB (SGA). Apesar de alguns deles possuírem a integração, ela não estava completa e nenhuma das opções gratuitas oferecia todas as funcionalidades desejadas, as que suportavam a integração era de forma bem limitada, sendo necessário uma assinatura do software.}