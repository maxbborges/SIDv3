\chapter[Resultados]{Resultados}
A ideia anterior do SID seria um sistema inteligente de divulgação, entretanto, no presente trabalho a ideia foi alterada para um sistema integrado de divulgação. Isso se deu pelo fato de o sistema não possuir nenhum tipo de inteligência própria e sim realizar a integração entre redes sociais e usuário.

A mudança da ideia do SID, mostra que ele passou por diversas reformulações de suas concepções iniciais. Em sua versão anterior, ele possuía integração limitada e nenhum tipo de interatividade do sistema com o usuário e vice-versa. Então, no presente trabalho foi implementado toda a interação com o usuário, além da criação de um protótipo \textit{mobile}.

O presente trabalho apresenta a terceira versão do SID, com melhorias significantes apresentadas no capítulo \ref{cap:sid}, estando pronto para implantação de modo a servir de aparato ao campus no processo de veiculação de informações.

\maxwell{COLOCAR MAIS TEXTO}

\maxwell{FALAR SOBRE MODULO ADMINISTRADOR}

O módulo cliente do SID pode ser implementado em diversos dispositivos, bastando o mesmo possuir um navegador de paginas \textit{WEB} e conexão constante a internet para que seja realizado as consultas HTTP. Essa facilidade é possível graças a implementação em modelo cliente-servidor, esse modelo foi proposto para que o módulo consumisse poucos recursos, permitindo o uso de dispositivos que dispõem de uma grande eficiência energética, como: \textit{Raspberry} PI, \textit{BoxTVs} ou até mesmo em sistemas integrados as próprias \textit{smartTvs}.

\maxwell{FALAR SOBRE MODULO MOBILE}

Na busca de soluções que seguem a ideia de criação de publicação para apresentação em dispositivos externos com uso de dispositivos pequenos e com grande eficiência energética, foram encontradas diversas ferramentas com essa finalidade, todas elas usando o conceito de arquitetura cliente-servidor, as que apresentaram melhores funcionalidades foram a OOZO, a MangoSings, o SID versão 2, a Screenly e a XIBO.

O OOZO, apresentado na sessão \ref{sec:oozo}, é um sistema pago, porém possui uma versão gratuita com acesso limitado aos recursos oferecidos. Ele oferece suporte a multiplataformas, controle WEB, integração a redes sociais e acesso a publicações via QRCode. Apresentando duas limitações, uma é no que tange a integração com as redes sociais, onde é possível apenas recuperar e deletar publicações, a outra limitação está no acesso as divulgações por dispositivo móvel, onde é necessário o uso de navegadores \textit{WEB}. 

A MangoSings, apresentado na sessão \ref{sec:mango}, é um sistema pago que possui uma versão gratuita com acesso limitado. A ferramenta possui acesso multiplataformas, inclusive móvel, diversos temas para apresentação e integração com as redes sociais. Possuindo a limitação da integração está disponível apenas na versão paga, além de ser possível apenas recuperar as publicações.

O SID versão 2, apresentado na sessão \ref{sec:sid}, é um sistema gratuito, que possui acesso somente \textit{WEB}, aceso a publicação via QRCode e integração as redes sociais. Nele é possível apenas realizar publicação da página de perfil do usuário, não sendo possível recuperar nenhum dado.

\maxwell{falar sobre o Screenly e o XIBO} 

Comparando cada uma das ferramentas encontradas com o novo SID é possível notar diversas funcionalidades que não estão presentes, são limitadas ou não são gratuitas, portanto diversos pontos que são considerados como essenciais no novo SID as ferramentas apresentadas não suportam.

Entre esses pontos, está a visualização das publicações por meio de um dispositivo móvel, onde com exceção do MangoSings, nenhum dos sistemas outros oferecem um aplicativo para \textit{smartphones}, necessitando de um dispositivo que possua um navegador WEB para acesso.

Outro ponto é o de integração com as redes sociais, nenhum das solução analisadas realizam uma integração tão completa quanto ao do SID, sendo possível criar novas publicações, recuperar comentários, curtidas e fotos de perfil, além do filtro de comentários.

\maxwell{Falar mais sobre o SID}

A Tabela 3 tem o intuito de fazer um comparativo entre os sistemas apresentados, agora com inclusão do SID em sua terceira versão, comparando algumas das funcionalidades consideradas importantes para sistemas que trabalham com a implantação de sinalização digital e \textit{maketing} digital, na qual seus elementos comparativos são descritos a seguir:
\begin{enumerate}[label=(\Roman*)]
	\item Comprometimento com o propósito: o sistema em questão possibilita a veiculação de informações através de mecanismos de sinalização digital?
	\item Criação simples de divulgações: o operador possui facilidade de incluir novas divulgações com aspecto atrativo?
	\item Portabilidade: é possível visualizar a divulgação em diferentes dispositivos?
	\item Integração com redes sociais: o sistema integra-se nativamente de alguma forma com redes sociais, mesmo que de forma limitada?
	\item O conteúdo pode ser gerenciado em um dispositivo diferente ao que é criado, fortalecendo a descentralização e manutenção?
	\item A versão gratuita explora toda a capacidade do programa?
	\item Usa um sistema/dispositivo de fácil obtenção (Aplicativo próprio ou de uso comum)?
\end{enumerate}

\begin{table}[h!]
	\caption{Comparativo}
	\centering
	\begin{tabular}{|c|c|c|c|c|c|c}
		\hline
		Quesito/Sistema & OOZO & MangoSigns & SID v2 & Screenly & XIBO & SID v3 \\ \hline
		I 				& SIM  & SIM		& SIM & SIM 	 & SIM	& \\ \hline
		II 				& SIM  & SIM 		& SIM & SIM 	 & SIM	 &\\ \hline
		III				& NÃO  & SIM 		& NÃO & NÃO 	 & NÃO	&\\ \hline
		IV 				& SIM  & NÃO 		& SIM & NÃO 	 & NÃO	&\\ \hline
		V 				& SIM  & SIM 		& SIM & SIM 	 & SIM	&\\ \hline
		VI 				& NÃO  & NÃO 		& SIM & NÃO 	 & SIM	&\\ \hline
		VII 			& SIM  & SIM 		& SIM & SIM 	 & SIM	&\\ \hline
	\end{tabular}
\end{table}

\maxwell{FALAR SOBRE AS DIFICULDADES}