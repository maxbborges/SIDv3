\chapter[Introdução]{Introdução}

\maxwell{FALANDO SOBRE A MIDIA}

Apesar de muito usada, a definição exata de mídia é complicada de ser explicada. De acordo com \cite[p.49]{guazina2007}, o uso predominante do termo ``mídia'' é para representar um conjunto de meios de comunicação, representado por meios de comunicação em massa como jornais, televisão, rádio, cinema e \textit{Internet}.

Sendo a mídia, talvez, responsável por introduzir mudanças comportamentais e comerciais nas mais diferentes sociedades, seja ela por informações presentes em canais de televisão, outdoors ou até mesmo panfletos, em ambientes públicos ou privados, \cite[p.3]{escobar2007} propõe que a mídia é capaz de redefinir ``o modo como o homem se comunica e se relaciona com os semelhante.''

Para se ter uma ideia do poder que a mídia tem, para \cite{silva2007}, quando a população não tem acesso a outras fontes de informações, as notícias e mensagens veiculadas pelas mídias são muitas vezes vistas como verdades inquestionáveis. \cite[p.53]{guazina2007} aponta que os meios de comunicação são visto como potenciais construtores de conhecimento e formadores de compreensão sobre mundo e política.

Ainda sobre a influência da mídia, para \cite[p.54]{hjarvard2012}, surge o conceito de midiatização. Para ele, a midiatização é um termo usado para caracterizar a influência da mídia e coloca também a midiatização como ``um processo contínuo em que os meios alteram as relações e o comportamento humanos e, assim, alteram a sociedade e a cultura''. 

\maxwell{POR QUE A MIDIA MUDOU?}

Com o passar dos anos, a mídia foi obtendo novas formas, as divulgações de forma estática e mais tradicionais (revistas e jornais) foram deixando de serem os meios mais eficientes de se expor um conteúdo ou propaganda, foi necessário então que novas formas de expor conteúdos fossem pensadas e desenvolvidas.

\maxwell{UNIAO MIDIA E MARKETING}

Nos meios de comunicação em massa sempre estão presente os mais diversos tipos de propagandas e divulgações, o \textit{marketing} é responsável por criar e melhorar as ferramentas que tem como objetivo influenciar pessoas a adquirir ou aderir determinado produto ou serviço. Assim como a mídia, novas formas e ferramentas de \textit{marketing} também foram surgindo, ampliando os meios de como as informações são repassadas. 

\maxwell{MELHORIA DO MARKETING e INTERATIVIDADE}

Para a ampliação, foi necessário que os caminhos da publicidade e da tecnologia fossem se convergindo para se então definir um novo conceito de \textit{marketing}, o \textit{marketing} Digital. O \textit{marketing} digital conta com novas tecnologias de comunicação que para \cite[p.2]{escobar2007}, isso coloca a interatividade em evidencia, então, a utilização de novas ferramentas mais interativas com seu receptor, tornam a leitura menos monótona e passível de atingir uma maior atenção do espectador de forma mais consistente.  

\maxwell{EXPLICAÇAO DE INTERATIVIDADE}

As vantagens para a utilização da interatividade estão presentes em diversas formas, como por exemplo para \cite[p.4]{escobar2007}, transmissões ao vivo por rádio ou televisão permite o acesso a um dado acontecimento no exato momento em que ele acontece, mas quando se tem o advento da Internet coloca-se a possibilidade de interação com a informação que é recebida, isso quase que instantaneamente, onde os integrantes atuam simultaneamente, comentando ou opinando sobre aquele determinado assunto. Para \cite{deuze2002}, o advento da \textit{Internet} traz a possibilidade do público ``responder, interagir ou mesmo customizar certas histórias''. 

\maxwell{LINK ENTRE MARKETING DIGITAL E MIDIAS DIGITAIS}

A evolução do \textit{marketing} digital trouxe consigo a união das mídias sociais e dispositivos móveis. Isso permite não somente que as informações circulem fora de ambientes específicos, mas também que os receptores das informações transmitidas possam interagir quase que em tempo real com o conteúdo que é apresentado. Para \cite{santos2014}, no contexto do novo cenário da web é necessário um marketing em ambiente digital.

\maxwell{LINK ENTRE MARKETING, MÍDIAS DIGITAIS E SINALIZACAO}

Para \cite[p.7]{machado2010}, o rápido crescimento das organizações juntamente com a Internet obrigou elas a aderir novos conceitos de gestão e apresentação das informações, usando não somente os veículos de comunicação e as mídias digitais. Pensando na maior abrangência, surge o conceito de sinalização digital, que para \cite[p.37]{machado2010}, consiste na transmissão de conteúdo via Internet, onde essa mesma informação pode se ter receptores no mais diversos locais, independente de cidade, estado ou país com o uso de painéis e televisores apresentando informações e propagandas de forma dinâmica, nos mais diversos pontos e com a possibilidade de gerenciá-las remotamente de acordo com a necessidade. 

A uso das redes para disseminação de uma informação ou conteúdo vem se tornando uma das ferramentas mais atraentes para divulgações. Não apenas por ser um dos meios mais acessados atualmente, mas também por conta da maior facilidade de interações dos espectadores, usuários e empresas. 

\maxwell{LINK DO MARKETING E DO MOBILE}

Com o passar dos anos os dispositivos móveis estão sendo cada vez mais usados pelas pessoas, isso vem atraindo cada vez mais o foco dele como ferramenta para divulgação de informações. A pesquisa da \cite{fgv2017} aponta que até outubro de 2017 teria-se 208 milhões de aparelhos ativos no brasil e que até maio de 2018 terá cerca de 220 milhões de de \textit{smartphones} ativos, mais de 1 aparelho por habitante. 

Não somente pelo grande numero de dispositivos, mas também pela grande integração com as redes sociais que segundo a \cite{forbes2016}, pesquisas feitas pela agencia eMarketer afirmavam que até o final de 2016, 42\% da população da América latina iriam acessar regularmente as redes sociais, em 2016 estimava-se que 74\% da população do Brasil que usasse a internet no pais teriam uma conta no Facebook, se comparado o mundo todo, o número chega a 2 bilhões de usuários em 2017. 

\maxwell{FALAR SOBRE OS MEIOS DE COMUNICAÇÃO DO IFB E O QUE O SID PODE MUDAR}

\section{Motivação}
A atenção do espectador a uma determinada informação que está sendo apresentada é algo crucial para o sucesso das notícias que estão sendo exibidas, quanto melhor disseminada ela for, maior a chance de sucesso. Quando se tem uma interatividade com o telespectador a sua atenção é atraída. Nesse intuito, a interatividade e o dinamismo com a utilização de ferramentas atuais, usadas no dia a dia, é algo que pode atrair a atenção do usuário, fazendo-o ter interesse em acompanhar e participar de uma determinada noticia ou matéria, tonando-a mais facilmente acessível. 

Pensando nisso, vê-se a necessidade de um sistema onde é possível expor notícias referentes a instituição com facilidade contando também com a interatividade do espectador dessa notícia, seja ele por meio de curtidas ou por comentários na publicação, tudo em tempo real. Partindo da necessidade de melhor exposição das notícias e de uma forma de contato fácil e rápida com a comunidade acadêmica, de um sistema mais interativo, com suporte a gestão acadêmia e com uma forma simplificada de contato entre professor e aluno. 

Além disso, a comunicação entre professor e turma é feita geralmente por meio físico, e-mail ou necessitando de outros \textit{softwares} complementares, sendo necessário o professor ter e-mails individuais de cada aluno ou de turma para encaminhar qualquer notícia e isso acaba sendo ruim tanto para os professores, quanto para os alunos. 

Então surge a ideia do SID, onde é possível a melhor interatividade do espectador com as publicações acadêmicas, envio de mensagens para turmas específicas, além do administrador ter uma maior facilidade de criar e editar publicação vinculadas a redes sociais, integrando vários serviços em um único.

\section{Proposta}
Usando uma estrutura cliente-servidor, sendo o sistema SID usado como base e com a união do conceito de sinalização digital e marketing digital, a proposta é fazer com que o sistema apresente conteúdos referentes ao IFB e essas informações tenham integração com o Facebook, apresentando postagens e comentários devidamente moderados em tempo real nas  telas espalhadas pelos Câmpus Taguatinga do Instituto Federal de Brasília ou nos dispositivos móveis de cada pessoa. 

Além disto, na versão para dispositivos móveis, o docente poderá enviar informações e avisos distintos para cada aluno ou turma, através de um login com a matricula cadastrada no SGA (Sistema de Gestão Acadêmica) do Câmpus.

\daniel{Não conseguimos integração ao SGA, no entanto, é interessante falar que nós emulamos o SGA consumindo uma API fictícia, que busca simular este sistema de gestão acadêmica. Em um futuro próximo será possível a integração ao SGA sem muito esforço. É muito importante deixar bem claro o que fizemos e o que não fizemos aqui na Proposta.}
\section{Justificativa}
Atualmente, o IFB utiliza o seu perfil do Facebook e sua página oficial como principal forma de veiculação das notícias referentes a informações da instituição, sejam elas notícias de eventos ou institucionais. Para realizar a criação de uma nova notícia é necessário o administrador acessar cada página e realizar uma postagem independente em cada uma delas. 

Além de ser uma tarefa não trivial para se realizar todos os dias ou sempre que necessário, a interatividade entre os usuários da página e a
publicação está restrita a necessidade do usuário acessar a página e posteriormente acessar a notícia para então visualizar e talvez interagir. Não somente pela falta de praticidade, mas também pela baixa divulgação da página, a disseminação das notícias acabam sendo prejudicadas e por muitas vezes obtendo poucos acessos.

O sistema proposto, visa proporciona a integração dos meios usados atualmente para apresentação das informações referentes ao Câmpus, além da inclusão de outros meios. Com o uso do sistema proposto, será possível apresentar as mesmas informações
em telas espalhadas por locais de maior movimento do Câmpus, na pagina do Facebook
da instituição e até mesmo em dispositivos móveis dos alunos ou professores. Além disso, será possível uma melhor integração entre as notícias publicadas e os telespectadores, pois o sistema contará com a exibição em tempo real de comentários feitos pelos usuários na publicação.

No sistema mobile, além da apresentação das notícias, os professores e os alunos terão acesso a uma nova funcionalidade, usando uma matricula e senha fictícia cadastrado no bancos de dados que simula o SGA, o professor poderá encaminhar notícias ou informações para as turmas em que ele leciona, enquanto os alunos poderão acessar a cada mensagem enviada pelo professor para a turma em que ele está cadastrado.

\section{Objetivos}
\subsection{Objetivos Gerais}
Com objetivo de uma melhor e fácil disseminação das informações e propagandas pertinentes ao IFB - Câmpus Taguatinga, o sistema deverá ser capaz de proporcionar objetividade e simplicidade nas informações a serem repassadas. Além de painéis instalados pelo Câmpus, ele deve ter a integração com as mídias sociais como o Facebook, unificando os atuais sistemas de comunicação do IFB.

Além das otimizações necessárias no sistema, é feito uma filtragem de comentários antes da exibição no sistema, esse filtro de comentários servirá para que não sejam apresentados comentários abusivos e que se tenha comentários mais propícios a reações positivas por partes dos telespectadores.

Com a versão mobile do sistema, o aluno poderá não só ter acesso as informações que serão publicadas de forma geral para o Câmpus, mas também a conteúdos específicos através de uma matricula fictícia do aluno ou professor que simula o sistema usado pelo IFB, no caso o SGA. 

O aplicativo \textit{mobile} contará com três telas, a primeira é para exibição das publicações, a segunda tela servirá para que o professor possa encaminhar mensagens para uma turma em que ele leciona ou para um aluno especifico, a terceira tela é a do aluno, onde ele poderá verificar todas as mensagens que a turma em que ele está matriculado possui.

\subsection{Objetivos Específicos}
	 \begin{itemize}
	\item Implementar um sistema para um âmbito mais acadêmico, para melhorar a disseminação de informações dentro do Câmpus.
	 	
	\item Melhorias do sistema usado como base, o SID \cite{sobrinho2017}.
	
	\item Aprimorar o uso da ferramenta Graph API para uma melhor integração do sistema com o Facebook, recuperando mensagens, curtidas entre outras informações que venham a ser necessárias.
	
	\item Integrar o sistema com outras mídias sociais.
	
	\item Implementação de uma versão mobile do sistema, para possíveis consultas ou exibição do conteúdo, tornando a exibição das informações multiplataforma, exibindo-a em painéis, TVs, paginas de Internet ou celulares.
	
	\item  Implementar um sistema em que os docentes possam trocar mensagens com alunos das turmas em que ele leciona.
	\end{itemize}
\section{Metodologia}
Partindo da pesquisa descritiva, será descrito os procedimentos e passos que foram seguidos e usados para obtenção do resultado desejado.
	
A revisão de bibliografia é usada como meio de direcionamento do trabalho, usando comparações entre ferramentas desenvolvidas com o proposito principais de sinalização e \textit{marketing} digital, partindo de tais soluções com o objetivo de avaliar os pontos negativos tendo como base as necessidades do Câmpus e então juntar ao processo de desenvolvimento os elementos que forem selecionados como principais e que são responsáveis por efetivar a disseminação da informação ao sistema de forma descentralizada e com o auxílio de ferramentas utilizadas no contexto WEB.
	 
Usando o SID como sistema base, uma estrutura cliente-servidor e conexão a \textit{Internet} , será implementado no sistema as interações com as redes sociais. As informações serão apresentadas em multiplataforma que podem ser televisores, painéis, paginas web ou celulares, essas informações podem ser alteradas acessando ao servidor com o sistema instalado e conectado a Internet. Após serem criadas ou modificadas, as publicadas poderão ser transmitidas e acessadas pelos clientes em distintas plataformas ao mesmo tempo.

A metodologia presente neste trabalho está direcionada aos aspectos específicos do desenvolvimento de ferramentas computacionais com o intuito de melhoria nos processo de comunicação e veiculação de informações através de varias plataformas, sejam elas \textit{mobile}, WEB ou painéis. Para a versão \textit{mobile} do sistema, será usado um \textit{framework} de desenvolvimento especifica para a plataforma.

\section{Organização}
