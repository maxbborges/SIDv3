\chapter[Introdução]{Introdução}
Apesar de muito utilizada, a definição exata de mídia é difícil de ser explicada. De acordo com \cite[p.49]{guazina2007}, o uso predominante do termo ``mídia'' é para representar um conjunto de meios de comunicação, que se representa por um conjunto em massa como jornais, televisão, rádio, cinema e \textit{Internet}.

Sendo a mídia, talvez, responsável por introduzir mudanças comportamentais e comerciais nas mais diferentes sociedades, seja ela por informações presentes em canais de televisão, outdoors ou até mesmo panfletos, em ambientes públicos ou privados, \cite[p.3]{escobar2007} propõe que a mídia é capaz de redefinir ``o modo como o homem se comunica e se relaciona com os semelhantes.''

Para se ter uma ideia do poder que a mídia possui, para \cite{silva2007}, quando a população não tem acesso a outras fontes de informações, as notícias e mensagens veiculadas pelas mídias são muitas vezes vistas como verdades inquestionáveis. \cite[p.53]{guazina2007} aponta que os meios de comunicação são vistos como potenciais construtores de conhecimento e formadores de compreensão sobre mundo e política.

Ainda sobre a influência da mídia, para \cite[p.54]{hjarvard2012}, surge o conceito de midiatização. Para ele, a midiatização é um termo usado para caracterizar a influência da mídia e a coloca também como ``um processo contínuo em que os meios alteram as relações e o comportamento humano e, assim, alteram a sociedade e a cultura''. 

Com o passar dos anos, a mídia foi obtendo novas formas, as divulgações de forma estática e mais tradicionais (revistas, jornais e canis de televisão) foram deixando de serem os meios mais eficientes de se expor um conteúdo ou propaganda. Para \cite{meditsch2001} o advento da \textit{Internet} e sua popularização trouxeram uma ameaça para estes meios, então foi necessário que novas formas de expor conteúdos fossem pensadas e desenvolvidas em conjunto com essa nova ferramenta.

Nos meios de comunicação em massa sempre estão presentes os mais diversos tipos de propagandas e divulgações, o \textit{marketing} é responsável por criar e melhorar as ferramentas que tem como objetivo influenciar pessoas a adquirir ou aderir determinado produto ou serviço. Assim como a mídia, novas formas e ferramentas de \textit{marketing} também foram surgindo, ampliando os meios por onde as informações são repassadas. 

Para a ampliação, foi necessário que os caminhos da publicidade e da tecnologia se convergissem para se então definir um novo conceito de \textit{marketing}, o \textit{marketing} digital. O \textit{marketing} digital conta com novas tecnologias de comunicação que para \cite[p.2]{escobar2007}, coloca a interatividade em evidencia, então, a utilização de novas ferramentas mais interativas com seu receptor tornam a leitura menos monótona e passível de atingir uma maior atenção do espectador de forma mais consistente.  

As vantagens para a utilização da interatividade estão presentes em diversas formas, como por exemplo, para \cite[p.4]{escobar2007}, transmissões ao vivo por rádio ou televisão permitem o acesso a um dado acontecimento no exato momento em que ele acontece, mas quando se tem o advento da Internet coloca-se a possibilidade de interação com a informação que é recebida, isso quase que instantaneamente, onde os integrantes atuam de forma simultânea, comentando ou opinando sobre aquele determinado assunto. Para \cite{deuze2002},o surgimento da \textit{Internet} traz a possibilidade do público ``responder, interagir ou mesmo customizar certas histórias''. 

A evolução do \textit{marketing} digital trouxe consigo a união das mídias sociais e dispositivos móveis. Isso permite não somente que as informações circulem fora de ambientes específicos, mas também que os receptores das informações transmitidas possam interagir quase que em tempo real com o conteúdo que é apresentado. Para \cite{santos2014}, no contexto do novo cenário da textit{WEB} é necessário um marketing em ambiente digital.

Para \cite[p.7]{machado2010}, o rápido crescimento das organizações juntamente com a Internet as obrigou a aderirem novos conceitos de gestão e apresentação das informações, usando não somente os veículos de comunicação e as mídias digitais. Pensando na maior abrangência, surge o conceito de sinalização digital, que para \cite[p.37]{machado2010}, consiste na transmissão de conteúdo via Internet, onde essa mesma informação pode se ter receptores nos mais diversos locais, independente de cidade, estado ou país, com o uso de painéis e televisores apresentando informações e propagandas de forma dinâmica, em diversos pontos e com a possibilidade de gerenciá-las remotamente de acordo com a necessidade. 

As mídias sociais também possuem seu importante papel para disseminação de uma informação ou conteúdo e vem se tornando uma das ferramentas mais atraentes para divulgações. De acordo com \cite{rosa2010} isso se deu pelo seu grande uso e por elas se tornarem a extensão da vida real. Não apenas por ser um dos meios mais acessados atualmente, mas também por conta da maior facilidade de interações dos espectadores, usuários e empresas que para \cite{rosa2010} esse tipo de mídia permite 
às empresas encontrarem a melhor solução para seus objetivos. 

Com o passar dos anos os dispositivos móveis estão sendo cada vez mais utilizados pelas pessoas, isso vem atraindo cada vez mais o seu foco como ferramenta para divulgação de informações. A pesquisa da \cite{fgv2017} aponta que até outubro de 2017 teria-se 208 milhões de aparelhos ativos no Brasil e que até maio de 2018 serão cerca de 220 milhões de \textit{smartphones} ativos, mais de 1 aparelho por habitante. 

Não somente pelo grande número de dispositivos, mas também pela grande integração com as redes sociais que segundo a \cite{forbes2016}, pesquisas feitas pela agência eMarketer afirmavam que até o final de 2016, 42\% da população da América latina iriam acessar regularmente as redes sociais, em 2016 estimava-se que 74\% da população do Brasil que fizessem uso da internet no pais teriam uma conta no Facebook, se comparado ao mundo todo, o número chega a 2 bilhões de usuários em 2017. 

O Instituto Federal de Educação Ciência e Tecnologia de Brasília - Campus Taguatinga (IFB), para divulgação de notícias e as mais variadas informações, se utiliza principalmente de suas páginas \textit{WEB} e Facebook. Para o uso desses meios, é necessário que os administradores façam publicações independentes para cada uma das páginas, onerando o trabalho do mesmo, além disso, as páginas não são bem divulgadas, muitas vezes os estudantes e visitantes nem ficam sabendo das notícias que lá são publicadas. Percebe-se que os professores contam com poucas formas fáceis e intuitivas para repassar informações a seus alunos.

Essa falta de integração entre os veículos institucionais, que atuam de forma independente acabam degradando a qualidade e a disseminação da notícia, não somente pela falta de um sistema integrado, mas também pela complicada interação e acesso dos usuários com as atuais mídias, o que acaba afetando o interesse de acompanhá-la.

O Sistema Integrado de Divulgação de Informações do IFB versão 3 (SIDv3) oferece uma maior visibilidade das notícias, sendo possível, por meio painéis espalhados pelo campus, uma melhor interação da comunidade com as notícias, apresentando nos painéis os comentários que foram publicados nas mídias sociais, além de uma melhor forma de comunicação entre professor e aluno, oferecendo um aplicativo \textit{mobile} que simula um sistema de comunicação institucional, integrado ao Sistema de Gestão Acadêmica (SGA).

\section{Motivação}
A atenção do espectador a uma determinada informação que está sendo apresentada é algo crucial para o sucesso das notícias que estão sendo exibidas, quanto melhor disseminada ela for, maior a chance de sucesso. Quando se tem uma maior troca com o telespectador a sua atenção é atraída. Nesse intuito, a interatividade e o dinamismo com a utilização de ferramentas atuais, usadas no dia a dia, é algo que pode despertar e manter a atenção do usuário, fazendo-o ter interesse em acompanhar e participar de uma determinada noticia ou matéria, tornando-a facilmente acessível e assimilável. 

Atualmente, o IFB utiliza o seu perfil do Facebook e sua página oficial é a principal forma de veiculação das notícias referentes a informações da instituição, sejam elas notícias de eventos ou institucionais. Para realizar a criação de uma nova notícia é necessário o administrador acessar cada página e realizar uma postagem independente em cada uma delas. 

Então é preciso ter a melhor exposição das notícias, em uma forma de contato fácil e rápida com a comunidade acadêmica, de um sistema mais interativo, com suporte a gestão acadêmica e com uma forma simplificada de contato entre professor e aluno. Além disso, a comunicação entre professor e turma é feita geralmente por meio físico, e-mail ou necessitando de outros \textit{softwares} complementares, sendo necessário o professor ter e-mails individuais de cada aluno ou de turmas para encaminhar qualquer notícia, o que acaba se tornando ruim tanto para os professores, quanto para os alunos. 

Pensando nisso, vê-se a indispensabilidade de um sistema onde é possível expor notícias referentes à instituição com facilidade, contando com uma melhor agilidade de acesso e interatividade do espectador dessa notícia, por meio de comentários na publicação e apresentação em tempo real. Nota-se que existe a carência de um sistema \textit{mobile} que ofereça suporte que possuam as características mencionadas acima, incluindo a possibilidade de reciprocidade entre professores e alunos, com a oportunidade  de troca de mensagens entre eles.

\section{Proposta}
Com uso da arquitetura cliente-servidor e tendo o  Sistema Inteligente de Divulgações do IFB em sua versão 2 (SIDv2) implementado por \cite{sobrinho2017} como base, é proposto a elaboração da terceira versão. Com o uso dos conceitos de sinalização digital e marketing digital, a proposta é fazer com que o sistema apresente conteúdos referentes ao IFB e essas informações tenham integração com o Facebook, apresentando postagens e comentários devidamente moderados em tempo real nas telas espalhadas por locais de maior movimento do Câmpus Taguatinga do Instituto Federal de Brasília ou nos dispositivos móveis de cada pessoa. 

O sistema proposto visa proporcionar a integração dos meios usados atualmente para apresentação das informações referentes ao Câmpus, além da inclusão de outros meios. O programa fará a integração entre a página do Facebook da instituição, televisões espalhados pelo campus e dispositivos móveis dos alunos ou professores.

Portanto, a ideia do SID, é prover a melhor troca do espectador com as publicações acadêmicas, e o administrador ter uma maior facilidade de criação e edição das publicações vinculadas as redes sociais, integrando vários serviços em um único.

Na versão para dispositivos móveis, além da apresentação das notícias, os professores e os alunos terão acesso a uma nova funcionalidade, o docente poderá enviar informações e avisos distintos para cada aluno ou turma, enquanto os alunos poderão acessar a cada mensagem enviada pelo professor para a turma em que ele está cadastrado, Isso se dará através de um login usando uma matricula e senha fictícia cadastrado no bancos de dados que simula plataformas acadêmicas já existentes, onde não será possível o uso de dados reais por restrições de acesso as essas plataformas.

\section{Objetivos}
Levantar os principais requisitos de um sistema de sinalização e marketing digital integrado a rede social Facebook.

Estudar e detalhar a documentação da API do Facebook para integração de sistemas.

Definir e implementar o Sistema Integrado de Divulgações do IFB (SIDv3).

Implementar aplicativo \textit{mobile} com funcionalidades adicionais, além das utilidades providas do SIDv3. Realizando o consumo de um API fictícia do Sistema de Gestão Acadêmica (SGA) visando uma futura integração à plataforma existente quando for disponibilizada.

\subsection{Objetivos Específicos}
	 \begin{itemize}
	\item Estudo e utilização de frameworks necessários para implementação do sistema, como ZEND, Doctrine, Cordova e Framework7.
	 	
	\item Disponibilização de uma API REST para interoperabilidade do SIDv3 com outros sistemas.
	
	\item Sugestão de uma proposta de implantação viável no Campus Taguatinga utilizando computadores Raspberry Pi.
	\end{itemize}
	
\section{Metodologia}
A revisão de bibliografia será feita como meio de direcionamento do trabalho, onde serão usadas comparações entre ferramentas desenvolvidas com o propósito principal de sinalização e \textit{marketing} digital, partindo de tais soluções com o objetivo de avaliar os pontos negativos baseando-se nas necessidades do Campus e então juntar ao processo de desenvolvimento os elementos que forem selecionados como principais e que são responsáveis por efetivar a disseminação da informação ao sistema de forma descentralizada e com o auxílio de ferramentas utilizadas no contexto \textit{WEB}.

O estudo da documentação da Graph API e de suas ferramentas, tais como a Graph API Explorer, viabilizará a integração do sistema com a rede social Facebook. Os instrumentos serão usados para realização de testes práticos de funcionalidade como a recuperação de dados da rede social para integração com o sistema e com o estudo da documentação e serão analisadas possíveis implementações dessas funcionalidades.
	 
Ao utilizar o SIDv2 como sistema base e com auxílio das operacionalidades  disponíveis na Graph API, será implementado no sistema as interações com as redes sociais. As informações serão apresentadas em multiplataformas que podem ser televisores, painéis, páginas \textit{WEB} ou celulares. Essas informações podem ser alteradas acessando ao servidor com o sistema instalado e conectado a Internet. Após serem criadas ou modificadas, as publicações poderão ser transmitidas e acessadas pelos clientes em distintas plataformas ao mesmo tempo.

A metodologia presente neste trabalho está direcionada ao desenvolvimento de ferramentas que possuem a finalidade de melhoria no processo de comunicação e veiculação de informações e notícias em diferentes plataformas. 

Todo o sistema, incluindo o \textit{mobile}, seguirá o padrão de desenvolvimento ágil, com metodologia SCRUM, sendo definido sprints semanais, comumente marcada as quartas feiras para definição das funcionalidades a serem desenvolvidas ou melhoradas. 

\section{Organização do documento}
O Capítulo 2, de Trabalhos Relacionados, tem como objetivo situar o leitor sobre ferramentas que possuem conceitos que se assemelham com o SID, expondo como elas funcionam a fim de realizar um comparativo entre as funcionalidades que apresentadas e as que são requeridas.

A diante, no Capítulo 3, Referencial Teórico, tem como propósito apresentar ao leitor de forma detalhada cada conceito e ferramenta abordada no decorrer do documento.

No Capítulo 4, Graph API, é apresentado um resumo detalhado das principais funcionalidades que a ferramenta Graph API oferece. Neste capítulo também são colocados alguns exemplos de aplicação destas diferentes funcionalidades.

A explicação detalhada sobre os aspectos intrínsecos do sistema, como a estrutura do SID e de como ele realiza as suas ações em conjunto com a Graph API apresentada no Capítulo 4 são expostas no Capítulo \ref{cap:sid}.

Os resultados obtidos estão expostos no Capítulo 6, que aborda de forma resumida todo o resultado final que foi obtido no desenvolvimento do sistema, além das dificuldades encontradas.

Por fim, o Capítulo 6, considerações finais, repassa ao leitor os benefícios de se usar o SID e o horizonte de possibilidades que o SID viabilizou.

