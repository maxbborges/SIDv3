\chapter[Introdução]{Introdução}
Apesar de muito usada, a definição exata de mídia é complicada de ser explicada. \marcar{De acordo com} \cite[p.49]{guazina2007} \riscar{coloca que até 2004}{}, o uso predominante do termo ``mídia'', era pra representar um conjunto de meios de comunicação, representado por meios de comunicação em massa.

\daniel{Simplifique este parágrafo. Fale que mídia, de acordo com este autor, se resumia a meio de comunicação massivos, tais como A, B e C. Além disso preste mais atenção na concordância verbal.}

Sendo a mídia, talvez, responsável por introduzir mudanças comportamentais e comerciais nas mais diferentes sociedades, seja ela por informações presentes em canais de televisão, outdoors ou até mesmo panfletos, em ambientes públicos ou privados, \cite[p.3]{escobar2007} propõe que a escrita, a mídia e até mesmo o computador são capazes de redefinir 
\begin{quote}\riscar{``o modo como o homem se comunica e se relaciona com os semelhante.''}{}  \end{quote} 

\daniel{Coloque isso junto ao parágrafo.}


Para se ter uma ideia do poder que a mídia tem, para \cite{silva2007}, quando a população não tem acesso a outras fontes de informações, as notícias e mensagens veiculadas pelas mídias são muitas vezes vistas como verdades inquestionáveis. 

Quando se é abordado questões como o papel e a influência da mídia, para \cite[p.54]{hjarvard2012}, surge o conceito de midiatização. Sendo a midiatização o termo usado para caracterizar a influência da mídia, este mesmo autor coloca a midiatização como \begin{quote}``um processo contínuo em que os meios alteram as relações e o comportamento humanos e, assim, alteram a sociedade e a cultura''.\end{quote} 

\daniel{Mais uma vez, coloque isso no parágrafo. Evite colocar citações diretas.}

A divulgação de forma estática e de formas mais tradicionais (revistas e jornais) já não são as formas mais eficientes de se expor um conteúdo ou propaganda. Para \cite[p.2]{escobar2007} novas tecnologias de comunicação colocaram a interatividade em evidencia, então, a utilização de novas ferramentas mais interativas com seu receptor, torna a leitura menos monótona e passível de atingir uma maior atenção do espectador de forma mais consistente.

\daniel{Transmissõe\textbf{S} permite\textbf{M}}

\daniel{novas ferramenta\textbf{S} (...) torna\textbf{M}}

No intuito de promover ou demonstrar algo, o \textit{Marketing} usa a mídia para se difundir e assim como a mídia, com o passar dos anos, os caminhos da publicidade e da tecnologia tem \riscar{se}{} convergido para se definir um novo tipo de \textit{Marketing}, o \textit{Marketing Digital}.

\daniel{Não está claro quais são os benefícios do Marketing Digital e sua definição precisa. Utilize aquela referência que eu te mandei para você se embasar. Apesar de estar em inglês é uma referência mais nova do que as que você está utilizando.}


Para \cite[p.4]{escobar2007}, transmissões ao vivo por rádio ou televisão permite o acesso a um dado acontecimento no exato momento em que ele acontece, mas quando se tem o advento da Internet coloca-se a possibilidade de interação com a mensagem que é recebida, isso quase que instantaneamente, onde os integrantes atuam simultaneamente. Para \cite{deuze2002}, a Internet e o webjornalismo traz a possibilidade do público ``responder, interagir ou mesmo customizar certas histórias''. 

\daniel{transmissões (...) permitem }

\daniel{A internet e o webjornalismo traze\textbf{m}}

\daniel{Retirar as aspas}


Pensando não só na maior abrangência, mas também na interatividade, a expansão do conceito de  digital com a união das mídias sociais e o uso de dispositivos móveis, permite não somente que as informações circulem fora de ambientes específicos, mas também que os receptores das informações transmitidas possam interagir quase que em tempo real com o conteúdo que é apresentado. Para \cite{santos2014}, no contexto do novo cenário da web é necessário um marketing em ambiente digital.

Para \cite[p.7]{machado2010}, o rápido crescimento das organizações juntamente com a Internet obrigou elas a aderir novos conceitos de gestão. Pensando na maior abrangência, surge o conceito de sinalização digital, que para \cite[p.37]{machado2010}, consiste na transmissão de conteúdo via Internet, onde essa mesma informação pode se ter receptores no mais diversos locais, independente de cidade, estado ou país com o uso de painéis e televisores apresentando informações e propagandas de forma dinâmica, nos mais diversos pontos e com a possibilidade de \riscar{gerencia-las}{gerenciá-las} remotamente de acordo com a necessidade. 

\daniel{Creio que a sinalização digital é uma coisa mais rústica, não tão sofisticada quanto o Marketing Digital. Tivemos essa discussão anteriormente e ainda tenho essa impressão. Você resolve esse}

Segundo a \cite{forbes2016}, pesquisas feitas pela agencia eMarketer afirmavam que até o final de 2016, 42\% da população da América latina iriam acessar regularmente as redes sociais, em 2016 estimava-se que 74\% da população do Brasil que usasse a internet no pais teriam uma conta no Facebook, se comparado o mundo todo, o número chega a 2 bilhões de usuários em 2017. O uso das redes para disseminação de uma informação ou conteúdo vem se tornando uma das ferramentas mais atraentes para divulgações. Não apenas por ser um dos meios mais acessados atualmente, mas também por conta da maior facilidade de interações dos espectadores, usuários e empresas. 

A pesquisa da \cite{fgv2017} aponta que até outubro de 2017 teria-se 208 milhões de aparelhos ativos no brasil e que até maio de 2018 terá cerca de 220 milhões de de \textit{smartphones} ativos, mais de 1 aparelho por habitante. 

\daniel{Este parágrafo precisa se correlacionar com os outros. Ele está muito solto.}

FALAR SOBRE OS MEIOS DE COMUNICAÇÃO DO IFB E O QUE O SID PODE MUDAR

\daniel{Esta Introdução está bem desconexa. Cada parágrafo fala sobre um assunto meio descorrelacionado ao outro. Tente integrar mais os parágrafos.}

\section{Motivação}
A atenção do espectador a uma determinada informação que está sendo apresentada é algo crucial para o sucesso das notícias que estão sendo exibidas, quanto melhor disseminada ela for, maior a chance de sucesso. Quando se tem uma interatividade com o telespectador a sua atenção é atraída. Nesse intuito, a interatividade e o dinamismo com a utilização de ferramentas atuais, usadas no dia a dia, é algo que pode atrair a atenção do usuário, fazendo-o ter interesse em acompanhar e participar de uma determinada noticia ou matéria, tonando-a mais facilmente acessível. 

Pensando nisso, vê-se a necessidade de um sistema onde é possível expor notícias referentes a instituição com facilidade contando também com a interatividade do espectador dessa notícia, seja ele por meio de curtidas ou por comentários na publicação, tudo em tempo real. Partindo da necessidade de melhor exposição das notícias com uma forma de contato fácil e rápida com a comunidade acadêmica, de um sistema mais interativo, com suporte a gestão acadêmia e com uma forma simplificada de contato entre professor e aluno. 

Além disso, a comunicação entre professor e turma é exclusivamente por meio físico ou e-mail, sendo necessário o professor ter e-mails individuais de cada aluno ou de turma para encaminhar qualquer notícia e isso acaba sendo ruim tanto para os professores, quanto para os alunos. 

\daniel{Existem outros meios, mas nenhum integrado ao sistema de gestão acadêmica.}

Então surge a ideia do SID, onde é possível a melhor interatividade do espectador com as publicações acadêmicas, envio de mensagens para turmas específicas, além da facilidade de publicação do administrador do sistema.

\daniel{E a interação via redes sociais? Você não falou nada disso aqui.}

\section{Proposta}
Usando uma estrutura cliente-servidor \marcar{e} utilizando o sistema SID como base e com a união do conceito de sinalização e marketing digital, a proposta é fazer com que o sistema apresente conteúdos referentes ao IFB e essas informações \riscar{tenha}{tenham} integração com \riscar{as redes sociais, incluindo o Facebook,}{o Facebook},\riscar{Realizando uma mineração de dados}{} e apresentar postagens e comentários devidamente moderados em tempo real nas  telas espalhadas pelos Câmpus Taguatinga do Instituto Federal de Brasília ou nos dispositivos móveis de cada pessoa. 

\marcar{Além disto}, na versão para dispositivos móveis, o servidor poderá enviar informações e avisos distintos para cada aluno, turma ou professor, através de um login com a matricula cadastrada no SGA (Sistema de Gestão Acadêmica) do Câmpus.

\daniel{Servidor máquina ou o docente?}

\daniel{Não conseguimos integração ao SGA, no entanto, é interessante falar que nós emulamos o SGA consumindo uma API fictícia, que busca simular este sistema de gestão acadêmica. Em um futuro próximo será possível a integração ao SGA sem muito esforço. É muito importante deixar bem claro o que fizemos e o que não fizemos aqui na Proposta.}
\section{Justificativa}
Atualmente, o IFB utiliza excepcionalmente o seu perfil do Facebook e sua página
oficial para realização das postagens referentes a informações da instituição, seja ela notí-
cias de eventos ou institucionais, para isso é necessário o administrador acessar cada página
e realizar uma postagem independente em cada uma delas. 

Além de ser uma tarefa não
trivial para se realizar todos os dias, a interatividade com os usuários da página com a
publicação está restrita a necessidade do usuário achar a noticia, acessar a pagina, acessar
a publicação para então visualizar e talvez interagir, não somente pela falta de praticidade,
mas também pela baixa divulgação da pagina, a disseminação da noticia acaba
sendo ruim e por muitas vezes tendo poucos acessos.



O sistema proposto, visa proporciona a integração dos meios usados atualmente
para apresentação das informações referentes ao Câmpus, além da inclusão de outros
meios. Com o uso do sistema proposto, será possível apresentar as mesmas informações
em telas espalhadas por locais de maior movimento do Câmpus, na pagina do Facebook
da instituição e até mesmo por um dispositivo móvel pessoal do aluno ou professor. Além
disso, será possível uma melhor integração entre as notícias publicadas e os telespectadores,
pois o sistema contará com a exibição em tempo real de comentários feitos pelos
usuários na publicação.

No sistema mobile, além da apresentação das noticias, os professores e os alunos
terão acesso a uma nova pagina, usando matricula e senha cadastrado no SGA, o professor
poderá encaminhar noticias ou informações para as turmas em que ele leciona, enquanto
os alunos terão acesso a cada mensagem enviada pelo professor para a turma em que ele
está cadastrado.

\daniel{Não fizemos isso}

\section{Objetivos}
\subsection{Objetivos Gerais}
Com objetivo de diminuição da carência e aumento da facilidade de disseminação
das informações e propagandas pertinentes ao IFB - Câmpus Taguatinga, o sistema deverá
ser capaz de proporcionar objetividade e simplicidade nas informações a serem repassadas.
Além de painéis instalados pelo Câmpus, ele deve ter a integração com as mídias sociais
como o Facebook, unificando os sistemas de comunicação do IFB.

\daniel{Carência do que?}

Além das otimizações necessárias no sistema, será usada também técnicas de mineração
de dados, para que seja possível selecionar conteúdos apropriados para inserção
e publicação no sistema, filtrando informações e comentários que sejam mais propícios a
ter reações positivas por partes dos telespectadores. Com a versão mobile do sistema, o
aluno poderá não só ter acesso as propagandas que serão publicadas de forma geral para o
Câmpus, mas também a conteúdos específicos através da matricula do aluno ou professor
vinculado ao SGA, uma tela para que o professor possa encaminhada mensagens para
uma turma em que ele leciona ou para um aluno especifico de uma determinada turma e
outra para o aluno, onde ele poderá verificar todas as mensagens que a turma em que ele
está matriculado possui.

\daniel{Não foi implementada a integração com o SGA.}


\subsection{Objetivos Específicos}
	 \begin{itemize}
	\item Implementar um sistema para um âmbito mais acadêmico, para melhorar a disseminação de informações dentro do Câmpus.
	 	
	\item Melhorias do sistema usado como base, o SID \cite{sobrinho2017}.
	
	\item Aprimorar o uso da ferramenta Graph API para uma melhor integração do sistema com o Facebook, recuperando mensagens, curtidas entre outras informações para apresentação.
	
	\item Integrar o sistema com outras mídias sociais como o twitter.
	
	\item Implementação de uma versão mobile do sistema, para possíveis consultas ou exibição do conteúdo, tornando a exibição das informações multiplataforma, exibindo-a em painéis, TVs, paginas de Internet ou celulares.
	
	\item  Integração da versão mobile como o sistema SGA.
	\end{itemize}
\section{Metodologia}
Partindo da pesquisa descritiva, será descrito os procedimentos e passos que foram seguidos e usados para obtenção do resultado desejado.
	
A revisão de bibliografia é usada como meio de direcionamento do trabalho, usando comparações entre ferramentas desenvolvidas com o proposito principais de sinalização e marketing digital, partindo de tais soluções com o objetivo de avaliar os pontos negativos tendo como base as necessidades do Câmpus e então juntar ao processo de desenvolvimento os elementos que forem selecionados como principais e que são responsáveis por efetivar a disseminação da informação ao sistema de forma descentralizada e com o auxílio de ferramentas utilizadas no contexto WEB
	 
Usando o SID como sistema base, uma estrutura cliente-servidor e conexão a \textit{Internet} , será implementado no sistema as interações com as redes sociais. As informações serão apresentadas em multiplataforma que podem ser televisores, painéis, paginas web ou celulares, essas informações podem ser alteradas acessando o servidor, um \textit{Raspberry Pi}, com o sistema instalado e conectado a Internet. Após serem criadas ou modificadas, as publicadas poderão ser transmitidas e acessadas pelos clientes em distintas plataformas ao mesmo tempo.
	
\daniel{Não insista mais neste Raspberry Pi, não é a ideia do trabalho.}

A metodologia presente neste trabalho está direcionada aos aspectos específicos	do desenvolvimento de ferramentas computacionais com o intuito de melhoria nos processo de comunicação e veiculação de informações através de varias plataformas, sejam elas mobile, web ou painéis. Para a versão mobile do sistema, será usado um framework de desenvolvimento especifica para a plataforma.

\daniel{Qual a metodologia de desenvolvimento de software? SCRUM?}

\section{Organização}
