\chapter[Referencial Teórico]{Referencial Teórico}
Este capítulo tem o propósito de apresentar ao leitor os principais conceitos e ferramentas utilizados para desenvolvimento do sistema. Apresentando de forma superficial o que é cada uma e sua finalidade. 

\section{Marketing Digital}
O \textit{marketing} já passou diversas modificações e melhorias com o passar dos anos, começando com jornais e revistas, passou por panfletos, pôsteres e \textit{outdoors}, teve-se também os rádios, televisões e telefones até se atingir hoje o \textit{marketing} digital. Ele se aproveita da popularização das redes sociais e da crescente facilidade de acesso a \textit{Internet} para aproximar produtores e vendedores as pessoas, que são consumidores, integrando ou expondo produtos ou serviços de maneira simplificada no cotidiano das pessoas.

O advento da textit{Internet}, obrigou o textit{marketing} a ser mais digital, obrigando ele a se inserir em novos contextos, \cite{canto2017} afirma que o textit{marketing} digital é uma \begin{quote} ``premissa básica para aqueles que desejam estreitar as relações com os clientes''\end{quote} criando melhores estratégias de textit{marketing} e inserindo-as em um ambiente digital, integrando as redes sociais para se ter uma ampla divulgação.

\cite[p.2]{ryan2016} relata que o textit{marketing} digital em 2016 se tornou uma grande fonte de negócios, onde com a revolução online trouxe uma nova onda de consumidores, aqueles que integram a tecnologia a sua vida cotidiana. Para ele, publicidade tem tudo a ver com influenciar pessoas e para isso propagandas e tecnologias vem se convergindo cada vez mais para se criar um panorama de textit{marketing}. A tecnologia e a publicidade se tornaram parte do cotidiano das pessoas.

\cite{santos2014} comenta que em um primeiro momento as estratégias de mercado eram voltadas para meios tradicionais como TVs, rádios, jornais, revistas e outros. Com a popularização da textit{Internet} foi necessário a busca por outros meios de para se realizar
o textit{marketing}, chegando ao textit{marketing} digital que é o uso das mídias sociais ou de ferramentas cotidianas, como a textit{Internet}, para exposição dos produtos.

O textit{marketing} tem por objetivo, como cita \cite{ryan2016}, persuadir as pessoas a tomar determinadas ações que são desejadas pela divulgação, por exemplo, influenciar alguém a escolher uma determinada marca ou objeto, apenas como aquela apresentação.

Na opinião de \cite{torres2000}, as mídias sociais são páginas de textit{Internet} onde os usuários são ao mesmo tempo produtor e consumidor das informações que nelas são criadas, sendo possível a criação e compartilhamento dessas informações. Ainda para eles, essas mídias receberam esse nome por serem livres e terem a possibilidade de
colaboração e interação de todos que nelas estão, além de ser um meio de transmissão das informações e conteúdos.

O textit{marketing} digital faz uso também das mídias sociais para uma melhor apresentação e disseminação do conteúdo desejado. Como apresenta \cite{torres2000}, a facilidade de exposição e compartilhamento que as redes sociais oferecem, as tornam mais atrativas quando se quer expor algo.

\section{Sinalização digital}
Para \cite{munari2006} a sinalização pode ser feita das mais diversas formas, desde códigos, sinais, luzes, canetas, lápis ou até mesmo por imagem. Ela deve ser melhorada de modo a atingir o objetivo proposto pelo criador, seja ela para propaganda, informativo ou entretenimento, mas sempre com o mesmo objetivo, obter a atenção do telespectador.

Em uma pesquisa com 91 telespectadores em 11 diferentes exibições realizada por \cite{muller2009}, usando painéis públicos e o conceito de sinalização digital, revela que os usuários consideram os anúncios chatos e por muitas vezes o ignoram. Quando o intuito é atrair a atenção do usuário, a interatividade por ser uma grande ferramenta.

Com o uso de programas para gerenciamento do conteúdo, \cite[p.31]{machado2010} explica que a a sinalização digital consiste na transmissão de conteúdo, via Internet, à televisões ou painéis dos mais diversos tamanhos e nos mais diversos locais, usualmente instalados em pontos estratégicos onde as pessoas costumam ficar algum tempo como bancos, ônibus, elevadores. Ele ainda apresenta diversas vantagens e modos de uso para a sinalização digital, em que cada empresa pode optar ou não por ter determinada funcionalidade, mas em geral as vantagens apresentadas são opções como a de atualização instantânea de conteúdo, agendamento de programação, painel para controle, anúncios diferenciados e divisão de tela com diferentes conteúdo.

\cite[p.11]{cintra2010} relaciona a sinalização digital como alternativa aos textit{outdoors}, que poluem visualmente grandes cidades. Usualmente colocadas em pontos estratégicos e de grande movimento, tem-se a grande vantagem da economia de tempo e dinheiro. A era da comunicação digital como apresenta a autora, coloca como vantagem a agilidade e a facilidade de troca do conteúdo contido nos painéis, por ser de forma digital e geralmente remota, além disso, é possível a inserção de novos recursos se comparados com outdoors, tais como movimentos nas imagens, efeitos digitais e som.

De acordo com \cite{mishima2016}, sistemas de sinalização
digital que possuem algum dispositivo eletrônico são amplamente usados para exibir informações dinâmicas, ao contrários de anúncios estáticos, que ficam por determinado ou muitas vezes logos tempos na mesma informação, o que acaba por muitas vezes, não atraindo a atenção ou interesse de quem os observa.

Entretanto, o uso da sinalização digital não fica restrito a uso externo, para propagandas, ela também pode ser usada para fins internos, para apresentar informações apenas para dentro de uma empresa. Além disso é possível a troca ou atualização de conteúdo em todos os pontos conectados ao mesmo tempo, economizando tempo e mão de obra para reajustar cada ponto específico.

\section{Arquitetura Cliente-Servidor}
Cliente/Servidor é uma arquitetura computacional que envolve requisições de serviços de clientes para servidores. \cite{cecin2005} explica que nesse paradigma, o servidor é responsável pela computação sobre os dados, enquanto o cliente lida apenas na camada de apresentação.

Uma definição para o que seria a arquitetura Cliente/Servidor, seria a existência de uma plataforma base para que as aplicações, onde um ou mais Clientes e um ou mais servidores, juntamente com o sistema operacional executem um processamento distribuído.

Em resumo, o cliente é um processo que interage com o usuário através de uma interface gráfica ou não, enquanto o servidor fornece um determinado serviço que fica disponível para todo Cliente que o necessita.

\section{Linguagens de programação}
Para \cite{sebesta2011}, existe uma lista de potenciais vantagens no estudo dos conceitos de linguagem de programação, entre elas está a da capacidade aumentada para expressar ideias. Ainda para \cite{sebesta2011}, para as pessoas que não possuem esse conhecimento é mais complicado criar conceitos de estruturas que elas não podem descrever ou expressar. 

Na lista feita por \cite{sebesta2011} aponta também o embasamento para escolher linguagens adequadas e a habilidade aumentada para aprender novas linguagens como parte vantagem do estudo desses conceitos.

Na linguagem de programação, metodologias de projeto, ferramentas de desenvolvimento e a própria linguagem de programação estão ainda em evolução, por esse motivo \cite{sebesta2011} considera o desenvolvimento de software uma profissão excitante, mas que obriga a se ter um aprendizado contínuo.

As linguagens de programação possuem vários conceitos, entre eles estão linguagens de programação estruturada, modular, orientada a objeto, orientada a aspecto e linear. Cada uma possui sua particularidade e seu grau de abstração que podem ser baixo nível, médio nível e alto nível.

Sendo o padrão seguido por diversas linguagens, a orientação a objeto é usada por diversas linguagens conhecidas, tais como PHP, java e C\#. Além do uso de classes e objetos, para \cite{ricarte2001}, um dos grandes diferenciais da orientação a objeto está o conceito de herança que facilita a extensão e o polimorfismo que permite selecionar funcionalidades de forma dinâmica.

As linguagens de baixo nível, são linguagens mais complexas para compreensão humana, onde o código é uma representação direta do código de máquina. O maior exemplo de linguagem de baixo nível é o Assembly.

As linguagens de médio nível, são linguagens intermediarias, que são abstratas, mas com uma facilidade maior de compreensão. Entre os exemplos de linguagem de médio nível estão o C e o C++.

Já as linguagens de alto nível são bem mais inteligíveis pelo ser humano, possuindo baixo nível de abstração. Entre os exemplos desse tipo de linguagem estão o Java e o PHP.

\subsection{PHP}
O PHP é uma das linguagens de programação mais populares no mundo, para \cite[p.2]{vaswani2010}, é a ferramenta  escolhida por milhões de pessoas no planeta para desenvolvimento de aplicações \textit{Web}. Isso porque o PHP é desenvolvido e mantido mundialmente por uma comunidade voluntária. 

Para \cite[p.2]{vaswani2010}, as razões pela popularização do PHP como linguagem de programação se dá pelo fato de não ser complicada de se aprender, escalável, de fácil obtenção e muito funcional com aplicações de terceiros. Além disso, desenvolvedores relatam grandes níveis de satisfação como a linguagem.

Além de ser uma linguagem que possui suporte a diversos \textit{frameworks} que auxiliam os desenvolvedores, possuindo compatibilidade com diferentes sistemas de banco de dados, ela ainda é livre e não necessita de licença para uso. Uma pesquisa da \cite{phpusage2018}, mostra que grandes sites como o do Facebook e Wikipedia foram desenvolvidos nessa linguagem, isso faz com que a linguagem seja mais utilizada, por possuir exemplos de sucesso.

\subsection{JavaScript - JS}
JavaScript - JS é uma linguagem de programação HTML e web. \cite{balduino2014} comenta que JS se faz presente nas mais diversas aplicações, em seu \textit{backend}, seja jogos, emulação de hardwares ou até como linguagem padrão de plataformas inteiras.

Uma aplicação com JS pode ser usada para invocações AJAX, onde é possível realizar diversos tipos de requisições de dados de um servidor.

\section{Banco de Dados}
Quanto maior é a quantidade dedados, maior é a necessidade de o uso de um banco de dados. O uso de papéis para armazenar informações podem necessitar de uma grande quantidade de espaço físico e de um grande controle, para que nada se perda.

Com o uso de um banco de dados, tem-se a possibilidade de se armazenar as mais diversas informações e criar uma relação entre cada uma delas para uma melhor organização.

Para \cite{elmasri2005}, é viável afirmar que os bancos de dados representam um papel crítico em quase todas as áreas que os computadores são utilizados. Considerando uma definição genérica usada por \cite{elmasri2005}, o banco é uma coleção de dados relacionados, onde os dados podem ser gravados e possuem um significado.

\section{Interface de Programação de Aplicativos - API}
A API(Application Programming Interface) é um instrumento que visa facilitar o desenvolvimento e a vida dos programadores. 

Exemplo do usos delas, são redes socais mais famosas, como Facebook, Twitter e Instagram que possuem suas próprias APIs e são usadas pelos programadores para se ter a integração mais facilmente realizada entre os serviço que as redes sociais oferecem e a sua aplicação.

As APIs das redes sociais por exemplo, oferecem recursos como recuperação e envio de informações, além de outros serviços, como os mecanismos de geolocalização, com uso do GPS.

\subsection{REST}
A Representational State Transfer (REST), é frequentemente usado em\textit{web services}, permitindo que os clientes acessem e manipulem os dados através de requisições. \cite{zhou2014} aborda que é possível que clientes requisitem diferentes dados de um mesmo recurso URI, ou seja, requisitar a partir do endereço da página, seguido do recurso que deseja solicitar.  

As requisições de dados podem ter como respostas aquivos do tipo XML, HTML, JSON, entre outras, podendo-ser receber um único recurso ou 
uma coleção deles.

Na arquitetura cliente-servidor, a REST é usada para que os clientes possam realizar requisições de dados aos servidores, seguindo as regras e com uso dos métodos GET, POST, DELETE e PUT.  

\section{Linguagem de modelagem unificada - UML}
Para \cite{guedes2009}, a UML é uma linguagem visual para modelagem de softwares. Possuindo um propósito geral, ela é a linguagem-padrão adotada pela engenharia de software para modelagem.

A UML tem o objetivo de auxiliar os desenvolvedores de software a definir as características que o sistema terá, definindo os requisitos, comportamento, estrutura lógica e até mesmo as necessidades físicas como equipamentos e sistemas que serão usados. 

Para as definições podem ser separadas em diversos diagramas, como o de classe, casos de uso, de objeto, entre outros. 

\section{Metodologia de desenvolvimento - SCRUM}
Existem diversas metodologias de desenvolvimento que são consideradas ágil, onde a SCRUM está inserida. Para \cite{dos2013}, o SCRUM se destaca onde os requisitos não estão claros ou mudam com frequência, sendo um processo de desenvolvimento de \textit{software} incremental.

Os processos são divididos em \textit{sprints}, que são as interações que ocorrem durante o desenvolvimento do \textit{software}. Dentro de cada \textit{sprints} acontecem reuniões que são colocadas as atualizações do status do projeto.  

\section{JSON}
JSON (JavaScript Object Notation) é uma formatação de fácil leitura e escrita por seres humanos, usado para troca de dados e baseado em subconjunto do javascript.

Podem ser estruturados de duas formas, podendo ser como uma coleção de pares relacionando nome e valor ou então uma lista ordenada de valores. 

Podendo possuir os quatro tipos básicos de conteúdo: numérico, booleano, caractere e String, o JSON é um modelo para armazenamento e transmissão de informações no formato texto e com capacidade de transmitir um grande volume de dados.

No exemplo \ref{lst:json}, é criado uma variável contendo várias informações sobre um usuário, por exemplo. 

\begin{lstlisting}[caption={Exemplo de JSON},label={lst:json}]
	$variavel = { 
		"name":"Fulano", 
		"age":50, 
		"city":"Brasilia",
		"inicialNome": "f",
		"PossuiAnimal": false
	};
\end{lstlisting}

\section{QRCode}
Sendo o QR Code um código bidimensional de leitura rápida, \cite{sousa2014} explica que o seu uso implica em duas etapas, que é a impressão e a leitura do código. A leitura está condicionada ao uso de um \textit{scanner}, estando diretamente ligado ao uso do dispositivo móveis e a internet para acesso ao conteúdo.

\cite{sousa2014} comenta que a ferramenta é cada vez mais usada na estratégia de marketing digital, concebendo a comunicação e a interação entre organizações e público-alvo.

\section{Frameworks}
\textit{Frameworks} é usado para diversas funções, entre elas estão a de automatizar tarefas repetitivas e a de melhorar a organização e estruturação do código.  

No geral, para \cite{minetto2007} um \textit{framework} é uma coleção de códigos-fontes, classes, funções, técnicas e metodologias que visam facilitar o desenvolvimento de novos softwares. A seguir, estão descritos alguns dos \textit{frameworks} utilizados.

\subsection{ZEND}
Zend Framework é uma coleção de pacotes profissionais para PHP, podendo ser usado na versão 5.6 ou maior do PHP \cite{zend2018}, ele oferece completa orientação a objeto, boas práticas, reusabilidade, internacionalização, código aberto, suporte da comunidade, boa documentação, entre várias outras vantagens \cite{vaswani2010}[p.4-5].

\cite[p.3]{vaswani2010} explica que o Zend oferece uma completa implementação do \textit{Model-View-Controler}(MVC), servindo para aplicações de médio e grande complexidade e é geralmente usada no desenvolvimento de aplicações \textit{Web}.

Usado em conjunto com o PHP para estruturação do projeto, o Zend oferece a facilidade de separação de módulos, essa funcionalidade foi usada para divisão do projeto em três diferentes módulos, são eles, administrador, cliente e API, onde cada um dos módulos oferece diferentes funcionalidades.

\subsection{Doctrine}
Nas aplicações que armazenam informações no banco de dados é feito uma camada de abstração dos dados do banco, com o gerenciamento de conexões e execução de consultas SQL. O Zend cria um \textit{Object-Relational Mapping}(ORM) usando o doctrine.

O doctrine é usado para obter dados de um banco, usando o ORM ele cria um mapeamento do dados para simplificar as tarefas de consulta \cite[p.102]{vaswani2010}.

O doctrine é usado na aplicação para fazer a comunicação entre a aplicação criada e o banco de dados. Com ele é possível fazer a inserção, listagem, edição e exclusão de dados.

\subsection{Cordova}
Cordova é um \textit{framework} mantido pela Apache, disponível para diversos sistemas e é usado para criação de aplicativos multiplataforma, utilizando-se de linguagens \textit{Web} como HTML, CSS e javascript é possível desenvolver para diversas plataformas distintas, seja ela \textit{browser}, Android, iOS e outras \cite{prezotto2017}.

No intuito da criação de um aplicativo multiplataforma, o SID \textit{mobile} foi desenvolvido usando o \textit{framework} Cordova. Com ele é possível criar o executável e posteriormente a instalação do aplicativo desenvolvido nos mais diversos \textit{smartphones}.

\subsection{Framework7}
Framework 7 - F7 é um \textit{framework} gratuito e de código aberto para desenvolver aplicações híbridas que funcionam em navegadores, Android e iOS com visões e estéticas nativas \cite{f72018}.

A criação de uma aplicação hibrida com o F7 é feita usando uma estrutura com HTML, CSS e JS. O HTML é usado para criação dos itens que serão mostrados para o usuário, as telas por exemplo. O CSS é usado para organizar e estilizar o conteúdo criado no HTML. Já o JS é usado para realizar as requisições e transições de tela.