\chapter[Referencial Teórico]{Referencial Teórico}
\section{Marketing Digital}
O marketing já passou diversas modificações e melhorias com o passar dos anos, começando com jornais e revistas, passou por panfletos, pôsteres e outdoores, teve-se também os rádios, televisões e telefones até se atingir hoje o marketing digital. Ele se aproveita da popularização das redes sociais e da crescente facilidade de acesso a Internet para aproximar produtores e vendedores as pessoas, que são consumidores, integrando ou expondo produtos ou serviços de maneira simplificada no cotidiano das pessoas.

O advento da Internet, obrigou o marketing a ser mais digital, obrigando ele a se inserir em novos contextos, \cite{canto2017} afirma que o marketing digital é uma \begin{quote} ``premissa básica para aqueles que desejam estreitar as relações com os clientes''\end{quote} criando melhores estrategias de marketing, inserindo-a em um ambiente digital e integrando as redes sociais para se ter uma ampla divulgação.

\cite[p.2]{ryan2016} relata que o marketing digital em 2016 se tornou uma grande fonte de negócios, onde com a revolução online trouxe uma nova onda de consumidores, aqueles que integram a tecnologia a sua vida cotidiana. Para ele, publicidade tem tudo a ver com influenciar pessoas e para isso propagandas e tecnologias vem se convergindo
cada vez mais para criar um novo panorama de marketing. A tecnologia e a publicidade de tornarem parte do cotidiano das pessoas.

\cite{santos2014} comenta que em um primeiro momento as estratégias de mercado eram voltadas para meios tradicionais como TVs, rádios, jornais, revistas e outros. Com a popularização da Internet foi necessário a busca por outros meios de para se realizar
o marketing, chegando ao marketing digital que é o uso das mídias sociais ou alguma ferramentas cotidianas, como a Internet, para exposição dos produtos.

O marketing tem por objetivo, como cita \cite{ryan2016}, persuadir as pessoas a tomar determinadas ações que são desejadas pela divulgação, por exemplo, influenciar alguém a escolher uma determinada marca ou objeto, apenas como aquela apresentação.

Na opinião de \cite{torres2000}, as mídias sociais são paginas de Internet onde os usuários são ao mesmo tempo produtor e consumidor das informações que nelas são criadas, sendo possível a criação e compartilhamento dessas informações. Ainda para eles, essas mídias receberam esse nome por serem livres e terem a possibilidade de
colaboração e interação de todos que nelas estão, além de ser um meio de transmissão das informações e conteúdos.

O marketing digital faz uso também das mídias sociais para uma melhor apresentação e disseminação do conteúdo desejado. Como apresenta \cite{torres2000}, a facilidade de exposição e compartilhamento que as redes sociais apresentam, as tornam mais atrativas quando se quer expor algo.

\section{Sinalização digital}
Para \cite{munari2006} a sinalização pode ser feita das mais diversas formas, desde códigos, sinais, luzes, canetas, lápis ou até mesmo por imagem. Ela deve ser melhorada de modo a atingir o objetivo proposto pelo criador, seja ela para propaganda, informativo ou entretenimento, mas sempre com o mesmo objetivo, obter a atenção do telespectador.

\cite[p.11]{cintra2010} relaciona a sinalização digital como alternativa aos outdoors,
que poluem visualmente grandes cidades. Usualmente colocadas em pontos estratégicos e de grande movimento, como academias, shoppings, elevadores e ônibus, tem-se a grande vantagem da economia de tempo e dinheiro. A era da comunicação digital como apresenta essa autora, coloca como vantagem o primeiro ponto sendo a agilidade e a
facilidade de troca do conteúdo contido nos painéis, por ser de forma digital e geralmente remota e o segundo ponto como sendo a inserção de novos recursos se comparados com outdoors, recursos como movimentos nas imagens, efeitos digitais e som.

De acordo com \cite{mishima2016} sistemas de sinalização
digital que possuem algum dispositivo eletrônico são amplamente usados para exibir informações dinâmicas, ao contrários de anúncios estáticos, que ficam por determinado ou muitas vezes logos tempos na mesma informação, o que acaba por muitas vezes, não atraindo a atenção ou interesse de quem os observa.

Com o uso de programas para gerenciamento do conteúdo, \cite[p.31]{machado2010} explica que a a sinalização digital consiste na transmissão de conteúdos, via Internet, à televisões ou painéis dos mais diversos tamanhos e nos mais diversos locais, usualmente instalados em pontos estratégicos onde as pessoas costumam ficar algum tempo como bancos, ônibus, elevadores.
\cite[p.36]{machado2010} apresenta diversas vantagens e modos de uso para a sinalização digital, cada empresa por optar ou não por ter determinada funcionalidade, mas em geral as vantagens apresentadas são opções como a de atualização instantânea de conteúdo, agendamento de programação, painel para controle, anúncios diferenciados e divisão de tela com diferentes conteúdos.

O uso da sinalização digital não fica restrito a uso externo, para propagandas, também pode ser usado para fins internos, para apresentar informações apenas para dentro de uma empresa. Além disso é possível a troca ou atualização de conteúdo em todos os pontos conectados ao mesmo tempo, economizando tempo e mão de obra para reajustar cada ponto específico.

\section{Linguagens}
Para \cite{sebesta2011}, existe uma lista de potenciais vantagens no estudo dos conceitos de linguagem de programação, entre elas esta a da capacidade aumentada para expressar ideias, onde para \cite{sebesta2011}, outras pessoas que não possuem esse conhecimento é complicado criar conceitos de estruturas que elas não podem descrever ou expressar. Na lista também aponta o embasamento para escolher linguagens adequadas e a habilidade aumentada para aprender novas linguagens como parte da lista de vantagens de se estudar esses conceitos.

Na linguagem de programação, metodologias de projeto, ferramente de desenvolvimento e a própria linguagem de programação estão ainda em evolução, por esse motivo \cite{sebesta2011} considera o desenvolvimento de software uma profissão excitante, mas que obriga a se ter um aprendizado contínuo.

\subsection{PHP}
\subsection{JavaScript - JS}

\section{ Linguagem de modelagem unificada - UML}

\section{Frameworks}
\subsection{ZEND}
\subsection{Doctrine}
\subsection{Cordova}
\subsection{Framework7}

\section{Arquitetura Cliente-Servidor}

\section{QRCode}

\section{Metodologia de desenvolvimento}

\section{Banco de Dados}