\chapter[Referencial Teórico]{Referencial Teórico}
\section{Sinalização digital}
Para \cite{munari2006} a sinalização pode ser feita das mais diversas formas, desde códigos, sinais, luzes, canetas, lápis ou até mesmo por imagem. Ela deve ser melhorada de modo a  atingir o objetivo proposto pelo criador, seja ela para propaganda, informativo ou entretenimento, devendo ela atrair a atenção do receptor. 

De acordo com \cite{mishima2016} sistemas de sinalização digital que possuem algum dispositivo eletrônico são amplamente usados para exibir informações dinâmicas, ao contrários de anúncios estáticos, que ficam por determinado tempo na mesma informação. 

Com o uso de um programas para gerenciamento do conteúdo, \cite{machado2010} explica que a a sinalização digital consiste na transmissão dos conteúdos, via \textit{Internet}, à televisões ou painéis instalado em pontos estratégicos onde as pessoas costumam ficar algum tempo como bancos, ônibus, elevadores.

\section{Marketing Digital}
\cite{santos2014} comenta que em um primeiro momento as estratégias de mercado eram voltadas para meios tradicionais como TVs, rádios, jornais, revistas e outros. Com a popularização da \textit{Internet} foi necessário a busca por outros meios de para se realizar o \textit{marketing}, chegando ao \textit{marketing} digital que é o uso das mídias sociais para exposição dos produtos.

Na opinião de \cite{torres2000}, as mídias sociais são paginas de Internet onde os usuários são ao mesmo tempo produtor e consumidor das informações que nelas são criadas, sendo possível a criação e compartilhamento dessas informações. Ainda para \cite{torres2000}, essas mídias receberam esse nome por serem livres e terem a possibilidade de colaboração e interação de todos que nelas estão, além de ser um meio de transmissão das informações e conteúdos.

\section{Mini Controladores}
Desde a sua invenção em 1971, microprocessadores vem sendo usados no desenvolvimento dos mais variados tipos de eletrônicos ou outros equipamentos, substituindo até mesmo sistemas mecânicos. Algo que vai além de um simples software, os microprocessadores devem ser capazes controlar as ações de um dispositivo. \cite{rosenstark2007}

Para \cite{aristotelous2016}, o objetivo essencial de todos os tipos de empresa é a rentabilidade, podendo ela ser alcançada usando uma solução de baixo custo, boas tecnologias e com um preço atrativo. Teste do \cite{aristotelous2016} apresenta a possibilidade de se ter um servidor completamente funcional com sistema operacional Linux por um equipamento de 35\$, possibilitando a criação de um servidor, por exemplo de um repositório na nuvem com um baixo custo, flexibilidade e eficiência energética. 

Grandes servidores oferecem um melhor desempenho, entretanto, o baixo uso, a pouca eficiência energética ou até mesmo o pouco espaço podem limitar o uso desse tipo de equipamento. Nesse sentido, para \cite{Cusick}, placas de circuito oferecem vantagens como o uso de pouco espaço, desempenho significante com baixo custo e consumo, além do suporte a diversas soluções de software oferecendo múltiplas opções de interface com uma variante do Linux. 

\section{Linguagem de Programação}
Para \cite{sebesta2011}, existe uma lista de potenciais vantagens no estudo dos conceitos de linguagem de programação, entre elas esta a da capacidade aumentada para expressar ideias, onde para \cite{sebesta2011}, outras pessoas que não possuem esse conhecimento é complicado criar conceitos de estruturas que elas não podem descrever ou expressar. Na lista também aponta o embasamento para escolher linguagens adequadas e a habilidade aumentada para aprender novas linguagens como parte da lista de vantagens de se estudar esses conceitos.

Na linguagem de programação, metodologias de projeto, ferramente de desenvolvimento e a própria linguagem de programação estão ainda em evolução, por esse motivo \cite{sebesta2011} considera o desenvolvimento de software uma profissão excitante, mas que obriga a se ter um aprendizado contínuo.

%\section{Interface de programação de aplicativos}
%API
%
%Continua ...

%\section{Framework}
%Um framework é um conjunto de códigos comuns abstraídos de vários %projetos com o objetivo de prover funcionalidades genéricas em um %projeto que utilize esse framework.

%Continua ...
%\section{Banco de Dados}
%Banco de dados
%
%Continua ...