\begin{resumo}[Abstract]
 \begin{otherlanguage*}{english}
   
This work presents the Integrated Information Disclosure System of IFB Câmpus Taguatinga - SID, which, through the use of the concepts of digital signage and digital marketing, aims to convey information to be transmitted in the Campus environment in a simple, effective, interactive and dynamic way .

To achieve the desired characteristics, a series of decisions and modifications were made in the SIDv2 system proposed by \cite{sobrinho2017}. Modifications were made to the existing architecture and new functionalities were implemented, in order to make flexible the implantation of new devices that could make use of this system.

SID also has the complete integration with the social network Facebook, making available the possibility of making publications on Facebook pages and presenting the contents related to those publications. This is achieved through the unification of vehicles other than the dissemination of information, one based on digital signage and another on social networks.

A communication mobile application is proposed that aims to pass on the divulgations created, besides making use of a fictitious API for interaction between students and teachers with the exchange of messages. This consumption of API allows future integration with the Academic Management System (SGA).

   \vspace{\onelineskip}
 
   \noindent 
   %\textbf{Key-words}: latex. abntex. text editoration.
 \end{otherlanguage*}
\end{resumo}
