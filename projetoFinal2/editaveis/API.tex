\chapter[GraphApi]{GraphApi}

\section{Visão Geral}
Para integração entre o Facebook e outros aplicativos externos é necessário o uso de uma API, a disponibilizada pela rede social em questão é a Graph. Ela é usada para que aplicativos externos possam realizar as requisições e envio de dados para a rede social, possibilitando consulta e gerência dos dados presentes nela. 

Um sistema de permissões é utilizando na Graph API para controle de acesso, controle de publicação e de edição de informações. Assim, para que alguma modificação na publicação por parte do modulo administrador possa ser efetivada, é necessário possuir as permissões adequadas. As permissões funcionam de forma a descrever como devem ser feitas as requisições para o completo uso do aplicativo, de acordo com a necessidade. Elas foram configuradas durante o desenvolvimento do SID e serão solicitadas durante o processo de login do usuário que tentar acessar.

As permissões descrevem quais as possíveis ações podem ser feitas em cooperação com a Graph API, elas determinam quais tipos de dados pode-se gerenciar e quais as possíveis respostas o sistema pode retornar.

A estrutura da Graph, segue a de um grafo, onde possui nós, arestas e atributos (campos). Nós podem representar diversos elementos, eles são objetos individuais, onde cada nova página, usuário, comentário ou foto criada no Facebook é considerado um nó \cite{facebook2018b}. Partindo de um nó é possível se obter as arestas (bordas), exemplos de arestas são as fotos de uma página ou comentários de uma foto. A página do SID é um nó, usando o ID único dela é possível criar novas publicações que serão vinculadas a esse nó, essas novas publicaçãos serão formados do ID da página acrescidos de um ID único da foto, esse ID único da foto é considerado um outro novo nó. Usando ele é possível recuperar comentários e \textit{likes} \cite{facebook2018b}.

O Facebook oferece diversos produtos para incorporar ao aplicativo externo, entre eles está o de login do Facebook. Esse produto oferece ao desenvolvedor a possibilidade de oferece ao usuário do aplicativo uma ferramenta de login usando as credenciais do Facebook.

A autenticação (\textit{login}) no módulo administrador é feita usando a ferramente de login oferecida pela rode social, então para acesso é necessário um usuário cadastrado no Facebook e cadastrado no no banco de dados do SID. O uso de um usuário vinculado a rede social se torna necessário pois existe a necessidade da página apresentar qual o perfil está realizando a ação, além da necessidade de moderação dos comentários que serão exibidos. 

Para autenticação e efetivação de todas as requisições feitas pelo aplicativo para o Facebook, seja ela para requisitar informações das publicações ou realizar uma nova publicação, é usando um \textit{token} de acesso, que é obtido após a efetivação de login do usuário com o Facebook. O \textit{token} é uma cadeia de caracteres que identifica um usuário, aplicativo ou página, identificando a sessão. Em cada nova requisição a rede social, o \textit{token} será usado, se autorizado, pela aplicação para permitir o envio de requisições HTTP usando seus identificadores únicos, recebendo a respectiva resposta.

\section{Elementos Usados}
Para publicação no perfil pessoal, a Graph API requisita duas permissões, sendo a “email”, onde é requisitado o acesso ao endereço de email do usuário para que seja possível a autenticação do SID com o Facebook e a “publish\underline{{ }}actions”, onde fornece acesso a realização de publicações em nome da pessoa que está usando o aplicativo \cite{facebook2018a}.

Como o SID será usado em uma página e não em um perfil pessoal, duas novas autenticações se tornaram necessárias, sendo a “manage\underline{{ }}pages”, usada para recuperar as permissões de acesso a pagina e a “publish\underline{{ }}pages”, usada parar permitir que aplicativos publiquem na página \cite{facebook2018a}.

Usando o identificador único da foto é possível recuperar informações das arestas, que são informações como a URL, comentários e curtidas. Esse recurso é utilizado para recuperar o endereço da publicação, os comentários e curtidas que serão apresentados respectivamente nos campos destinados ao QRCode e na coluna de comentários do módulo cliente.

Na criação de uma nova publicação, usando o SID, são passados para a Graph API dois parâmetros para inserção, o primeiro deles é “message”, onde será passado o texto que será exibido na publicação e o outro é “source”, onde será passado a imagem para ser exibida juntamente com o texto. Para envio de imagem para a rede social, é necessário passar na imagem como parâmetro o método “fileToUpload”. 

Alguns dos elementos que são solicitados pela aplicação na criação de uma nova publicação, são omitidos no envio para o Facebook, pois esses dados serão usados apenas para serem armazenados no banco. Os elementos omitidos são os campos data de inicio, a data de termino e a legenda. 

------------------ CONTINUA  P48-------------------

\section{Integração}
Para acesso a funções restritas do sistema é necessário que um \textit{login} seja feito. Para isso, é utilizado a ferramenta de \textit{login} do Facebook em conjunto com a Graph. Para funcionamento é requisitado alguns parâmetros obrigatórios, são eles: app\underline{{ }}id, app\underline{{ }}secret, default\underline{{ }}graph\underline{{ }}version e fileUpload.

\begin{itemize}
\item O app\underline{{ }}id e o app\underline{{ }}secret é o id único de cada aplicativo vinculado a rede social, esse id é criado pelo Facebook no momento da criação do que será o seu aplicativo, o app\underline{{ }}id é publico, enquanto o app\underline{{ }}secret é secreto.

\item O parâmetro defalt\underline{{ }}graph\underline{{ }}version irá identificar qual versão da Graph o seu programa irá usar.

\item fileUpload é o parâmetro necessário para informar se será enviado arquivo de imagem ou não.
\end{itemize}

A rede social oferece um botão para ser colocado na página, esse botão faz com que o processo de login seja iniciado, então é feito a chamada do método ''getRedirectLoginHelper``, para ele é passado as informações de permissões que serão necessárias e o endereço de callback, que será o endereço de retorno caso o processo seja efetivado. O endereço de callback deve ser o mesmo informado no aplicativo criado na página da rede social.

Após o login, as informações usadas são guardadas na sessão do usuário, para usos posteriores, além disso, todo o processo deve ser transparente ao usuário. 

Qualquer divulgação inserida na pagina do SID no Facebook utiliza-se do nó “photo”, ou seja, a divulgação que o SID repassa à Graph API contém a mesma estrutura de uma imagem postada por um usuário convencional desta rede social. 

A Figura \ref{fig:imgfacebook1} apresenta como ficará uma publicação no Facebook após o uso do SID para criação da mesma. Nela é apresentado quem fez a publicação, o texto e a imagem que foi informado durante a criação, com uso do SID.

\begin{figure}[H]
\centering
\includegraphics[scale=1]{figuras/imgfacebook1}
\caption{Divulgação enviada ao Facebook com auxilio do SID}
\label{fig:imgfacebook1}
\end{figure}

O envio das informações para o Facebook é feita estritamente ao nó ''photo``, com o uso do método \textit{post} em conjunto com a Graph, a rede social recebe os parâmetros e faz a criação de um novo objeto, criando uma nova aresta vinculado a página, onde o retorno da requisição é o id desse novo objeto criado. Os parâmetros obrigatórios para criação do novo objeto são os mesmos do usados no login, com adição do: id\underline{{ }}pagina e do token. 

\begin{itemize}
\item O \textit{token} é um conjunto de caracteres que identifica para o Facebook as autorizações que a aplicativo possui.

\item O id\underline{{ }}pagina é o identificador único da sua página ou perfil.
\end{itemize}

É possível também solicitar dados referentes a postagens efetivadas. Para isso é necessário a passagem de alguns parâmetros no método ''get`` e a Graph. Assim como no ''post``, o ''get`` usa todos os parâmetros com exceção do id\underline{{ }}pagina, onde será informado o id\underline{{ }}publicação em substituição do da página.

Com essa requisição é possível recuperar diversas informações referentes a publicação, informações como \textit{url}, comentários, \textit{likes}, data, hora, foto de perfil e nome do usuário que realizou o comentário. O retorno da requisição, se feita com sucesso, retornará as informações solicitadas.