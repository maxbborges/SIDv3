\chapter[API]{API}
\section{Visão Geral}
\section{Elementos Usados}
Um sistema de permissões é utilizando na Graph API para controlar o acesso, a publicação e a edição de informações. Assim, para que alguma publicação possa ser feita ou visualizada pelo aplicativo, é necessário possuir as permissões adequadas. As permissões do o aplicativo são configuradas durante o desenvolvimento do SID, para publicações em paginas e no perfil, a Graph API requer as permissões de  "email", "publish\textunderscore{}actions" , "manage\textunderscore{}pages"  e "publish\textunderscore{}pages".

Para o gerenciamento do perfil pessoal a API requisita as permissões  "email" e "publish\textunderscore{}actions",  Para o gerenciamento e publicação em páginas  do Facebook, a solicita duas permissões, são elas a "manage\textunderscore{}pages" e a "publish\textunderscore{}pages".

A "publish\textunderscore{}actions"

Enquanto a "email"

Com a "manage\underline{{ }}pages"

Já a "publish\underline{{ }}pages"

\section{Integração}