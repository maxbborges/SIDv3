\chapter[API]{API}
\section{Visão Geral}
Um sistema de permissões é utilizando na Graph API para controlar o acesso, a publicação e a edição de informações. Assim, para que alguma modificação no modulo administrador possa ser feita, é necessário possuir as permissões adequadas. As permissões necessárias para o completo uso do aplicativo foram configuradas durante o desenvolvimento do SID e serão solicitadas autorização por parte do usuário que tentar acessar.

-------------- CONTINUA -----------

\section{Elementos Usados}
Para publicação no perfil pessoal, a Graph API requisita duas permissões, sendo a “email”, onde é requisitado o acesso ao endereço de email do usuário para que seja possível a autenticação do SID com o Facebook e a “publish\underline{{ }}actions”, onde fornece acesso a realização de publicações em nome da pessoa que está usando o aplicativo \cite{facebook2018a}.

Como o SID será usado em uma página e não em no perfil no pessoal, duas novas autenticações foram necessárias, sendo a “manage\underline{{ }}pages”, usada para recuperar as permissões de acesso a pagina e a “publish\underline{{ }}pages”, usada parar permitir que aplicativos publiquem na página \cite{facebook2018a}.

Cada nova página ou foto criada no Facebook é considerado um nó, possuindo um ID único, usando o nó é possível se obter as bordas, no caso fotos de uma página ou comentários em uma foto. Partindo desse princípio, a página do SID é um nó, usando o ID único dela é possível criar novas publicações que serão vinculadas a esse nó, possuindo o ID da página acrescidos de um ID único da foto, gerando um novo nó \cite{facebook2018b}.

Usando o identificador da foto é possível recuperar informações da borda, que são informações como a URL, comentários e curtidas. Esse recurso é utilizado para recuperar o endereço da publicação e os comentários e curtidas que serão apresentados respectivamente no QRCode e na coluna de comentários do módulo cliente.

Na criação de uma nova publicação, usando o SID, são passados para a Graph API dois parâmetros para inserção, onde o primeiro deles é “messagem”, onde será passado o texto que será exibido no publicação e o outro é “source”, onde será passado a imagem para ser exibida juntamente com o texto.

——————– CONTINUA ———————



\section{Integração}