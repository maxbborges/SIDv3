\chapter[Resultados Preliminares]{Resultados Preliminares}
Foi feito a instalação e configuração do SID no ambiente Linux, instalando as dependências necessárias para a devida execução, tais como o Apache2, PHP 7.0 e composer. Usando o componser foi possível instalar as dependências do SID, que ficam na pasta vendor e foi necessário atualizar algumas dependências para versões mais novas.

No banco de dados foram feitas algumas alterações, tais como a coluna de armazenamento da imagem em formato bytea para uma coluna de armazenamento de string com object\textunderscore{}id da postagem. Em sua primeira versão o SID armazenava as imagens que eram enviadas para o Facebook no banco de dados, o que poderia causar lentidão nas consultas após algum tempo de uso. Na versão atual, as imagens são salvas em um local especifico no computador e no banco de dados é armazenado apenas o id da postagem, recuperando a string da coluna object\textunderscore{}id é possível recuperar uma imagem armazenada em um local específico no computador.

No modulo administrador foram feitas as mudanças de conta vinculada ao Facebook e ao aplicativo vinculado a essa conta, atualizando os identificadores de usuário, de aplicativo e a versão da Graph API. Além disso foi inserido novas integrações do aplicativo com o Facebook, sendo possível recuperar informações como data e proprietário da postagem, curtida ou comentário. Para as novas integrações, houve a necessidade de incluir duas novas permissões junto a Facebook, foram elas manage\textunderscore{}pages e publish\textunderscore{}pages, usadas para gerenciamento de paginas. Na versão anterior era possível apenas a postagem no perfil do usuário.

No que tange a recuperação de postagens, na versão atual é possível recuperar uma lista com todas as postagens que foram publicados na pagina. Recuperando essas postagens, é possível selecionar uma especifica e verificar informações únicas dela, como o a string object\textunderscore{}id usado para atribuir um nome único para a nova foto da postagem que será armazenada no computador, além de ser usado para recuperação dessa imagem para exibição. Usando essa mesma string é possível recuperar a URL da postagem, que será usada no QRCODE para exibição completa da publicação na rede social.

Recuperando o ID publicação é possível obter  informações especificas da postagem, como as comentários e os curtidas. No primeiro é possível recuperar o id do proprietário a hora em que ele foi publicado. Com as curtidas é possível recuperar data e proprietário, o proprietário da curtida será posteriormente usada para filtragem de comentários que serão exibidos nas telas. 

As imagens armazenadas no computador, após a postagem é salva em um diretório especifico e é atribuído o formato png para ela. O nome da imagem será o mesmo vinculado ao object\textunderscore{}id da postagem. No detalhamento da postagem foi inserido um novo campo, onde é mostrado também o object\textunderscore{}id da publicação.