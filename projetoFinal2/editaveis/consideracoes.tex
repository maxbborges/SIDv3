\chapter[Considerações Finais]{Considerações Finais}
Com objetivo de aprimorar os meios de comunicação do IFB - Campus Taguatinga, foi usado os conceitos de marketing digital e sinalização digital para implementação de um sistema fácil e intuitivo. Inserindo interação com o usuário que visualiza as divulgações e nova forma de comunicação interna entre professores e alunos.  

Divido em dois módulos, sendo o administrador e o cliente. O primeiro é responsável por desempenhar todo o processamento de recuperação, armazenamento e estruturação dos dados, enquanto o segundo que inclui o aplicativo mobile, tem a função de exibição desses dados de forma acessível para o usuário.  

Em suma, o SID obtém benefícios consideráveis em ralação às outras soluções que são utilizadas atualmente, além de a divisão entre módulos possibilitar a exibição das divulgações em diversos dispositivos distintos, incluindo o aplicativo \textit{mobile} que foi desenvolvido.

\section{Trabalhos Futuros}
Apesar de atingido todos os objetivos, a implementação de uma REST API abre diversas possibilidades de ampliação do sistema. Sendo possível a inserção de novos módulos e funcionalidades.

A melhoria e finalização do aplicativo mobile é uma das propostas para melhoria, onde pode ser feita uma melhor integração com o SGA, ao invés do uso de perfis fictícios.

Além disso, é 