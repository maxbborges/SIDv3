\chapter[Considerações Finais]{Considerações Finais}
\label{consideracoes}
Com objetivo de aprimorar os meios de comunicação do IFB - Campus Taguatinga, foi usado os conceitos de marketing digital e sinalização digital para implementação de um sistema fácil e claro. Inserindo interação com o usuário que visualiza as divulgações, além de um sistema que simula o consumo da API do SGA, para possibilitar uma posterior inclusão quando esta interface do SGA for disponibilizada.

O SID está divido em módulo administrador, módulo cliente e aplicativo. O administrador é responsável por desempenhar todo o processamento de recuperação, armazenamento e estruturação dos dados, enquanto o módulo cliente e o aplicativo realiza a função de exibição desses dados de forma acessível para o usuário.  

Em suma, o SID apresenta benefícios consideravelmente melhores em relação às outras soluções que são utilizadas atualmente, além da divisão entre módulos possibilitar a exibição das divulgações em diversos dispositivos distintos, incluindo o aplicativo \textit{mobile} que foi desenvolvido.

Entre os benefícios apresentados estão: integração completa com as redes sociais, possibilitando recuperação de comentários, curtidas e postagens em tempo real; divulgação de eventos e notícias através de telas espalhadas pelo campus; unificação de dois meios de comunicação distintos, facilitando a vida do operador do sistema; consumo de uma API Fictícia visando a simulação com o SGA para posterior integração com o mesmo, quando esta API do SGA estiver disponível.

\section{Trabalhos Futuros}
Apesar de atingido todos os objetivos, a implementação de uma REST API abre diversas possibilidades de ampliação do sistema. Sendo possível a inserção de novos módulos e funcionalidades.

\subsection{Aplicativo \textit{mobile}}
A melhoria e finalização do aplicativo \textit{mobile} é uma das propostas para melhoria, onde pode ser feita uma melhor integração com o SGA após a liberação da interface para acesso ao SGA.

\subsection{Moderação dos comentários}
Além disso, é necessário uma nova forma de moderação dos comentários, pois existe as limitações descritas para os \textit{tokens} de aplicativo.

\subsection{Atraso da recuperação de dados}
Alguns dados que são recuperados pelo módulo API, são repassados para o cliente em forma de URL, sendo fundamental a busca \textit{online} desse dado pelo cliente, o que pode acabar gerando um atraso da entrega da informação. Implementar uma nova forma de recuperação desses dados como a utilização de uma cachê, poderia diminuir o atraso. 
