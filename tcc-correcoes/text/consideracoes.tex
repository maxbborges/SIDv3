

\chapter[Considerações Finais]{Considerações Finais}
\label{consideracoes}
Foram usados diversos conceitos e ferramentas para implementação das melhorias apresentadas, visando apresentar novos meios de comunicação do IFB - \textit{Campus} Taguatinga. Inserindo interação com o usuário que visualiza as divulgações, além de um sistema que simula o consumo da API do SGA, para possibilitar uma posterior inclusão assim que a interface do SGA for disponibilizada.

O SID está divido em módulo administrador, módulo cliente e aplicativo. O administrador é responsável por desempenhar todo o processamento de envio, recuperação, armazenamento e estruturação dos dados, enquanto o módulo cliente e o aplicativo realizam a função de exibição desses dados para o usuário. 

Entre os benefícios apresentados estão: integração completa com as redes sociais, possibilitando recuperação de comentários, curtidas e postagens; divulgação de eventos e notícias através de telas espalhadas pelo \textit{Campus}; unificação de dois meios de comunicação distintos, tendo em vista a disseminação de informações; consumo de uma API fictícia visando a simulação com o SGA para posterior integração com o mesmo, quando esta API do SGA estiver disponível.

\section{Trabalhos Futuros}
Apesar de atingido todos os objetivos, a implementação de uma REST API abre diversas possibilidades de ampliação do sistema. Sendo possível a inserção de novos módulos e funcionalidades.

\subsection{Aplicativo móvel}
A melhoria e finalização do aplicativo móvel é uma das propostas para melhoria, onde pode ser feita uma melhor integração com o SGA após a liberação da interface para acesso ao SGA.

\subsection{Moderação dos comentários}
Além disso, é necessário uma nova forma de moderação dos comentários, pois existem as limitações descritas para os \textit{tokens} de aplicativo.

\subsection{Atraso da recuperação de dados}
Alguns dados que são recuperados pelo módulo API, são repassados para o cliente em forma de URL, sendo fundamental a busca \textit{online} desse dado pelo cliente, o que pode acabar gerando um atraso da entrega da informação. Implementar uma nova forma de recuperação desses dados como a utilização de um \textit{cache}, poderia diminuir o atraso. 
