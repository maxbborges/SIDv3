\chapter[Introdução]{Introdução}
Apesar de muito utilizada, a definição exata de mídia é difícil de ser explicada. De acordo com \citet[p.51]{guazina2007}, o termo ``mídia'' é usado para representar os meios de comunicação em massa como jornais, televisão, rádio, cinema e \textit{Internet}.

Canais de televisão, outdoors ou até mesmo panfletos, em ambientes públicos ou privados, podem introduzir diversas mudanças comportamentais e comerciais, por isso \citet[p.3]{escobar2007} explica que a mídia é capaz de redefinir ``o modo como o homem se comunica e se relaciona com os semelhantes'', assim como \citet[p.54]{hjarvard2012} coloca que ela pode modificar a relação e o comportamento humano, criando o conceito de midiatização, que caracteriza a influência que a mídia possui sobre as pessoas.

\citet[p.53]{guazina2007} aponta que os meios de comunicação influenciam no processo de construção e formação das consciências. Além disso, \citet{silva2007} indica que a falta de acesso ou interesse por parte da população em buscar outras fontes, induzem no questionamento da veracidade das informações que são repassadas, mostrando o poder que a mídia pode ter sobre as pessoas.

Com o passar dos anos, as divulgações de forma estática e tradicionais (revistas, jornais e canais de televisão) deixaram de ser os meios mais eficientes de se expor um conteúdo ou propaganda. Para \citet{meditsch2001}, o advento e popularização da \textit{Internet} trouxeram ameaças para esses meios, então foi necessário que novas formas de expor conteúdos fossem pensadas e desenvolvidas em conjunto com a \textit{Internet}.


Para \citet[p.4]{escobar2007}, os atuais meios de veiculação de notícias (televisão e radio) oferecem a possibilidade da informação ser obtida em tempo real, mas quando há a inclusão da \textit{Internet}, coloca-se a possibilidade de interação com as informações que são recebidas, onde os usuários atuam de forma simultânea, comentando ou opinando sobre aquele determinado assunto. Para \citet{deuze2002}, o surgimento da \textit{Internet} traz a possibilidade do público ``responder, interagir ou mesmo customizar certas histórias''. 

Diante disso, surgiram novas formas e ferramentas de \textit{marketing} ampliando os meios de disseminação de notícia. Visto que, esse é responsável por criar e melhorar essas publicações, posto que essas possuem o objetivo de influenciar pessoas a adquirirem ou aderirem determinados produtos ou serviços, com o uso de diversos aparatos. Sendo assim foram necessários que a publicidade e a tecnologia convergissem, definindo um novo conceito, o \textit{marketing} digital, que englobam novas tecnologias de comunicação e incluem as redes sociais, como meio de exposição das propagandas.

A \textit{Internet}, também resultou na popularização do uso das redes sociais para as diversas finalidades, incluindo a exposição de anúncios. \cite{seo2017} apontam que em uma pesquisa realizada pela \url{marketingcloud.com}, 90\% das compras efetuadas, durante o período da pesquisa, foram influenciadas pelas redes sociais. O incremento destes meios, para \citet[p.2]{escobar2007}, colocam a interatividade em evidencia, pois por muitas vezes são usadas para ter um melhor contato com o usuário, então, a utilização de novas técnicas mais interativas com seu receptor tornam a leitura menos monótona e com maior possibilidade de obter a atenção do usuário, de forma mais consistente.

Assim sendo, as mídias sociais obtiveram um importante papel para disseminação de informações ou conteúdos, tornando-se uma das ferramentas mais atraentes para divulgações. \citet{rosa2010} aborda que crescentemente as redes sociais estão no cotidiano das pessoas. Essas são interessantes não apenas por serem um dos meios mais acessados atualmente, bem como possuem a facilidade de interação dos usuários com as notícias.

A evolução do \textit{marketing} trouxe consigo a exibição de propagandas não só nas mídias sociais, mas também em dispositivos móveis. Isso permite que se tenha receptores dos informes em varias localidades, além da possibilidade de interação com essas noticias que estão sendo transmitidas. Para \citet{santos2014}, a utilização do \textit{marketing} na \textit{Internet}, favorece a criação desse em ambiente digital.

Para ter maior abrangência, apodera-se do conceito de sinalização digital, que para \citet[p.37]{machado2010} dar-se com o uso de telas espalhadas por diversos pontos com diferentes informações que são repassadas via \textit{Internet}. Com isso poderá haver receptores em diversos locais, independente de cidade, estado ou país. As telas ou televisores apresentam informações e propagandas de forma dinâmica, que podem ser gerenciadas remotamente de acordo com a necessidade. 

O uso de dispositivos móveis, concomitante com as telas, podem atrair um maior público, pois os dispositivos móveis, com o passar dos anos, estão sendo cada vez mais utilizados pelas pessoas dados que podem ser confirmados por pesquisas feitas pela agência \cite{emarketer} no qual afirmam que até 2019 mais de 80\% das pessoas que acessam a \textit{Internet}, usarão o celular para acessa-la. Por isso, é crescente a utilização destes aparelhos como ferramenta para divulgação de informações, não somente pelo grande número de equipamentos, mas também pela integração com as redes sociais que esses oferecem. 

Para divulgação de notícias e informações, o Instituto Federal de Educação Ciência e Tecnologia de Brasília -- \textit{Campus} Taguatinga (IFB), utiliza-se principalmente de suas páginas \textit{web} e Facebook. Para o uso desses meios, é necessário que os administradores façam publicações independentes para cada uma das páginas, além disso, essas não são bem divulgadas, na maioria das vezes os estudantes e visitantes não ficam cientes das notícias que são publicadas. 

Ademais, as únicas formas de contato entre professores e alunos é por meio do e--mail ou com uso de aplicativos externos a instituição, necessitando que cada estudante encaminhe o seu e--mail para que o professor possa entrar em contato, pois o atual sistema da instituição não dispõe de uma funcionalidade, na qual os professores possam obter o e--mail dos alunos automaticamente.

Para \citet{pinheiro2010}, a comunicação interna é de fato importante para o sucesso de uma organização, em uma instituição educacional não é diferente, é interessante que os alunos e os professores sejam informados de futuros eventos, palestras e notícias. Os atuais veículos institucionais do IFB não possuem integração e atuam de forma independente, o que acabam degradando a qualidade e a disseminação dos informes. Além disso é complicada a interação dos usuários com as noticias, pois essas são expostas de forma descentralizada, ocasionando a baixa interatividade com as informações do IFB. Visto que, para \citet{santos2014}, a interatividade  é significante pois tem como proposito ``estreita o relacionamento com o público''.

O Sistema Integrado de Divulgação de Informações do IFB versão 3 (SIDv3) oferece uma maior visibilidade das notícias, sendo possível, por meio de painéis espalhados pelo \textit{Campus}, uma melhor interação da comunidade com as notícias, apresentando nos painéis os comentários que foram publicados na mídia social, além de uma melhor forma de comunicação entre professor e aluno, oferecido pelo aplicativo móvel que simula um sistema de comunicação institucional, integrado ao Sistema de Gestão Acadêmica (SGA).

\section{Motivação}
\citet{bianchi2006} abordam que algo inédito e atual deve atrair a atenção do telespectador, sendo esse crucial para o sucesso das notícias que estão sendo exibidas, quanto melhor essas forem disseminadas, maior a chance de sucesso. Nesse intuito, a interatividade e o dinamismo podem ser considerados modernos e o uso de ferramentas do cotidiano das pessoas, despertam e mantém a atenção do usuário, fazendo--o ter interesse em acompanhar e participar das noticias ou matérias.

Atualmente, o IFB utiliza o seu perfil na rede social Facebook, sua página oficial e os murais de cada \textit{Campus} para veicular notícias intrínsecas a instituição, sejam essas referentes a eventos ou institucionais. Para cada notícia é necessário que o administrador acesse as páginas separadamente e realize postagens independentes, além da necessidade de fixação de algumas dessas informações nos murais. Variados meios de divulgação e falta de uma ordem definida de onde cada publicação deve ser postada, pode ocasionar a confusão ou a não localização da mesma.

\citet{pinheiro2010} cita a importância da comunicação interna, quando se tem um sistema de divulgação defasado, as informações podem não atingir o resultado esperando, então é preciso uma melhor forma de exposição e edição dessas no IFB, uma forma mais precisa e ágil de apresentação dessas para alunos, professores e visitantes, contando com um sistema mais interativo, que seja capaz de ter uma participação mais adequada do público com a notícia, como \citet{santos2014} aborda ser importante para o sucesso da informação.

Pensando nisso, verifica-se a indispensabilidade de um sistema onde é possível expor de forma mais precisa, as notícias referentes à instituição. Além de facilitar a criação, edição e exclusão das informações, faça a integração dos meios de comunicação possibilitando a troca do conteúdo em todos os pontos em que está sendo exibido, utilizando um único sistema. Isso permite que os usuários tenham um acesso mais rápido as informações e uma melhor forma de interatividade do espectador com a notícia, disponibilizando o endereço para acesso completo a publicação no Facebook e exibindo os comentários feitos na mesma.

A escolha da rede social Facebook é baseada nas pesquisas realizadas por \citet{muchardie2016} na qual a apontam como uma das mais utilizadas, a maioria das pessoas que trabalham com \textit{marketing} digital a empregam para a exposição de conteúdos. Não somente pela grande quantidade de usuários que essa detêm, mas também por possuir funcionalidades como: compartilhamento fácil e suportar textos com mais de 63000 caracteres, visto que o limite é superior em relação a outras redes sociais como Instagram e Twitter, na qual a capacidade é de aproximadamente 2200 e 140 caracteres, respectivamente.

As pesquisas da \citet{emarketer} abordam que é crescente a utilização de celulares, portanto, um aplicativo para dispositivos móveis que exibe o mesmo conteúdo que são apresentados na rede social e nos televisores, disseminaria melhor as informações. Além disso, uma funcionalidade adicional estaria disponível apenas para estudantes e professores, essa ofereceria uma forma de contato entre os mesmos, observando que atualmente é realizada presencialmente, através de e-mails ou com uso de outros \textit{softwares} complementares.

\section{Proposta}
Com uso da arquitetura cliente-servidor e tendo como base o Sistema Inteligente de Divulgações do IFB em sua versão 2 (SIDv2) implementado por \citet{sobrinho2017}, é proposto a elaboração da terceira versão. Com o uso dos conceitos de sinalização digital e marketing digital, a proposta é implementar as melhorias para que o sistema seja mais completo na apresentação de conteúdos referentes ao IFB. 

O conteúdo criado, com auxilio do sistema, será vinculado a página do IFB no Facebook, sendo exibido nos painéis juntamente com os comentários realizados, devidamente moderados pelos administradores, em telas espalhadas em locais de maior movimento no \textit{Campus} Taguatinga do Instituto Federal de Brasília e nos dispositivos móveis de cada pessoa que possua o aplicativo instalado. 

O sistema visa proporcionar a integração dos meios usados atualmente para apresentação das informações referentes ao \textit{Campus}, além da inclusão de outros meios. Portanto, a ideia do SID é prover um meio que a comunidade possa interagir mais com as publicações acadêmicas e o administrador tenha uma maior facilidade de criação e edição das notícias, integrando vários serviços em um único.

Na versão para dispositivos móveis, além da apresentação das notícias, os professores e os alunos terão acesso a outra funcionalidade: o docente poderá enviar informações e avisos distintos para cada turma que ele leciona, enquanto os alunos poderão acessar cada mensagem enviada pelo professor para a turma em que ele está cadastrado.

A função de envio de mensagem é restrita e se dará através de um login, usando uma matrícula e senha fictícia cadastrados no bancos de dados que simula plataformas acadêmicas já existentes, onde não será possível o uso de dados reais por restrições de acesso as essas plataformas.

\section{Objetivos}
Levantar os principais requisitos de um sistema de sinalização e marketing digital integrado a rede social Facebook.

Estudar e detalhar a documentação da API do Facebook para integração de sistemas.

Definir e implementar o Sistema Integrado de Divulgações do IFB (SIDv3).

Implementar aplicativo móvel com funcionalidades adicionais, além das utilidades providas do SIDv3. Realizando o consumo de um API fictícia do Sistema de Gestão Acadêmica (SGA) visando uma futura integração à plataforma existente quando for disponibilizada.

\subsection{Objetivos Específicos}
	 \begin{itemize}
	\item Estudo e utilização de frameworks necessários para implementação do sistema, como ZEND, Doctrine, Cordova e Framework7.
	 	
	\item Disponibilização de uma API REST para interoperabilidade do SIDv3 com outros sistemas.
	
	\item Sugestão de uma proposta de implantação viável no \textit{Campus} Taguatinga utilizando computadores Raspberry Pi.
	\end{itemize}
	
\section{Metodologia}
A revisão de bibliografia será feita como meio de direcionamento do trabalho, onde serão usadas comparações entre ferramentas que apresentam o conceito de sinalização e \textit{marketing} digital, com o objetivo de avaliar as deficiências de cada uma delas, baseando-se nas necessidades do \textit{Campus}.

A análise servirá de forma a definir o que será necessário desenvolver ou alterar, para melhorar a maneira com que as informações são disseminadas, seguindo os conceitos que são considerados primordiais em um sistema de sinalização digital.

O estudo da documentação da Graph API e de suas ferramentas, tais como a Graph API Explorer, viabilizará a melhor integração do sistema com a rede social Facebook. Esses instrumentos serão utilizados para realização de testes práticos das diversas funcionalidade que a API dispõe, selecionando quais serão necessários a implementação, para que seja possível a recuperação e envio de dados e a melhoria no processo de \textit{login}.
	 
O SIDv2 será o sistema base e com auxílio das operacionalidades disponíveis na Graph API, será implementado no sistema as interações com o Facebook. As informações serão apresentadas em multiplataformas que podem ser televisores, painéis, páginas \textit{web} ou celulares. Essas informações podem ser alteradas acessando ao servidor com o sistema instalado e conectado a \textit{Internet}. Após serem criadas ou modificadas, as publicações poderão ser transmitidas e acessadas pelos clientes em distintas plataformas ao mesmo tempo.

A metodologia presente neste trabalho está direcionada ao desenvolvimento de ferramentas que possuem a finalidade de melhoria no processo de comunicação e veiculação de informações e notícias em diferentes plataformas. 

Todo o sistema, incluindo o móvel, seguirá o padrão de desenvolvimento ágil, com metodologia SCRUM, sendo definido \textit{sprints} semanais, para definição das funcionalidades a serem desenvolvidas ou melhoradas. 

\section{Organização do documento}
O Capítulo \ref{relacionados}, tem como objetivo situar o leitor sobre outras ferramentas que possuem conceitos que se assemelham com o SID, expondo como elas funcionam a fim de realizar um comparativo entre as funcionalidades que apresentadas e as que são requeridas.

O Capítulo \ref{cap:referencial}, tem como propósito apresentar ao leitor de forma detalhada cada conceito e ferramenta abordada no decorrer do documento.

No Capítulo \ref{cap:api}, Graph API, é apresentado um resumo detalhado das principais funcionalidades que a ferramenta Graph API oferece. Neste capítulo também são colocados alguns exemplos de aplicação destas diferentes funcionalidades.

As explicações detalhadas sobre cada detalhe do sistema, como a estrutura do SID, os detalhes de algumas interfaces e como ele realiza as suas ações em conjunto com a Graph API, apresentada no Capítulo \ref{cap:api}, são expostas no Capítulo \ref{cap:sid}.

Os resultados obtidos estão descritos no Capítulo \ref{resultados}, que aborda de forma resumida todo o resultado final que foi obtido no desenvolvimento do sistema, uma comparação entre os sistemas testados e o SIDv3, além das dificuldades encontradas.

E no Capítulo \ref{consideracoes}, considerações finais, repassa ao leitor os benefícios de se usar o SID e possíveis melhorias futuras que poderão ser implementadas.