\chapter[Introdução]{Introdução}
Apesar de muito utilizada, a definição exata de mídia é difícil de ser explicada. De acordo com \citet[p.51]{guazina2007}, o termo ``mídia'' é usado para representar os meios de comunicação em massa como jornais, televisão, rádio, cinema e \textit{Internet}.

Canais de televisão, outdoors ou até mesmo panfletos, em ambientes públicos ou privados podem introduzir diversas mudanças comportamentais e comerciais, por isso \citet[p.3]{escobar2007} propõe que a mídia é capaz de redefinir ``o modo como o homem se comunica e se relaciona com os semelhantes'', assim como \citet[p.54]{hjarvard2012} coloca que ela pode modificar a relação e o comportamento humano, criando o conceito de midiatização, que caracteriza a influência que a mídia possui sobre as pessoas.

\citet[p.53]{guazina2007} aponta que os meios de comunicação influenciam no processo de construção e formação das consciências. Além disso, \citet{silva2007} indica que a falta de acesso ou interesse por parte da população influencia no questionamento da veracidade das informações que são repassadas.

Com o passar dos anos, a mídia foi obtendo novas formas, as divulgações de forma estática e mais tradicionais (revistas, jornais e canais de televisão) foram deixando de serem os meios mais eficientes de se expor um conteúdo ou propaganda. Para \citet{meditsch2001}, o advento da \textit{Internet} e sua popularização trouxeram uma ameaça para estes meios, então foi necessário que novas formas de expor conteúdos fossem pensadas e desenvolvidas em conjunto com essa nova ferramenta.

Nos meios de comunicação em massa sempre estão presentes os mais diversos tipos de propagandas e divulgações, o \textit{marketing} é responsável por criar e melhorar as ferramentas que tem como objetivo influenciar pessoas a adquirir ou aderir determinado produto ou serviço. Assim como a mídia, novas formas e ferramentas de \textit{marketing} também foram surgindo, ampliando os meios por onde as informações são repassadas. 

Para a ampliação, foi necessário que os caminhos da publicidade e da tecnologia se convergissem para se então definir um novo conceito de \textit{marketing}, o \textit{marketing} digital que conta com novas tecnologias de comunicação, onde para \citet[p.2]{escobar2007}, colocam a interatividade em evidencia, então, a utilização de novas ferramentas mais interativas com seu receptor tornam a leitura menos monótona e passível de atingir uma maior atenção do espectador de forma mais consistente.  

As vantagens para a utilização da interatividade estão presentes em diversas formas, como por exemplo, para \citet[p.4]{escobar2007}, os atuais meios de veiculação de informações (televisão e radio) oferecem a possibilidade de se ter a informação em tempo real, mas quando se tem o advento da \textit{Internet}, coloca-se a possibilidade de interação com a informação que é recebida, isto quase que instantaneamente, onde os integrantes atuam de forma simultânea, comentando ou opinando sobre aquele determinado assunto. Para \citet{deuze2002}, o surgimento da \textit{Internet} traz a possibilidade do público ``responder, interagir ou mesmo customizar certas histórias''. 

A evolução do \textit{marketing} digital trouxe consigo a união das mídias sociais e dispositivos móveis. Isso permite que se tenha receptores dos informes nas mais variadas localidades, além da possibilidade interação com essas noticias que estão sendo transmitidas. Para \citet{santos2014}, a utilização do \textit{marketing} na internet, favorece a criação de um \textit{marketing} em ambiente digital.

Pensando na maior abrangência e na interatividade, surge o conceito de sinalização digital, que para \citet[p.37]{machado2010} dar-se com o uso de telas espalhadas por diversos pontos para apresentação das mais diferentes informações, repassadas via internet para as telas. Com isso, pode ter receptores nos mais diversos locais, independente de cidade, estado ou país. As telas ou televisores apresentam informações e propagandas de forma dinâmica, nos pontos em que foram instaladas e com a possibilidade de gerenciá-las remotamente de acordo com a necessidade. 

As mídias sociais também possuem seu importante papel para disseminação de uma informação ou conteúdo e vem se tornando uma das ferramentas mais atraentes para divulgações. \cite{rosa2010} aborda que cada vez mais, elas fazem parte do cotidiano das pessoas, com uso cada vez maior. Não apenas por ser um dos meios mais acessados atualmente, mas também por conta da maior facilidade de interações das notícias com os usuários.

Com o passar dos anos os dispositivos móveis estão sendo cada vez mais utilizados pelas pessoas, isso vem atraindo cada vez mais o seu foco como ferramenta para divulgação de informações. Não somente pelo grande número de dispositivos, mas também pela integração com as redes sociais que esses dispositivos oferecem, pesquisas feitas pela agência eMarketer afirmam que até 2019 mais de 80\% das pessoas que acessam a \textit{Internet}, usarão o celular para acessa-la.

O Instituto Federal de Educação Ciência e Tecnologia de Brasília -- Câmpus Taguatinga (IFB), para divulgação de notícias e as mais variadas informações, se utiliza principalmente de suas páginas \textit{web} e Facebook. Para o uso desses meios, é necessário que os administradores façam publicações independentes para cada uma das páginas, além disso, as páginas não são bem divulgadas, muitas vezes os estudantes e visitantes nem ficam sabendo das notícias que lá são publicadas. Além disso, a única forma de contato entre professores e alunos é por email ou com uso de aplicativos externos a instituição, necessitando que cada estudante encaminhe o seu email pessoal para que o professor possa entrar em contato.

Para \citet{pinheiro2010}, a comunicação interna é de fato importante para o sucesso de uma organização, em uma instituição educacional não é diferente, onde é interessante que os alunos e os professores sejam informados de futuros eventos, palestras e notícias. Os atuais veículos institucionais do IFB não possuem integração e atuam de forma independente, o que acabam degradando a qualidade e a disseminação dos informes. Além disso, existe a complicada interação dos usuários com os atuais meios de veiculação de notícias, para \citet{santos2014}, a interatividade é importante para que se possa ``estreita o relacionamento com o público'', no IFB essa interatividade é baixa, o que acaba afetando o interesse de acompanhá-la por parte do usuário.

O Sistema Integrado de Divulgação de Informações do IFB versão 3 (SIDv3) oferece uma maior visibilidade das notícias, sendo possível, por meio painéis espalhados pelo campus, uma melhor interação da comunidade com as notícias, apresentando nos painéis os comentários que foram publicados nas mídias sociais, além de uma melhor forma de comunicação entre professor e aluno, oferecendo um aplicativo móvel que simula um sistema de comunicação institucional, integrado ao Sistema de Gestão Acadêmica (SGA).

\section{Motivação}
\citet{bianchi2006} coloca que algo inédito e atual atrai a atenção do telespectador, a atenção dele a uma determinada informação que está sendo apresentada é algo crucial para o sucesso das notícias que estão sendo exibidas, quanto melhor disseminada as notícias forem, maior a chance de sucesso. Nesse intuito, a interatividade e o dinamismo podem ser considerados algo inédito e atual, o uso de ferramentas do cotidiano das pessoas, poderia despertar e manter a atenção do usuário, fazendo-o ter interesse em acompanhar e participar de uma determinada noticia ou matéria.

Atualmente, o IFB utiliza o seu perfil do Facebook, sua página oficial e os murais de cada \textit{Campus} como formas de veiculação das notícias referentes a instituição, sejam elas notícias de eventos ou institucionais. Para criação de uma nova notícia é necessário o administrador acessar cada página e realizar uma postagem independente em cada uma delas, além da necessidade de fixação de algumas dessas informações nos murais. 

\citet{pinheiro2010} cita a importância da comunicação interna, quando se tem um sistema de divulgação defasado e rígido, as informações podem não atingir o resultado esperando, então é preciso ter uma melhor forma de exposição e edição destas notícias no IFB, uma forma mais precisa e ágil de apresentação delas para alunos, professores e visitantes, contando com um sistema mais interativo, para que se possa ter um melhor participação do público com a notícia como \citet{santos2014} aborda ser importante para o sucesso da informação.

Pensando nisso, vê-se a indispensabilidade de um sistema onde é possível expor de forma mais precisa, as notícias referentes à instituição. Além de facilitar a criação, edição e exclusão, faça a integração dos sistemas, contando com a simplicidade na troca de conteúdo, agilidade de acesso e uma forma de interatividade do espectador com a notícia, por meio de apresentação de comentários que são feitos na publicação postada no Facebook.

As pesquisas da \cite{emarketer} apontam que cada vez mais pessoas usam os celulares, então, um aplicativo para dispositivos móveis serviria de forma a disseminar melhor as informações que também seriam exibidas no aplicativo, além de oferecer uma forma de contato entre professor e turma, que atualmente é feita geralmente por meio físico, e-mails ou usando outros \textit{softwares} complementares.

\section{Proposta}
Com uso da arquitetura cliente-servidor e tendo o  Sistema Inteligente de Divulgações do IFB em sua versão 2 (SIDv2) implementado por \cite{sobrinho2017} como base, é proposto a elaboração da terceira versão. Com o uso dos conceitos de sinalização digital e marketing digital, a proposta é fazer com que o sistema apresente conteúdos referentes ao IFB integrados com a página do IFB no Facebook, apresentando postagens e comentários devidamente moderados pelos administradores, nas telas espalhadas por locais de maior movimento do Câmpus Taguatinga do Instituto Federal de Brasília ou nos dispositivos móveis de cada pessoa. 

O sistema proposto visa proporcionar a integração dos meios usados atualmente para apresentação das informações referentes ao Câmpus, além da inclusão de outros meios. O sistema fará a integração entre a página do Facebook da instituição, televisões espalhados pelo campus e dispositivos móveis dos alunos e professores.

Portanto, a ideia do SID, é prover a melhor troca do espectador com as publicações acadêmicas e o administrador ter uma maior facilidade de criação e edição das publicações vinculadas as redes sociais, integrando vários serviços em um único.

Na versão para dispositivos móveis, além da apresentação das notícias, os professores e os alunos terão acesso a uma funcionalidade extra, o docente poderá enviar informações e avisos distintos para cada aluno ou turma, enquanto os alunos poderão acessar a cada mensagem enviada pelo professor para a turma em que ele está cadastrado. Isso se dará através de um login, usando uma matricula e senha fictícia cadastrado no bancos de dados que simula plataformas acadêmicas já existentes, onde não será possível o uso de dados reais por restrições de acesso as essas plataformas.

\section{Objetivos}
Levantar os principais requisitos de um sistema de sinalização e marketing digital integrado a rede social Facebook.

Estudar e detalhar a documentação da API do Facebook para integração de sistemas.

Definir e implementar o Sistema Integrado de Divulgações do IFB (SIDv3).

Implementar aplicativo móvel com funcionalidades adicionais, além das utilidades providas do SIDv3. Realizando o consumo de um API fictícia do Sistema de Gestão Acadêmica (SGA) visando uma futura integração à plataforma existente quando for disponibilizada.

\subsection{Objetivos Específicos}
	 \begin{itemize}
	\item Estudo e utilização de frameworks necessários para implementação do sistema, como ZEND, Doctrine, Cordova e Framework7.
	 	
	\item Disponibilização de uma API REST para interoperabilidade do SIDv3 com outros sistemas.
	
	\item Sugestão de uma proposta de implantação viável no Campus Taguatinga utilizando computadores Raspberry Pi.
	\end{itemize}
	
\section{Metodologia}
A revisão de bibliografia será feita como meio de direcionamento do trabalho, onde serão usadas comparações entre ferramentas desenvolvidas com o propósito principal de sinalização e \textit{marketing} digital. Partindo de tais soluções com o objetivo de avaliar os pontos negativos baseando-se nas necessidades do Campus e então juntar ao processo de desenvolvimento os elementos que forem selecionados como principais e que são responsáveis por efetivar a disseminação da informação ao sistema de forma descentralizada e com o auxílio de ferramentas utilizadas no contexto \textit{web}.

O estudo da documentação da Graph API e de suas ferramentas, tais como a Graph API Explorer, viabilizará a integração do sistema com a rede social Facebook. Estes instrumentos serão utilizados para realização de testes práticos de funcionalidade como a recuperação de dados da rede social para integração com o sistema e com o estudo da documentação e serão analisadas possíveis implementações dessas funcionalidades.
	 
Ao utilizar o SIDv2 como sistema base e auxílio das operacionalidades  disponíveis na Graph API, será implementado no sistema as interações com as redes sociais. As informações serão apresentadas em multiplataformas que podem ser televisores, painéis, páginas \textit{web} ou celulares. Essas informações podem ser alteradas acessando ao servidor com o sistema instalado e conectado a \textit{Internet}. Após serem criadas ou modificadas, as publicações poderão ser transmitidas e acessadas pelos clientes em distintas plataformas ao mesmo tempo.

A metodologia presente neste trabalho está direcionada ao desenvolvimento de ferramentas que possuem a finalidade de melhoria no processo de comunicação e veiculação de informações e notícias em diferentes plataformas. 

Todo o sistema, incluindo o móvel, seguirá o padrão de desenvolvimento ágil, com metodologia SCRUM, sendo definido sprints semanais, comumente marcada as quartas feiras para definição das funcionalidades a serem desenvolvidas ou melhoradas. 

\section{Organização do documento}
O Capítulo \ref{relacionados}, de Trabalhos Relacionados, tem como objetivo situar o leitor sobre ferramentas que possuem conceitos que se assemelham com o SID, expondo como elas funcionam a fim de realizar um comparativo entre as funcionalidades que apresentadas e as que são requeridas.

A diante, no Capítulo \ref{cap:referencial}, Referencial Teórico, tem como propósito apresentar ao leitor de forma detalhada cada conceito e ferramenta abordada no decorrer do documento.

No Capítulo \ref{cap:api}, Graph API, é apresentado um resumo detalhado das principais funcionalidades que a ferramenta Graph API oferece. Neste capítulo também são colocados alguns exemplos de aplicação destas diferentes funcionalidades.

A explicação detalhada sobre os aspectos intrínsecos do sistema, como a estrutura do SID e de como ele realiza as suas ações em conjunto com a Graph API apresentada no Capítulo \ref{cap:api} são expostas no Capítulo \ref{cap:sid}.

Os resultados obtidos estão expostos no Capítulo \ref{resultados}, que aborda de forma resumida todo o resultado final que foi obtido no desenvolvimento do sistema, além das dificuldades encontradas.

Por fim, o Capítulo \ref{consideracoes}, considerações finais, repassa ao leitor os benefícios de se usar o SID e o horizonte de possibilidades que o SID viabilizou.

