\begin{resumo}
Este trabalho apresenta o Sistema Integrado de Divulgação de Informações do IFB Câmpus Taguatinga - SID que, por meio de uso dos conceitos de sinalização digital e marketing digital, visa veicular notícias em forma de publicações que são repassadas por meio de televisões no ambiente do \textit{Campus} ou fora dele por meio de um aplicativo móvel instalado nos celulares. 

Foram realizadas diversas alterações no sistema inicial chamado de SIDv2. As alterações vão desde modificações na arquitetura existente para implementação de novas funcionalidades, até a implementação de um aplicativo exclusivos para celulares. As alterações foram realizadas de modo a flexibilizar a implantação de novas funcionalidade e serviços, para que outros sistemas pudessem fazer de um mesmo sistema. 

O SIDv3 também passou a possuir a integração completa com a rede social Facebook, disponibilizando a possibilidade de realização de publicações em páginas do Facebook e apresentação dos conteúdos referentes a essas publicações. Com isto é realizado uma junção de meios que anteriormente eram distintos no IFB.

Além disso, o aplicativo \textit{mobile} servirá de forma a repassar as divulgações criadas, além de realizar o consumo de uma API fictícia para interação entre alunos e professores com a troca de mensagens. Esse consumo de API possibilita uma futura integração com o Sistema de Gestão Acadêmica - SGA.
    
 \noindent
 %\textbf{Palavras-chaves}: latex. abntex. editoração de texto.
\end{resumo}
