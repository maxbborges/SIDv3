\begin{resumo}
O \textit{marketing} tem por objetivo influenciar as pessoas com o uso dos diversos aparatos, a sinalização digital é um deles. Neste documento, com uso dos conceitos de \textit{marketing} e sinalização digital, são implementadas melhorias e novas funcionalidades no Sistema Inteligente de Divulgação de Informações do IFG - Formosa (SID). A nova versão do sistema apresenta várias diferenças em relação a sua versão anterior. Entre elas está a completa integração ao Facebook, que antes era limitada; mais interatividade na apresentação das notícias, possibilitando o usuário interagir através da rede social com as divulgações inseridas; criação de um aplicativo móvel capaz de repassar as divulgações cadastradas no sistema para o dispositivo do usuário e que possibilita uma comunicação do aluno com o professor. Este último aplicativo foi desenvolvido pensando em uma futura integração com o Sistema de Gestão Acadêmica do IFB.


%Estas vão desde a interface que é apresentada ao usuário até ações internas que o sistema realiza, por exemplo, apresentação de mensagens de erro, restrições de acesso, cadastramento de múltiplos administradores e inserção de interatividade. Além disso, é apresentado um protótipo de aplicativo móvel que dispõe de funcionalidades já existentes na versão \textit{Web} do SID e apresentação de novas.

% Este trabalho apresenta o Sistema Integrado de Divulgação de Informações do IFB Câmpus Taguatinga - SID que, por meio de uso dos conceitos de sinalização digital e marketing digital, visa veicular notícias em forma de publicações que são repassadas por meio de televisões no ambiente do \textit{Campus} ou fora dele por meio de um aplicativo móvel instalado nos celulares. 

% Foram realizadas diversas alterações no sistema inicial chamado de SIDv2. As alterações vão desde modificações na arquitetura existente para implementação de novas funcionalidades, até a implementação de um aplicativo exclusivos para celulares. As alterações foram realizadas de modo a flexibilizar a implantação de novas funcionalidade e serviços, para que sistemas distintos pudessem fazer uso de um mesmo sistema. 

% O SIDv3 também passou a possuir a integração completa com a rede social Facebook, disponibilizando a possibilidade de realização de publicações em páginas do Facebook e apresentação dos conteúdos referentes a essas publicações. Com isto é realizado uma junção de meios que anteriormente eram distintos no IFB.

% Além disso, o aplicativo \textit{mobile} servirá de forma a repassar as divulgações criadas, além de realizar o consumo de uma API fictícia para interação entre alunos e professores com a troca de mensagens. Esse consumo de API possibilita uma futura integração com o Sistema de Gestão Acadêmica - SGA.

\vspace{3cm}
\noindent
\textbf{Palavras-chaves}: Aplicativo Móvel. Sistema Integrado. Facebook.
\end{resumo}
