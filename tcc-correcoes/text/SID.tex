\chapter[SID]{SID}
\label{cap:sid}
O Sistema Integrado de Divulgação de informações do IFB \textit{Campus} Taguatinga - SID, tem como principal objetivo fornecer informações através de sinalização em sua forma digital, oferecendo de forma fácil o gerenciamento dessas informações que serão repassadas. 

Neste Capítulo, será feito o detalhamento do SID, primeiramente apresentando os detalhes de forma superficial, macroscópica. Posteriormente é apresentado de maneira microscópica, apresentando cada detalhe de como e onde são realizadas as demandas de cada elemento presente no sistema. 

\section{Visão Geral} 
O SID está estruturado em duas vertentes, o \textit{Web} e o móvel, ele foi desenvolvido com o objetivo de oferecer um sistema de divulgação que realiza integração com as redes sociais de forma ágil, intuitiva, dinâmica e amigável para os administradores e para os usuários comuns, além de melhorar a efetividade do processo de disseminação das informações referentes ao IFB \textit{Campus} Taguatinga. Atendendo ao objetivo principal, que é divulgação de informações através de uma plataforma \textit{Web} e móvel.

Necessitando sempre do uso da rede para realizar atualizações, o SID está dividido em dois módulos e um aplicativo. O primeiro módulo é o administrador, onde é possível fazer o gerenciamento completo do conteúdo que será apresentado no segundo módulo chamado cliente. No segundo módulo será apresentado as informações que foram cadastradas no módulo administrador, onde então serão propagadas por monitores ou celulares. O aplicativo é um sistema móvel, onde o usuário poderá ter acesso a todas as divulgações assim como no módulo cliente, mas com o incremento de troca de mensagens entre professores e alunos através do consumo de uma API fictícia que simula a interface a ser oferecida pelo Sistema de Gestão Acadêmica do IFB (SGA).

A divisão de módulos foi feita para que se possa atender a arquitetura cliente-servidor, ela se mostrou necessária para minimizar o processamento do módulo cliente e do aplicativo que se utiliza do módulo cliente, centralizando o processamento das informações em um sistema mais robusto, ficando a cargo dos clientes exibirem as informações recebidas e realizar pequenos processamentos.

O sistema WEB foi desenvolvido na linguagem PHP, usada para estruturação do projeto, a linguagem JavaScript para realização das requisições de trocas de informações entre os módulos e as linguagens HTML e CSS para desenvolvimento das telas que serão apresentadas no sistema. Usa-se também o Banco de dados PostgreSQL para armazenamento e persistência das informações.

Baseado em desenvolvimento ágil, com metodologia SCRUM, foram definidos sprints semanais, para definição das funcionalidades a serem desenvolvidas ou melhoradas. Esta metodologia se torna importante para que o progresso seja acompanhado em cada parte do seu desenvolvimento.

No desenvolvimento, foram usados alguns \textit{frameworks}. O primeiro deles, usado em conjunto com o PHP, foi o ZEND que tem a finalidade de estruturar o código em modelo, visão e controle. Outro \textit{framework} foi o Doctrine, usado para efetivação da comunicação entre o banco de dados e a orientação a objeto. O terceiro foi o Framework7 (F7), em conjunto com o Cordova, para que fosse possível o desenvolvimento do aplicativo móvel. 

Se utilizando dos conceito de UML abordados na Seção \ref{sec:uml}, é  feito a representação dos elementos, demostrando cada artefato do SIDv3, como o diagrama de classe, de sequência e seus casos de uso.

Uma apresentação mais detalhada do funcionamento interno de cada tela disponível do SID é apresentada no Apêndice \ref{apendice}.

\section{Modulo Administrador}
Com o uso da arquitetura cliente-servidor, esse módulo fica situado no servidor e é responsável por conceder ao usuário administrador a funcionalidade de gerenciar todas as informações do sistema podendo inserir, alterar, e retirar as divulgações, fazendo todo o gerenciamento das informações que são repassadas ao módulo cliente e móvel.

O modelo de casos de uso que representa o módulo é apresentado na Figura \ref{fig:casosDeUsoADM}. Ele tem a finalidade de demostrar os principais processos que o administrador pode executar no módulo.

\begin{figure}[H]
\centering
\includegraphics[scale=0.4]{figuras/casosDeUsoADM}
\caption{Casos de uso da ações do módulo administrador}
\label{fig:casosDeUsoADM}
\end{figure}

O acesso ao módulo é restrito, sendo necessário autenticação e para esse fim foi escolhido a ferramenta ``\textit{login} com o Facebook'', que a própria rede social disponibiliza como explicado na Seção \ref{sec:autenticacao}. Optou-se pelo uso dela, pois provê maior segurança, sendo baseada em sistemas criptográficos e também por facilitar o processo de recuperação de alguns dados básicos que são necessários para o processo de autenticação com o sistema. O diagrama de sequência que representa o processo de autenticação está representado na Figura \ref{fig:sequencialogin}.

\begin{figure}[H]
\centering
\includegraphics[scale=0.3]{figuras/sequencialogin}
\caption{Sequencia para efetivação de \textit{login}}
\label{fig:sequencialogin}
\end{figure}

A estruturação do banco de dados é composta por três tabelas, sendo mapeadas com as entidades ``config'', ``divulgacao'' e ``adm''. A função da tabela ``divulgacao'' é armazenar todos os objetos relacionados as divulgações que serão exibidas no cliente e no aplicativo. A tabela ``adm'' é usada para armazenar as informações referentes aos administradores do sistema. A tabela ``config'' é destinada a armazenar informações únicas para efetivação da comunicação entre aplicativo e Graph API.

\begin{figure}[H]
\centering
\includegraphics[scale=1.2]{figuras/entidaderelacionamento}
\caption{Modelo entidade e relacionamento}
\label{fig:casosDeUso}
\end{figure}

A Figura \ref{fig:diagramaclasseADM}, mostra a representação e o relacionamento entre as classes disponíveis no módulo. No diagrama de sequência \ref{fig:sequenciainserir} é exposta a sequência de uso das classes ao inserir uma nova publicação no sistema. 
\begin{figure}[H]
\centering
\includegraphics[scale=0.7]{figuras/diagramaclasseADM}
\caption{Diagrama de classe do módulo administrador}
\label{fig:diagramaclasseADM}
\end{figure}

 \begin{figure}[H]
\centering
\includegraphics[scale=0.4]{figuras/sequenciainserir}
\caption{Diagrama de sequência para inserção}
\label{fig:sequenciainserir}
\end{figure}

Após a efetivação da autenticação, o usuário poderá ter acesso a todas as funcionalidades disponíveis no sistema, divididas entre as telas de ``inserir'' e ``Home''. A tela de inserir, é usada para criação de uma nova publicação, enquanto a tela Home oferece a opção de listar, detalhar, deletar e editar essa publicações que foram criadas.

A Figura \ref{fig:administrador1}, representa a visão do operador da tela de inserção, onde cada elemento da página está enumerado.

\begin{figure}[H]
\centering
\includegraphics[scale=0.4]{figuras/administrador1}
\caption{Página de inserção do módulo administrador.}
\label{fig:administrador1}
\end{figure}

Os elementos presentes na tela de inserir são listados abaixo, juntamente com suas funcionalidades: 

\begin{enumerate}
   \item Legenda: É um campo de texto destinado a informação a ser repassada de forma sucinta e clara ao usuário final por intermédio cliente ou móvel. 
   \item Texto: É todo o texto que será enviado ao Facebook, contendo informações como descrição da publicação, data e horário da realização e um possível \textit{link} para acesso a mais informações. 
   \item Data de Início: É a data inicial em que a publicação começará a ser exibida no cliente e no móvel.
   \item Data de Término: É a data final em que a publicação deixará de ser exibida no cliente e no móvel.
   \item Imagem: Será a imagem enviada para o Facebook, para o cliente e para o móvel. Essa imagem pode ser um \textit{banner} de apresentação de um evento, por exemplo.
 \end{enumerate}

Todos os campos são de preenchimento obrigatório. O campo 5 é repassado para a API como parâmetro \textit{source}, já o campo 2 é repassado como parâmetro \textit{messagem}, os outros campos são enviados somente para o banco de dados.

Se durante o envio alguma das transações não forem efetivadas, será retornado a mensagem de erro na tela, assim como retornará a mensagem de sucesso, caso não ocorra nenhum erro.

O texto do campo 1 poderá conter letras ou números, entretanto possui a limitação máxima de 80 caracteres, isso é necessário para facilitar a leitura do usuário ao texto que será inserido, pois ele será armazenado no banco de dados e recuperado no módulo cliente para exibição.

Os outros campos não possuem limitação de caracteres, entretanto os campos 3, 4 e 5 possuem restrição de tipo de entrada, onde os campos 3 e 4 obrigatoriamente será um número inteiro e o campo 5, será um arquivo de imagem do tipo JPEG, JPG, GIF ou PNG.

Os campos 3 e 4 são botões do tipo dropdown, onde é preciso selecionar um dos valores que aparecem em cada um dos campos. Os possíveis valores para o campo ``dia'' vão de 1 a 31, os valores do campo ``mês'' vão de 1 a 12, enquanto os valores do campo ``ano'' irão do ano atual a dois anos seguintes.

A página ``Home'' é responsável por listar todas as publicações criadas, a visão do usuário será a mesma que representada na Figura \ref{fig:administrador2}. As funcionalidades de cada elemento da página são descritas logo abaixo. Essa é a página inicial do sistema, nela é possível ter acesso a página de inserção, listagem, detalhamento e ao botão excluir.

\begin{figure}[H]
\centering
\includegraphics[width=\textwidth]{figuras/administrador2}
\caption{Página de listagem  do módulo administrador.}
\label{fig:administrador2}
\end{figure}

\begin{enumerate}
   \item Home: Botão para acesso a página inicial, onde são listadas todas as publicações criadas.
   \item Inserir: Botão para acesso a página de inserção.
   \item Detalhes: Ao clicar sobre a foto é possível obter dados específicos da publicação selecionada, tais como: Legenda, ID do autor, link de acesso a publicação, data de início, data de término, ID da publicação.
   \item Editar: Ao clicar no botão editar, será apresentado uma nova página edição dos dados específicos da publicação selecionada.
   \item Excluir: Exclui a publicação selecionada.
 \end{enumerate}

\subsection{Módulo API}
O módulo API faz parte do módulo administrador e possui a finalidade de recuperar os dados das publicações realizadas pelo administrador no Facebook, estruturá-los de forma que atendam a arquitetura REST e realizar o controle de alguns dados. Isso é crucial para que o módulo cliente e o aplicativo possam realizar requisições de solicitação ou envio de dados.

O controle de dados realizado pelo módulo API é indispensável para que conteúdos como comentários inadequados não sejam exibidos. Para esse controle é efetuado uma varredura de todos os usuários que curtiram cada comentários da publicação, caso seja encontrado a curtida de um dos administradores cadastrados no banco de dados o comentário é aprovado para ser exibido.

Além disso, as requisições para esse módulo devem seguir uma sintaxe de acordo com os dados desejados. Como mostra a Listagem \ref{lst:bd}, onde é requisitado as informações das publicações presentes no banco.

\begin{lstlisting}[caption={Requisitando dados para divulgação},label={lst:bd}]
	http://{IP DO SERVIDOR}:{PORTA}/bd
\end{lstlisting}

A resposta para a solicitação da Listagem \ref{lst:bd} é um arquivo do tipo JSON contendo o conteúdo apresentado no Listagem \ref{lst:retornobd}. O uso de cada um dos itens será explicado na Seção \ref{sec:cliente}.

\begin{lstlisting}[caption={Retorno da requisição \ref{lst:bd}},label={lst:retornobd}]
{
	0:[
		"bd": [
			"linkqr": {LINK DA DIVULGACAO}
			"legenda": {LEGENDA QUE SERA EXIBIDA}		
		],
		"comentarios": null,
		"imagem": []
	1:[
		"bd": [
			"linkqr": {LINK DA DIVULGACAO}
			"legenda": {LEGENDA QUE SERA EXIBIDA}
		],
		"comentarios": [
			0: [
				"created_time": {DATA DA CIRACAO},
				"from": [
					"nome": {NOME DO USUARIO},
					"id": {ID DO USUARIO}
				],
				"message": {MENSAGEM APRESENTADA},
				"id": {ID DA PUBLICACAO},
				"urlFoto": {URL DA FOTO DE PERFIL DO USUARIO}
			],
		],
		"imagem": {IMAGEM DE EXIBICAO, CODIFICADA}
	]
}
\end{lstlisting}

A propriedade ``bd'', contém uma \textit{array} de duas posições, contendo 2 \textit{strings}, sendo elas legenda e link. Esses dados foram armazenados pelo servidor no banco de dados, na criação da publicação. A propriedade ``comentarios'' é um array de \textit{strings} que pode possuir zero ou mais itens. Cada item representa um comentário da publicação, cada qual contendo data de criação, autor, foto de perfil e identificador do comentário. Já a ``imagem'' é uma \textit{string} que representa uma imagem codificada em base64 a ser decodificada pelo módulo cliente.

O módulo API tem a função de, em conjunto com o banco de dados, realizar a simulação de um sistema de gestão acadêmico, sendo possível consultar alguns dados através de requisições. Os dados que podem ser recuperados, variam de acordo com a sintaxe utilizada.

É possível requisitar diversos dados distintos ou todos eles juntos. 
No Listagem \ref{lst:todos} é feito a requisição de todos os dados armazenados, que retornará um JSON de diversas posições, onde cada posição representa um mensagem enviada por um professor.
\begin{lstlisting}[caption={Requisitar todos os dados},label={lst:todos}]
	http://{IP DO SERVIDOR}/mobile
\end{lstlisting}

\begin{lstlisting}[caption={Retorno da requisição \ref{lst:todos}},label={lst:retornotodos}]
	0:[
		"id_mensagem": {NUMERO DA MENSAGEM NO SERVIDOR},
		"matricula_professor": {MATRICULA DO PROFESSOR},
		"nome_professor": {NOME DO PROFESSOR QUE ENVIOU},
		"id_turma": {IDENTIFICACAO DA TURMA},
		"mensagem": {MENSAGEM ENCAMINHADA PELO PROFESSOR}
	],
	1:[		
		"id_mensagem": {NUMERO DA MENSAGEM NO SERVIDOR},
		"matricula_professor": {MATRICULA DO PROFESSOR},
		"nome_professor": {NOME DO PROFESSOR QUE ENVIOU},
		"id_turma": {IDENTIFICACAO DA TURMA},
		"mensagem": {MENSAGEM ENCAMINHADA PELO PROFESSOR}
	]
\end{lstlisting}

A inclusão de um novo parâmetro na requisição da Listagem \ref{lst:todos} torna a busca mais específica. Por exemplo, pode ser usado ``/professores'', ``/alunos/'', ``/turmas/'' ou ``/mensagens/'', isso retornará um JSON contendo uma lista somente com os dados solicitados. Na Listagem \ref{lst:professores}, o retorno será uma lista com os dados de todos os professores cadastrados como mostra a Listagem \ref{lst:retornoprofessores}.

\begin{lstlisting}[caption={Requisitar lista de dados especifica},label={lst:professores}]
	http://{IP DO SERVIDOR}/mobile/professores/
\end{lstlisting}

\begin{lstlisting}[caption={Retorno da requisição \ref{lst:bd}},label={lst:retornoprofessores}]
	0:[
		"matricula": {MATRICULA DO PROFESSOR},
		"nome": {NOME DO PROFESSOR QUE ENVIOU},
		"senha": {SENHA CRIPTOGRAFADA}
	],
	1:[		
		"matricula": {MATRICULA DO PROFESSOR},
		"nome": {NOME DO PROFESSOR QUE ENVIOU},
		"senha": {SENHA CRIPTOGRAFADA}
	]
\end{lstlisting}

O diagrama de classe que representa esse módulo é descrito na Figura \ref{fig:diagramaclasseAPI}. Cada uma das classes é usada para uma determinada requisição.
\begin{figure}[H]
\centering
\includegraphics[scale=0.5]{figuras/diagramaclasseAPI}
\caption{Diagrama de classe do módulo API}
\label{fig:diagramaclasseAPI}
\end{figure}

\section{Módulo Cliente}
\label{sec:cliente}
O modulo cliente tem a função de apresentar as divulgações de maneira atrativa, intuitiva e dinâmica para o usuário. O seu caso de uso é apresentado na Figura \ref{fig:casosDeUsoCliente}. A estrutura e organização do que será apresentado, está disposto na Figura \ref{fig:cliente1} e o seu diagrama de classe é apresentado na Figura \ref{fig:diagramaclasseCLIENTE}.

\begin{figure}[H]
\centering
\includegraphics[scale=0.6]{figuras/CasosDeUsoCliente}
\caption{Casos de uso da ações do módulo cliente}
\label{fig:casosDeUsoCliente}
\end{figure} 

\begin{figure}[H]
\centering
\includegraphics[scale=0.4]{figuras/diagramaclasseCLIENTE}
\caption{Diagrama de classe do módulo cliente}
\label{fig:diagramaclasseCLIENTE}
\end{figure}

Os elementos que compõem a página que será apresentada são divididos em cinco itens, que estão descritos abaixo.

\begin{enumerate}
   \item Imagem: O conteúdo poderá ser apresentado como uma imagem estática ou como GIF (imagem com animação), não oferecendo suporte a vídeos, até o momento, mas que poderá ser implementado em versões futuras. 
   \item Legenda: A legenda é apresentada em movimento linear da direita para a esquerda, possibilitando a leitura de forma dinâmica de toda a frase.
   \item QR Code: Imagem, que por meio de um aplicativo celular, possibilita a leitura que contém o \textit{link} de acesso a publicação publicada na rede social Facebook.
   \item Horário: Relógio que apresentará a data e hora atual.  
   \item Comentários: Espaço que será apresentado comentários publicados e devidamente moderados.
 \end{enumerate}
  
\begin{figure}[H]
\centering
\includegraphics[scale=0.3]{figuras/cliente1}
\caption{Página do cliente}
\label{fig:cliente1}
\end{figure}

Consumindo uma API REST com requisições do tipo GET feitas a URL da Listagem \ref{lst:bd}, o módulo cliente receberá como resposta um JSON, contendo o número de posições equivalentes ao número de publicações válidas. Cada posição desse JSON terá três variáveis, são elas: ``bd'',``comentarios'' e ``imagem'', como apresentado na Listagem \ref{lst:retornobd}. 

O conteúdo recuperado da variável ``bd'', será apresentado nos itens 2 e 3, respectivamente. Já o conteúdo recuperado da variável ``comentarios'' será apresentado no item 5. Enquanto os da variável ``imagem'', será descriptografado e apresentado em forma de imagem no item 1.

O campo 5 irá apresentar 5 elementos recuperados na variável ``comentarios''. Esses elementos serão: a imagem de perfil, o nome, o comentário, a data e a hora da publicação do comentário feito pelo usuário. Antes da apresentação das informações de comentários, é escolhido um número randômico que representará a posição do array do comentário que será apresentado.

\section{Aplicativo móvel}
Desenvolvido com auxílio dos \textit{Frameworks} Cordova, Framework7 e Dom. O aplicativo móvel é responsável por apresentar de forma pública as divulgações, assim como o módulo cliente, com algumas funcionalidades a mais que são restritas.

As funcionalidades adicionais do aplicativo são as de disponibilizar uma melhor forma de comunicação entre professores e alunos, necessitando apenas de autenticação. Para autenticação é preciso que o aluno ou o professor tenha uma matrícula e senha cadastradas no banco.

Os casos de uso do aplicativo são demostrados na Figura \ref{fig:casosdeusomobile}, enquanto o modelo entidade e relacionamento é apresentado na Figura \ref{fig:entidaderelacionamentomobile}.
\begin{figure}[H]
\centering
\includegraphics[scale=0.4]{figuras/CasosDeUsoMobile}
\caption{Casos de uso aplicativo móvel.}
\label{fig:casosdeusomobile}
\end{figure}

\begin{figure}[H]
\centering
\includegraphics[scale=0.8]{figuras/entidaderelacionamentomobile}
\caption{Modelo entidade e relacionamento aplicativo móvel.}
\label{fig:entidaderelacionamentomobile}
\end{figure}

O aplicativo contará com três telas distintas que serão a tela inicial, a tela do aluno e a tela do professor. Em cada tela o sistema faz requisição de um serviço diferente para o módulo API. Na tela inicial é feito a requisição GET a Listagem \ref{lst:bd}, na tela do aluno é feito a requisição GET a Listagem \ref{lst:todos} e na tela do professor é feito um POST a Listagem \ref{lst:incluirmensagem},

\begin{lstlisting}[caption={Incluir novas mensagens},label={lst:incluirmensagem}]
	http://{IP DO SERVIDOR}/mensagens/
\end{lstlisting}

A tela inicial será apresentada como na Figura \ref{fig:mobile1}, exibindo o nome do SID e o botão de login na parte superior e as informações das publicações válidas, que foram requisitadas do módulo API, no restante da tela.
\begin{figure}[H]
\centering
\includegraphics[scale=0.5]{figuras/mobile1}
\caption{Página de inserção no módulo administrador.}
\label{fig:mobile1}
\end{figure}

A funcionalidade de login é indispensável para que alunos e professores efetuem a autenticação e possam acessar as páginas restritas destinadas a cada um.

A tela do aluno será apresentada como na Figura \ref{fig:mobile2}, após o login, será apresentado ao aluno todas as mensagens que o aluno possui em ordem cronológica da mais nova para a mais antiga. Os campos apresentados são os de professor, turma e mensagem, separando cada item por blocos.

\begin{figure}[H]
\centering
\includegraphics[scale=0.6]{figuras/mobile3}
\caption{Página do Aluno.}
\label{fig:mobile2}
\end{figure}

Já a tela do professor será apresentada conforme a Figura \ref{fig:mobile3}, após o login, será apresentado ao professor a matricula dele, a lista de turmas que ele leciona e uma caixa de texto que o professor usará para digitar a mensagem.

\begin{figure}[H]
\centering
\includegraphics[scale=0.5]{figuras/mobile2}
\caption{Página do professor.}
\label{fig:mobile3}
\end{figure}

Todos os campos são de preenchimento obrigatório e caso algum erro ocorra no processo de envio, o sistema apresentará um notificação para o professor com o erro ocorrido.

\section{Integração}
A integração entre o sistema e o Facebook se dá com o uso da Graph API. Ferramenta disponibilizada pela rede social para que feita a integração de sistemas externos a rede social Facebook.

Com o uso dela, pode-se recuperar os mais diversos dados de acordo com as permissões que são concedidas.

\subsection{\textit{login}, \textit{token} de acesso e permissões}
Para autenticação e efetivação de todas as requisições feitas pelo módulo administrador para o Facebook é necessário que um \textit{login} seja feito. Para isso, é utilizado a ferramenta de ``\textit{login} do Facebook''  que é disponibilizada pela rede social.

A ferramenta de ``\textit{login} do Facebook'' que é disponibilizada pelo mesmo, oferece um botão que pode ser colocado na página que possui a finalidade de iniciar o processo de login. Após ser clicado é efetuado o processo de login conforme segue padrão estabelecido, solicitando as credenciais cadastradas na rede social, com isso é feito a requisição com as permissões necessárias conforme Listagem \ref{lst:solicitacaologin}.

No processo de \textit{login} são solicitados as permissões necessárias para que o sistema funcione, essas permissões são a “email”, “manage\underline{{ }}pages” e “publish\underline{{ }}pages”, sendo obrigatório o aceite para acesso ao sistema. Usadas respectivamente para que seja possível a identificação do usuário do SID, para recuperar as permissões de acesso a página e a última para permitir que aplicativos publiquem na página \cite{facebook2018a}.

Após a efetivação do login o sistema realiza a requisição da Listagem \ref{lst:permissoesaceitas} para verificar as permissões que o usuário concedeu, onde o retorno será um JSON contendo todas as permissões concedidas.

\begin{lstlisting}[caption={Permissões concedidas},label={lst:permissoesaceitas}]
  	$response = $fb->get(
    	'/me/permissions',
		'{access-token}'
	);
	
	$graphNode = $response->getGraphNode();
\end{lstlisting}

Então o sistema compara as permissões aceitas com as permissões necessárias e caso alguma não esteja presente, o acesso é negado, retornando para a página de login e apresentado a mensagem de erro para o usuário.

A existência de um usuário vinculado a rede social é irrefutável, pois existe a necessidade da página apresentar qual o perfil está realizando a ação, além da necessidade de moderação dos comentários que serão exibidos.  Além disso, para efetivação de todas as requisições feitas pelo módulo administrador para o Facebook, seja ela para requisitar informações das publicações ou realizar uma nova publicação, é usando o \textit{token} de acesso de usuário, que é obtido no processo de login, conforme Listagem \ref{lst:tokenUsuario}.

Entretanto, para o acesso é indispensável também que o um usuário esteja cadastrado no banco de dados do SID. Essa ação é necessária para que somente usuários autorizados possam realizar alteração na página usando o sistema.

Após o login, algumas informações usadas são guardadas na sessão do usuário, para usos posteriores. Porém, todo o processo de armazenamento de informações, envios ao Facebook e retornos de solicitações não devem ser visíveis ao usuário. 

\subsection{Publicar}
Todas as publicações que são realizadas no Facebook usando o SID são feitas ao nó ``/photo'', onde a cada nova publicação será gerado um ID para ela e terá a mesma estrutura de uma imagem postada por um usuário convencional desta rede social.

Na criação de uma nova publicação, usando o SID, são passados para a Graph API dois parâmetros para inserção através do método POST, conforme Listagem \ref{lst:requisicao9}. O primeiro deles é “message”, onde será passado o texto que será exibido na publicação e o outro é “source”, onde será passado a imagem para ser exibida juntamente com o texto. Para envio de imagem para a rede social, é necessário passar na imagem como parâmetro o método “fileToUpload” conforme Listagem \ref{lst:appesdk}. 

Alguns dos elementos que são solicitados pela aplicação na criação de uma nova publicação são omitidos no envio para o Facebook, pois esses dados serão usados apenas para serem armazenados no banco para uso do módulo cliente. Os elementos omitidos são os campos data de início, a data de termino e a legenda. 

A Figura \ref{fig:imgfacebook1} apresenta como fica uma publicação no Facebook após o uso do SID para criação da mesma, enquanto a Figura \ref{fig:cliente1} apresenta como será exibido no módulo cliente. Nela são apresentadas informações como quem realizou a publicação, o texto e a imagem que foram informados durante a criação da divulgação pelo SID.

\begin{figure}[H]
\centering
\includegraphics[scale=0.8]{figuras/imgfacebook1}
\caption{Divulgação enviada ao Facebook com auxílio do SID}
\label{fig:imgfacebook1}
\end{figure}


\subsection{Requisição}
Usando o ID da foto é possível recuperar informações das arestas, que podem ser a URL, os comentários e as curtidas conforme Listagem \ref{lst:requisicao8}. 

Esse recurso é utilizado para recuperar os mais diversos elementos que as permissões concedidas autorizam, entre os elementos recuperados estão o endereço da publicação, os comentários, curtidas e a imagem de perfil que serão apresentados respectivamente nos campos destinados ao QR Code e na coluna de comentários do módulo cliente.

A URL da publicação que é criada, é armazenada no banco de dados no momento da criação da mesma pelo modulo administrador. A requisição segue a Listagem \ref{lst:requisitarurl}.

\begin{lstlisting}[caption={Foto de usuário},label={lst:requisitarurl}]
  	$response = $fb->get(
    	'/{ID PUBLICACAO}?fields=link',
		'{access-token}'
	);
	
	$graphNode = $response->getGraphNode();
\end{lstlisting}

A recuperação dos outros dados necessários é feito pelo módulo API, onde ele realiza requisições a Graph. As requisições feitas pelo módulo são para recuperação de comentários da Listagem \ref{lst:comentariosPostagem}, de curtidas desses comentários, conforme Listagem \ref{lst:curtidasComentario} e foto de perfil do usuário que realizou o comentário conforme Listagem \ref{lst:fotousuario}.

\begin{lstlisting}[caption={Foto de usuário},label={lst:fotousuario}]
  	$response = $fb->get(
    	'/{ID USUARIO}?fields=picture',
		'{access-token}'
	);
	
	$graphNode = $response->getGraphNode();
\end{lstlisting}

A recuperação dos comentários e da foto do perfil do usuário que publicou cada comentário é crucial para que essa informação seja exibida no módulo cliente, no elemento 5, conforme a Figura \ref{fig:cliente1}.

A recuperação de curtidas é necessária para que possa ser feito a moderação dos comentários a serem exibidos. A moderação é feita de forma a verificar o ID de todos os usuários que curtiram cada comentário a fim de verificar se algum dos IDs encontrados é o ID do administrador que está cadastrado no banco.

Os comentários que possuem a curtida do administrador é registrado para exibição no elemento 5 da Figura \ref{fig:cliente1}.

\section{Possível solução para Implantação - Raspberry}
Para viabilizar uma implementação barata ao \textit{Campus} Taguatinga, computadores Raspberry Pi podem ser utilizados.

Teste do \cite{aristotelous2016} apresenta a possibilidade de se ter um servidor completamente funcional com sistema operacional Linux por um equipamento de 35\$, possibilitando a criação de um servidor, por exemplo de um repositório na nuvem com um baixo custo, flexibilidade e eficiência energética. 

Para \cite{cusick2014}, placas de circuito oferecem vantagens como: Uso de pouco espaço; Desempenho significante com baixo custo e consumo; Suporte a diversas soluções de software, oferecendo múltiplas opções de interface com uma variante do Linux. 

Portanto, este tipo de dispositivo provê diversos benefícios que podem ser de grande importância na escolha de um sistema para implementação do módulo cliente do SID.