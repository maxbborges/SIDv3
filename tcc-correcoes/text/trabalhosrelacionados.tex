\chapter[Trabalhos Relacionados]{Trabalhos Relacionados}
\label{relacionados}
Este capítulo tem como propósito apresentar ao leitor algumas ferramentas que fazem uso dos conceitos de sinalização e \textit{marketing} digital, exibindo em telas externas, divulgações criadas pelo administrador no servidor. Para cada ferramenta, serão apresentadas as suas funcionalidades e suas críticas, expondo suas limitações.

Todas elas seguem a ideia de uma arquitetura cliente-servidor, onde o servidor é responsável por gerenciar os conteúdos que serão exibidos em telas que seguem os conceitos de sinalização digital e possuem o cliente instalado.

\section{OOZO}
\label{sec:oozo}
Atualmente, uma solução encontrada na literatura que faz integração entre \textit{marketing} digital e o conceito de sinalização digital é o software \cite{oozo2017}, onde o módulo cliente é multiplataformas e pode ser executado em qualquer sistema que possua navegador de página \textit{web} e instalado em sistemas operacionais Linux, Windows ou macOS para acesso local.

O módulo administrador usado para criação, edição e gerenciamento fica localizado na página oficial do aplicativo, necessitando de login para acesso. Para acessá-lo é possível criar uma nova conta, usar uma conta do Facebook ou Google. 

Em sua versão profissional, a mais completa, pode-se sincronizar as redes sociais Twitter, Instagram e Facebook, sendo necessário efetuar login em cada rede social separadamente. Já em sua versão gratuita, alguns recursos não estão disponíveis como o filtro de \textit{hashtag} e a opção de publicar imagens e vídeos, além de exibir nas telas públicas, propagandas de outras empresas juntamente com as suas publicações. 

Após a sincronização, o sistema recupera automaticamente \textit{hashtag} do Twitter que foram definidas pelo administrador e postagens do Facebook, Instagram e Twitter, sendo possível limitar o número de itens recuperados. Entretanto, há possibilidade de apenas deletar a publicação do sistema, não sendo possível realizar modificações nas postagens recuperadas, não oferece suporte a enviar uma nova publicação para as redes sociais e não se pode recuperar comentários ou curtidas das publicações.

Todo o conteúdo publicado na ferramenta fica armazenado em servidores proprietários, onde o módulo administrador é acessado usando uma página \textit{web}, necessitando de acesso para a manutenção e gerenciamento do sistema. 

Na página inicial do módulo administrador se pode gerenciar suas publicações que já foram postadas. Selecionando uma delas, é possível ter o acesso direto a publicação ou excluí-la da ferramenta. Entretanto, ao excluir do sistema, a publicação não é excluída da rede social, além disso, na versão gratuita parte do tempo de exibição das informações é de outras propagandas que não foram criadas pelo usuário. 

No módulo cliente, o OOZO não exibe os comentários feitos na publicação da rede social, a ferramenta somente captura de forma automatica a foto, o texto publicado e o endereço da publicação e então os envia para exibição. O acesso a notícia completa pelo usuário, é feita com a leitura de um \textit{QR code}, que contem o endereço que foi recuperado da publicação no Facebook.

\section{MangoSigns}
\label{sec:mango}
A segunda solução encontrada é o \citet{mango2017}, o módulo cliente é multiplataformas e pode ser instalado localmente no navegador Chrome, sistema operacional Windows ou dispositivos móveis com sistema Android.

A versão paga possui integração com as principais redes sociais como Twitter, Instagram e Facebook, informações do tempo, entre outras funcionalidades. A versão gratuita não possui integração as redes sociais, oferecendo somente informações básicas como data, envio de imagens e estilos de fonte que serão apresentados no cliente, além de possuir limite de 3 imagens por publicação que será exibida. Além de no módulo cliente não possuir um \textit{QR code} para que o telespectador possa acessar a notícia completa.

A página de administração do aplicativo é acessível com o uso de navegadores, é possível criar uma nova conta, usar uma conta do Facebook ou uma conta do Google. A interface do módulo administrador é usado para criação, edição e gerenciamento de publicações, entretanto, é necessário um dispositivo Android, navegador Chrome ou um dispositivo próprio chamado MangoSing Box conectado a uma TV ou monitor para exibição do conteúdo criado.

As publicações criadas são imagens fixas com blocos adicionados manualmente que sobrepõem as imagens. Elas são exibidas e trocadas automaticamente de acordo com o tempo configurado, sendo possível determinar tempo de transição e até quando a publicação será exibida, além da localização (exclusivo da versão paga). 

A integração com as redes sociais é somente para recuperação automática de publicações feitas, não sendo possível recuperar informações como comentários e curtidas feitas na publicação. Em nenhuma de suas versões há a possibilidade de enviar a divulgação criada para as redes sociais a partir da ferramenta. A página do MangoSigns oferece diversas interfaces prontas para criação dos imagens personalizadas, podendo também ser criada uma nova. 

\section{SID Formosa}
\label{sec:sid}
O SID versão 2 (Sistema Inteligente de Divulgação de Informações do IFG-Formosa), é outra solução, dispondo de dois módulos, sendo o módulo administrador, usado para criação, editação e gerenciamento das publicações e o módulo cliente, usado para exibição das notícias criadas.

O acesso ao módulo administrador se dá quando utilizada a conta do Facebook, que esteja cadastrada no banco de dados. Apesar de possuir integração com as redes sociais é limitada a criação de novas publicações no perfil do usuário do sistema, não é possível recuperar nenhuma informação de publicação, comentário ou curtida.

O SID, no módulo cliente, apresenta as informações que lhe são configuradas, recuperando informações previamente armazenadas no banco de dados. As publicações apresentam na tela somente a imagem, um \textit{QR Code} usado pelo usuário caso tenha interesse em acessar a notícia completa e a legenda que foi configurada. \citet{sobrinho2017}

Entre os problemas encontrados estão: o link que será apresentado no QRCode deve ser configurado manualmente na criação da publicação, não oferece suporte para configurar data de início da apresentação das publicações, falta de uma documentação detalhada, integração bastante limitada, além de não oferecer nenhum tipo de aplicação móvel.

\section{Screenly}
O \citet{screenly2017} usa o Raspberry Pi no módulo cliente, necessitando de um programa próprio que deve ser instalado no equipamento para seu funcionamento. Para acesso a área restrita, é possível apenas criar uma nova conta ou logar com uma existe, não sendo oferecido nenhum outro meio de automatização de \textit{login}. 

A exibição das publicações é feita por meio de um Raspberry Pi com o uso de um \textit{software} proprietário da Screenly instalado e uma televisão conectada a ele. Todas as versões possuem limitação do número de equipamentos que podem exibir o conteúdo.

Em nenhuma das versões o Screenly possui integração com as redes sociais. Em sua versão gratuita, ele é limitado a criar no máximo duas publicações para serem exibidas de forma sequencial, essas publicações podem ser imagens, links ou vídeos.

\section{Xibo}
Outra solução encontrada é o \textit{software} \citet{xibo2017}, que se trata de um sistema baseado em arquitetura cliente-servidor que permite diversas customizações, onde cada item que será apresentado fica dentro de um bloco escolhido. Além disso, ele suporta o envio de diferentes tipos de arquivos, tais como como vídeos, áudios e Flash.

Apesar de suportar diferentes formatos, oferecer \textit{templates} prontos e ser gratuito, ele não possui integração com as redes sociais, não sendo possível recuperar ou enviar para o Facebook qualquer dado das divulgações criadas por ele. 

\section{Comparativo}
A Tabela 1 faz um comparativo entre os sistemas citados acima, comparando algumas das funcionalidades consideradas importantes para sistemas que trabalham com a implantação de sinalização digital e \textit{maketing} digital, na qual seus elementos comparativos são descritos a seguir:
\begin{enumerate}[label=\Roman*)]
	\item Comprometimento com o propósito: o sistema em questão possibilita a veiculação de informações através de mecanismos de sinalização digital?
	\item Criação simples de divulgações: o operador possui facilidade de incluir novas divulgações com aspecto atrativo?
	\item Portabilidade móvel: o módulo cliente está disponível para a plataforma móvel em forma de aplicativo?
	\item Integração com redes sociais: o sistema realiza integração completa com as redes sociais, oferecendo recuperação e envio de dados.
	\item A versão gratuita explora toda a capacidade do programa?
	\item É possível criar uma nova publicação e enviá-la para as redes sociais?
\end{enumerate}

\begin{table}[h!]
	\caption{Comparativo}
	\label {tlb:comparativo1}
	\centering
	\begin{tabular}{|c|c|c|c|c|c|}
		\hline
		Quesito/Sistema & OOZO & MangoSigns & SID v2 & Screenly & XIBO \\ \hline
		I 				& SIM  & SIM		& SIM & SIM 	 & SIM	\\ \hline
		II 				& SIM  & SIM 		& SIM & SIM 	 & SIM	\\ \hline
		III				& NÃO  & SIM 		& NÃO & NÃO 	 & NÃO	\\ \hline
		IV 				& NÃO  & NÃO 		& NÃO & NÃO 	 & NÃO	\\ \hline
		V 				& NÃO  & NÃO 		& SIM & NÃO 	 & SIM	\\ \hline
		VI 				& NÃO  & NÃO 		& SIM & NÃO 	 & NÃO	\\ \hline
	\end{tabular}
\end{table}

Com o comparativo pode se notar que nenhum dos sistemas descritos atendem as funcionalidades requeridas. Por isso, o sistema SIDv3 tem o objetivo de atender todas as funcionalidades descritas como necessárias na tabela de comparativo \ref{tlb:comparativo1}. Oferecendo um aplicativo móvel para exibição das publicações, melhor interação acadêmica entre professores e alunos, e uma integração completa com as redes sociais, possibilitando o envio e recuperação de dados das páginas/comunidades do Facebook.