\chapter[Trabalhos Relacionados]{Trabalhos Relacionados}
\label{relacionados}
Este capítulo tem como propósito apresentar ao leitor algumas ferramentas que fazem uso dos conceitos de sinalização e \textit{marketing} digital, exibindo em telas externas, divulgações criadas pelo administrador no servidor. Para cada ferramenta, serão apresentadas as suas funcionalidades e críticas, expondo suas limitações.

Todas elas seguem a ideia de uma arquitetura cliente-servidor, onde o servidor é responsável por gerenciar os conteúdos que serão exibidos em telas que seguem os conceitos de sinalização digital e possuem o cliente instalado.

\section{OOZO}
\label{sec:oozo}
Uma solução encontrada na literatura que faz integração entre \textit{marketing} digital e o conceito de sinalização digital, é o software \cite{oozo2017}, onde o módulo cliente é multiplataformas e pode ser executado em qualquer sistema que possua navegador de página \textit{Web} e instalado em sistemas operacionais Linux, Windows ou macOS para acesso local.

O módulo administrador usado para criação, edição e gerenciamento fica localizado na página oficial do aplicativo, necessitando de login para acesso. Para acessá-lo é possível criar uma nova conta, usar uma conta do Facebook ou Google. 

Em sua versão profissional, a mais completa, pode-se sincronizar as redes sociais Twitter, Instagram e Facebook, sendo necessário efetuar login em cada rede social separadamente. Já em sua versão gratuita, alguns recursos não estão disponíveis como o filtro de \textit{hashtag} e a opção de publicar imagens e vídeos. Juntamente com o conteúdo criado são exibidas propagandas de outras empresas, devendo possuir uma assinatura paga para que não sejam mostradas.

Após a sincronização, o sistema recupera automaticamente \textit{hashtag} do Twitter que foram definidas pelo administrador e postagens do Facebook, Instagram e Twitter, sendo possível limitar o número de itens recuperados. Entretanto, há possibilidade de apenas deletar a publicação do sistema, não sendo possível realizar modificações nas postagens recuperadas, não oferece suporte para enviar uma nova publicação para as redes sociais e não se pode recuperar comentários ou curtidas das publicações.

Todo o conteúdo publicado na ferramenta fica armazenado em servidores proprietários. No gerenciamento de publicações do módulo administrador é possível ter o acesso direto à publicação ou excluí-la da ferramenta.

No módulo cliente, o OOZO não exibe os comentários feitos na publicação da rede social, a ferramenta somente captura de forma automatica a foto, o texto publicado e o endereço da publicação e então os envia para exibição. O acesso à notícia completa pelo usuário é feita com a leitura de um \textit{QR code}, que contém o endereço que foi recuperado da publicação no Facebook.

\section{MangoSigns}
\label{sec:mango}
A segunda solução encontrada é o \citet{mango2017}, em que o módulo cliente é multiplataforma e pode ser instalado localmente no navegador Chrome, sistema operacional Windows ou dispositivos móveis com sistema Android.

A versão paga possui integração com as principais redes sociais como Twitter, Instagram e Facebook, informações climáticas, entre outras funcionalidades. A versão gratuita não possui integração às redes sociais, oferecendo somente informações básicas como data, envio de imagens e estilos de fonte que serão apresentados no cliente, além de possuir limite de 3 imagens por publicação que será exibida. Além do módulo cliente não possuir um \textit{QR code} para que o telespectador possa acessar a notícia completa.

A página de administração do aplicativo é acessível através de navegadores, é possível criar uma nova conta, usar uma conta do Facebook ou uma conta do Google. A interface do módulo administrador é usado para criação, edição e gerenciamento de publicações, entretanto, é necessário um dispositivo Android, navegador Chrome ou um dispositivo próprio chamado MangoSign Box conectado a uma TV ou monitor para exibição do conteúdo criado.

As publicações criadas são imagens fixas com blocos adicionados manualmente que sobrepõem as imagens. Elas são exibidas e trocadas automaticamente de acordo com o tempo configurado, sendo possível determinar tempo de transição e até quando a publicação será exibida.

A integração com as redes sociais é somente para recuperação automática de publicações feitas, não sendo possível recuperar informações como comentários e curtidas feitas na publicação. Em nenhuma de suas versões há a possibilidade de enviar a divulgação criada para as redes sociais a partir da ferramenta. A página do MangoSigns oferece diversas interfaces prontas para criação de imagens personalizadas, podendo também ser criada uma nova interface. 

\section{SID versão 2}
\label{sec:sid}

O SID versão 2 (Sistema Inteligente de Divulgação de Informações do IFG-Formosa), é outra solução, dispondo de dois módulos, sendo o módulo administrador, usado para criação, editação e gerenciamento das publicações e o módulo cliente, usado para exibição das notícias criadas. O acesso ao módulo administrador se dá quando utilizada a conta do Facebook associada ao e-mail cadastrado no banco de dados do SID.

Apesar de possuir integração, com a rede social Facebook, ela é limitada. As publicação criadas pelo sistema são vinculadas a um perfil de usuário da rede social, não sendo possível vincular páginas oficiais. Além disso, não esta disponível a funcionalidade de recuperar informações das publicações criadas, tais como comentários ou curtidas.

O SID, no módulo cliente, apresenta as informações que lhe são configuradas, recuperando informações previamente armazenadas no banco de dados. As publicações apresentam na tela somente a imagem, um \textit{QR Code} usado pelo usuário caso tenha interesse em acessar a notícia completa e a legenda que foi configurada. \citet{sobrinho2017}

Entre os problemas encontrados estão: o link que será apresentado no QRCode, deve ser configurado manualmente na criação da publicação, não oferece suporte para configurar data de início da apresentação das publicações, falta de uma documentação detalhada, integração bastante limitada, além de não oferecer nenhum tipo de aplicação móvel.

\section{Screenly}
O \citet{screenly2017}, pra uso do módulo cliente é necessário um programa proprietário que esta disponível somente para computadores Raspberry Pi. O módulo administrador é uma página \textit{Web} que para acesso é possível apenas criar uma nova conta ou logar com uma criada anteriormente, não sendo oferecido nenhum outro meio de automatização de \textit{login} como, por exemplo, usar uma rede social para se autenticar. 

O computador deve ser conectado a uma televisão para exibição das publicações criadas. Todas as versões do sistema possuem limitação do número de equipamentos que podem exibir o conteúdo.

Além de não possuir integração com as redes sociais, o Screenly exibe somente imagens, não possibilitando a inserção de texto ou endereço para acesso a publicação.

\section{Xibo}
Outra solução encontrada é o \textit{software} \citet{xibo2017}, ele permite diversas customizações. No painel de administrador é disponibilizado diversos blocos, cada item que será apresentado fica dentro de um bloco escolhido. Ele suporta o envio de diferentes tipos de arquivos, tais como: vídeos, áudios e Flash.

Apesar de suportar diferentes formatos, oferecer modelos prontos e ser gratuito, ele não possui integração com as redes sociais, não sendo possível recuperar ou enviar para o Facebook qualquer dado das divulgações criadas por ele. 

\section{Comparativo}
A Tabela 1 faz um comparativo entre os sistemas testados, comparando algumas das funcionalidades que os sistemas que funcionam com o conceito de sinalização digital e \textit{marketing} digital disponibilizam, na qual os elementos comparativos são descritos a seguir:
\begin{enumerate}[label=\Roman*)]
	\item O sistema em questão possibilita a veiculação de informações através de mecanismos de sinalização digital? Para \cite{machado2010}, estes sistemas apresentam diversas vantagens como a possibilidade de divulgações mais abrangentes e consistentes.
	\item O operador consegue criar novas publicações pelo sistema possibilitando a centralização na criação e gerenciamento do conteúdo? 
	\item O módulo cliente está disponível para a plataforma móvel em forma de aplicativo? As pesquisas da \cite{emarketer} abordam que é crescente a utilização de celulares, então, um aplicativo para dispositivos móveis que exibe os mesmos conteúdos que são apresentados na rede social e nos televisores, disseminaria mais as informações.
	\item Integração com as redes sociais: o sistema possibilita recuperação e envio de dados para alguma rede social? As pesquisas de \cite{seo2017} apontam como as redes sociais influenciam as pessoas e para \cite{rosa2010}, é crescente o uso delas no cotidiano das pessoas. Para \cite{escobar2007} elas colocam a interatividade em evidência, por possibilitar que o público participe com comentários, compartilhamentos e curtidas. Conteúdos estes que podem ser recuperados e apresentados como forma de interação.
	\item A versão gratuita explora toda a capacidade do programa?
	\item É possível criar uma nova publicação e enviá-la para as redes sociais? Para \cite{teles2013} a união de dispositivos móveis e redes sociais deram origem as Redes Sociais Móveis (RSM), na qual permite que os usuários façam tudo aquilo que antes era feito nas redes sociais por computadores, em dispositivos móveis. Isso corroborou para o aumento do uso de redes sociais no cotidiano e para disseminação de notícias, como afirmam pesquisas da \cite{emarketer}. As pesquisas da \cite{emarketer2013} afirmam que o investimento em \textit{marketing} nessas plataformas são cada vez maiores.
\end{enumerate}

\begin{table}[H]
	\caption{Comparativo}
	\label{tlb:comparativo1}
	\centering
	\begin{tabular}{|c|c|c|c|c|c|}
		\hline
		Quesito/Sistema & OOZO & MangoSigns & SIDv2 & Screenly & XIBO \\ \hline
		I 				& SIM  & SIM		& SIM & SIM 	 & SIM	\\ \hline
		II 				& SIM  & SIM 		& SIM & SIM 	 & SIM	\\ \hline
		III				& NÃO  & SIM 		& NÃO & NÃO 	 & NÃO	\\ \hline
		IV 				& NÃO  & NÃO 		& NÃO & NÃO 	 & NÃO	\\ \hline
		V 				& NÃO  & NÃO 		& SIM & NÃO 	 & SIM	\\ \hline
		VI 				& NÃO  & NÃO 		& SIM & NÃO 	 & NÃO	\\ \hline
	\end{tabular}
\end{table}

Com o comparativo pode se notar que nenhum dos sistemas descritos atendem as funcionalidades requeridas. Por isso, o sistema SIDv3 tem o objetivo de atender todas as funcionalidades descritas como necessárias na Tabela \ref{tlb:comparativo1}. Oferecendo um aplicativo móvel para exibição das publicações, melhor interação acadêmica entre professores e alunos, e uma integração completa com as redes sociais, possibilitando o envio e recuperação de dados das páginas/comunidades do Facebook.