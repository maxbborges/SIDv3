\chapter[Resultados]{Resultados}
\label{resultados}
A ideia anterior do SIDv2, desenvolvido por \cite{sobrinho2017}, era a de um sistema inteligente de divulgação. Entretanto, no presente trabalho a ideia foi alterada para um sistema integrado de divulgação. Isso se deu pelo fato do sistema não possuir nenhum tipo de inteligência própria e sim realizar a integração entre redes sociais e usuário.

Em sua versão anterior, o sistema possuía integração bastante limitada, nenhum tipo de aplicativo móvel para melhor interação acadêmica e pouca interatividade do sistema com o usuário e vice-versa. Então, no presente trabalho foi melhorado a integração com o Facebook e as formas de interação com o usuário, além da criação de um protótipo para dispositivos móveis.

O detalhamento das funcionalidades e apresentação do sistema foi exposto no Capítulo \ref{cap:sid}. O código para implantação encontra pronto e disponível para acesso no GitHub através do endereço \url{https://github.com/maxbborges/SIDv3}.
 
Por falta de uma documentação anterior, foi necessária a criação de uma extensa documentação do software de modo a possibilitar futuras modificações e melhorias de maneira fácil.

No módulo administrador foi possível alcançar todos os objetivos desejados, ficando responsável por todo o processamento e gerenciamento das informações no banco de dados. A partir dele é possível inserir, listar, deletar e editar as informações que servirão para que o módulo API forneça dados conforme a estrutura de uma arquitetura REST, permitindo que outras aplicações futuramente venham a consumir esta API.

Além disso, foi implementado uma verificação de segurança no processo de login que não existia no SIDv2, onde qualquer usuário cadastrado poderia acessar a aplicação. Entretanto, algumas funcionalidades oferecidas pelo módulo só estão disponíveis se forem concedidas algumas permissões específicas. Portanto, o login no módulo só é realizado caso todas as permissões solicitadas sejam aceitas.

O submódulo API, parte do módulo administrador, é estruturado de acordo com a arquitetura REST, com a capacidade de realizar um CRUD nas publicações e nas mensagens entre professor aluno.

Este submódulo foi implementado com duas funcionalidades distintas. A primeira é obter os dados armazenados no banco e realizar as chamadas a Graph para obter os dados dos comentários das publicações que serão exibidas e organizar esses dados para que sejam enviados. A segunda é servir de aparato ao aplicativo, para que ele possa simular o consumo de uma API externa. 

O módulo cliente do SID ficou estruturado e pronto para ser implantado e executado em diversos dispositivos, bastando o mesmo possuir um navegador de páginas \textit{Web} e conexão constante à \textit{Internet} para que sejam realizadas as consultas HTTP.

No protótipo do aplicativo móvel foi implementado a funcionalidade de consumo de uma REST API fictícia, simulando o Sistema de Gestão Acadêmica do IFB, para exibição das divulgações e um sistema de comunicação interna entre professores e alunos, onde é necessário um login criado de maneira fictícia para simulação de um sistema acadêmico, necessitando apenas da liberação dos dados do SGA para autenticação do sistema e a adaptação das consultas feitas a esta API.

Na busca de soluções que seguem a ideia de criação de publicação para apresentação em dispositivos externos, foram encontradas diversas ferramentas com essa finalidade, todas elas usando o conceito de arquitetura cliente-servidor. As que apresentaram melhores funcionalidades foram: a OOZO, a MangoSigns, o SID versão 2, a Screenly e a XIBO.

O OOZO, apresentado na Seção \ref{sec:oozo}, é um sistema pago, porém possui uma versão gratuita com acesso limitado aos recursos oferecidos. Ele oferece suporte a multiplataformas, controle \textit{Web}, integração à redes sociais e acesso à publicações via QRCode. Apresentando duas limitações, uma é no que tange a integração com as redes sociais, onde é possível apenas recuperar e deletar publicações, a outra limitação está no acesso às divulgações por dispositivo móvel, onde é necessário o uso de navegadores \textit{Web}. 

A MangoSigns, apresentada na Seção \ref{sec:mango}, é um sistema pago que possui uma versão gratuita com acesso limitado. A ferramenta possui acesso multiplataformas, inclusive móvel, diversos temas para apresentação e integração com as redes sociais. Possuindo a limitação da integração está disponível apenas na versão paga, além de ser possível apenas recuperar as publicações.

O SID versão 2, apresentado na Seção \ref{sec:sid}, é um sistema gratuito, que possui acesso somente \textit{Web}, acesso à publicação via QRCode e integração com a rede social Facebook. Nele é possível apenas realizar publicação da página de perfil do usuário, não sendo possível recuperar nenhum dado.

O Screenly, além de usar um aplicativo proprietário, necessitando de um \textit{raspberry}, ele não possui aplicativo móvel e nenhuma forma de integração com as redes sociais.

Já o Xibo, apesar de ser uma ferramenta gratuita, não possui aplicativo móvel e nenhuma integração com as redes sociais. 

Comparando cada uma das ferramentas encontradas com o SIDv3, é possível notar diversas funcionalidades que não estão presentes, são limitadas ou não são gratuitas, portanto diversos pontos que são considerados como essenciais e foram implementados no SIDv3, as ferramentas apresentadas não suportam.

Entre esses pontos, está a visualização das publicações por meio de um aplicativo móvel, onde com exceção do MangoSigns, nenhum dos sistemas oferece um aplicativo para \textit{smartphones}, necessitando de um dispositivo que possuísse um navegador \textit{Web} para acesso.

Outro ponto é o de integração com as redes sociais, nenhuma das solução analisadas realizavam uma integração tão completa quanto a do SIDv3, como a possibilidade de criação de novas publicações, recuperação de comentários, curtidas e fotos de perfil, além do filtro de comentários a serem exibidos. 

A Tabela \ref{tlb:comparacaoresultados} tem o intuito de fazer um comparativo entre os sistemas apresentados, agora com inclusão do SID em sua terceira versão. Na qual seus elementos comparativos são descritos a seguir:
\section{Comparativo}
\begin{enumerate}[label=\Roman*)]
\label{tlb:comparacaoresultados}
    \item O sistema em questão possibilita a veiculação de informações através de mecanismos de sinalização digital?
	\item O operador consegue criar novas publicações pelo sistema possibilitando a centralização na criação e gerenciamento do conteúdo? 
	\item O módulo cliente está disponível para a plataforma móvel em forma de aplicativo?
	\item Integração com as redes sociais: o sistema possibilita recuperação e envio de dados para alguma rede social?
	\item A versão gratuita explora toda a capacidade do programa?
	\item É possível criar uma nova publicação e enviá-la para as redes sociais?
\end{enumerate}

\begin{table}[H]
	\caption{Comparativo}
	\label {tlb:comparativo2}
	\centering
	\begin{tabular}{|c|c|c|c|c|c|c|}
		\hline
		Quesito/Sistema & OOZO & MangoSigns & SIDv2 & Screenly & XIBO & SIDv3 \\ \hline
		I 				& SIM  & SIM		& SIM & SIM 	 & SIM	& SIM \\ \hline
		II 				& SIM  & SIM 		& SIM & SIM 	 & SIM		& SIM \\ \hline
		III				& NÃO  & SIM 		& NÃO & NÃO 	 & NÃO		& SIM \\ \hline
		IV 				& NÃO  & NÃO 		& NÃO & NÃO 	 & NÃO		& SIM \\ \hline
		V 				& NÃO  & NÃO 		& SIM & NÃO 	 & SIM		& SIM \\ \hline
		VI 				& NÃO  & NÃO 		& SIM & NÃO 	 & NÃO		& SIM \\ \hline
	\end{tabular}
\end{table}

Portanto, a tabela mostra que o SIDv3 atende a todos os requisitos desejados para o sistema.

\section{Melhorias em relação ao SIDv2}
Em relação ao SIDv2, diversas melhorias foram realizadas. O detalhamento destas melhorias é exposto nesta seção.

\subsection{Módulo Administrador}
    \begin{itemize}
        \item Múltiplos administradores: No SIDv2, as informações dos administradores eram armazenadas em um arquivo texto no código fonte do sistema, no SIDv3 os dados são armazenados em um banco de dados, onde é possível registrar inúmeras contas, diferentemente do arquivo texto, onde só se recuperava um único usuário.

        \item Mensagens de erro ao realizar o registro: Na segunda versão, quando tentava-se uma autenticação com um usuário ou senha que não estava registrado no arquivo texto, o sistema não retornava nenhuma informação de erro, apenas retornava para a tela de \textit{login}. Na nova versão, o sistema apresenta na tela de \textit{login} qual o problema foi encontrado pelo sistema para a não autenticação.

        \item Restrição de acesso: Na versão anterior, bastava o administrador possuir os seus dados cadastrados que a autenticação no sistema era efetivada. Na nova versão o sistema realiza a verificação das permissões do Facebook que são necessárias para o funcionamento do sistema e se essas foram concedidas pelo usuário no momento da autenticação.

        \item Recuperação automática de endereços: No SIDv2, na página de inserção, o SID usava o campo de texto ``Link para mais informações'', entretanto no SIDv3 o campo foi retirado para que o endereço fosse obtido de forma automática pelo sistema, recuperando o endereço da publicação no Facebook no momento da criação da notícia.
        
        \item Retirada da caixa ``Prioridade'': O \textit{combobox} ``Prioridade'' foi retirado pois não estava implementado em nenhuma versão.
        
        \item Separação de legenda e texto: Na versão anterior, o SID possuía apenas uma caixa de texto chamada ``legenda'', o conteúdo dessa caixa era apresentado no cliente e enviado ao Facebook, como o cliente apresenta o texto de linear da direita para a esquerda acabava influenciando na leitura caso o texto fosse grande. Na nova versão está disponível o campo ``texto'' e o campo ``legenda'', onde o primeiro será enviado junto a notícia para o Facebook e o segundo será o texto apresentado no cliente.

        \item Data de início: Na v2, na página de inserção, o administrador somente tinha a funcionalidade de definir a data em que a publicação deixaria de ser exibida no módulo cliente. Na v3, foi inserido a nova funcionalidade de se definir também a data em que a publicação começará a ser exibida, não exibindo-a caso não esteja com data válida.
        
        \item Acesso direto a publicação: Na tela de detalhes, na versão 2 do SID, era mostrado os campos de \textit{link} e prioridade. Esses campos foram substituídos para que fosse colocado o endereço para acesso direto a postagem no Facebook.
        
        \item Exclusão: Na segunda versão, o SID somente excluía os dados localmente do banco de dados e não era apresentado nenhuma mensagem para o usuário. Em sua terceira versão, foi implementado a funcionalidade de excluir não somente localmente, mas também a publicação do Facebook, e para que a ação seja efetivada é necessário o administrador confirmar em uma janela \textit{popup} que é apresentada. 
    \end{itemize}
\subsection{Módulo Cliente}
    \begin{itemize}
        \item Atualização automática: Em sua segunda versão, o módulo cliente realizava a requisição so servidor, que recuperava as informações do banco de dados e as enviava para serem exibidas no cliente, entretanto, essa requisição era realizada uma única vez, necessitando de uma atualização da página todas as vezes que um conteúdo fosse alterado. Na terceira versão, é realizado uma nova requisição de tempos em tempos de forma automática, quando uma nova publicação ou comentário é feito, o sistema automaticamente irá exibi-lo.
        
        \item Exibição de comentários: A v2 do SID, não apresentava nenhuma informação de comentário. Foi implementado todo o sistema de requisição, atualização e apresentação dessas informações.
    \end{itemize}
\subsection{Submódulo API e Aplicativo móvel}
Em sua versão 2, o SID não possuía um submódulo API e nem um aplicativo móvel. Para a versão 3, foi realizada toda a criação da documentação e implementação do módulo e do sistema. As funcionalidades e a descrições de cada um são expostas na Subseção \ref{sub:api} e Seção \ref{sub:mobile}, respectivamente.

\subsection{Banco de dados}
    \begin{itemize}
        \item Armazenamento das imagens publicadas: Na segunda versão, ao se criar uma nova publicação, eram armazenados todas as informações no banco de dados em formato texto. Na nova versão, as imagens que eram armazenadas em formato bytea, agora são armazenadas localmente no servidor.
            
        \item Tabela ``divulgacao'': Os atributos ``prioridade'' e ``imagem'' foram retirados pois não seriam mais usados, no lugar, foram criados os atributos ``datainicio'' e ``object\underline{{ }}id''.
            
        \item Tabela ``config'': A tabela foi criada para armazenar as informações necessárias para realizar requisições ao Facebook. Na versão 2, esses dados eram recuperados de um arquivo texto, não possibilitando a inclusão de novos aplicativos e vincula-los ao sistema.
        
        \item Tabelas para módulo móvel: Foram criadas as tabelas: ``professor'', ``designado'', ``turma'', ``mensagem'', ``aluno'' e ``matriculado''. Esta ação foi necessária para que fosse possível realizar a criação de um sistema fictício que simula parte das funcionalidades do SGA.
    \end{itemize}

\section{Dificuldades encontradas}
Durante o desenvolvimento do sistema foram enfrentados diversos problemas, entre eles estão: a falta da documentação detalhada da versão anterior, problemas com a aprovação de permissões solicitadas para o Facebook, a limitação do \textit{token} de aplicativo e a não disponibilização da API do SGA.

A falta de uma documentação detalhada do SIDv2 acabou influenciando no entendimento de como as classes, métodos e o banco de dados se interagiam um com os outros. Sendo necessário um maior tempo de pesquisa e entendimento do sistema, além de ter sido necessário a criação de uma documentação totalmente nova, não sendo possível o aproveitamento da anterior.

Com exceção da permissão ``email'' e ``\textit{public\underline{{ }}profile}'', todas as outras permissões necessitam passar pelo processo de análise do Facebook, necessitando do envio de um vídeo demostrando o uso da permissão. O processo de análise não foi concluído até a finalização deste documento, portanto, o uso das permissões estão concedidas por tempo limitado.

Os \textit{tokens} de aplicativo apresentam a limitação de não autorizar que sejam recuperados dados considerados privados pela rede social, como as curtidas de um comentário, que são usadas para moderação dos comentários que serão exibidos. Para contornar o problema, foi preciso o uso de um \textit{token} de usuário para realizar as requisições de solicitação de dados.

Um dos objetivos do aplicativo móvel, é o uso de informações de alunos e professores presentes no SGA para efetivação da autenticação do usuário. Entretanto, o IFB não disponibilizou em tempo hábil a base de dados para que fosse possível a sua implementação. O problema foi contornado com o uso de uma base de dados fictícia, para posterior migração.