\chapter[Trabalhos Relacionados]{Trabalhos Relacionados}
\section{OOZO}
Atualmente, uma solução encontrada na literatura que faz integração entre \textit{marketing} digital e o conceito de sinalização digital é o software \cite{oozo2017}, com suporte a multiplataformas como Linux, Windows, MacOs, Raspberry e pagina WEB, suas publicações podem ser exibidas online como \textit{streaming} ou também em telas públicas que são painéis colocada em locais estratégicos de grande acesso público e com programações especificas, designadas pelo desenvolvedor com a assinatura de um serviço de exibição.

Na pagina de login do aplicativo é possível criar uma nova conta, usar uma conta do Facebook ou Google. Na sua versão profissional, a mais completa, é possível sincronizar redes sociais, capturar comentários com \textit{hashtag}, não apresenta anúncios em exibições online ou telas publicas, cria novas publicações na sua pagina sincronizada, bloqueia usuários para novas publicações de comentários, além de ser possível definir o tempo em que a informação ficará sendo exibida na tela. Em sua versão gratuita, onde grande partes dos recursos não estão disponíveis, é possível sincronizar as principais redes sociais como Facebook, Instagram, Twitter, entre outras. 

Após a sincronização, na pagina inicial do aplicativo é possível gerenciar suas publicações que já foram postadas, selecionando uma delas, é possível ter o acesso direto a pagina ou excluir a publicação do aplicativo, ao excluir do aplicativo, a publicação não é excluída da pagina, não oferece a possibilidade de criar uma nova publicação pelo próprio aplicativo, metade do tempo de exibição das informações é de outras propagandas que não foram criadas pelo usuário. O OOZO não exibe comentários feitos na publicação, ele somente captura ela e a envia para exibição, além disso não é possível vincular uma página, somente um \textit{perfil} de usuário. Durante a exibição do conteúdo publicado, é exibido um \textit{QR code} de redirecionamento para a página da noticia completa.

\section{MangoSings}
Outra solução encontrada é o \cite{mango2017}, possui integração com as redes sociais, atualidades, informações do tempo e uma interface amigável, uma nova publicação só é possível a partir da pagina WEB. Entretanto, é necessário um dispositivo Android ou um próprio chamado Mango Sing Box conectado a uma TV ou Monitor para exibição do conteúdo criado, as publicações são \textit{slides}, que passam de acordo com o tempo configurado, sendo possível determinar tempo de transição e até quando a publicação será exibida, além da localização. 

A integração com os redes sociais é somente para captura de publicações feitas, sem comentários e curtidas. A exibição é dentro dos \textit{slides}, em nenhuma de suas versões é possível publicar nas redes sociais a partir do aplicativo. A pagina do MangoSigns oferece diversas interfaces prontas para criação dos \textit{slides}, mas também é possível criar uma nova. Na pagina de login do aplicativo é possível criar uma nova conta, usar uma conta do Facecioso ou uma conta do Google. Em sua versão gratuita, não é possível fazer sincronização com redes sociais, e é limitado a uma única publicação com 3 \textit{slides}. Além de não possuir um \textit{QR code} para que o telespectador possa acessar a noticia completa.

\section{SID Formosa}
O SID (Sistema Inteligente de Divulgação de Informações do IFG-Formosa), é outra solução, a pagina de login acessada via WEB, é possível fazer o acesso usando a conta do facebook ou uma conta cadastrada no banco de dados. Apesar de possuir integração com as redes sociais, é limitando a publicações no perfil do usuário do sistema. Usando o \textit{Raspberry Pi} como cliente, o SID apresenta as informações que lhe são configuradas. As publicações apresentadas na tela possui um \textit{QR Code} usado pelo telespectador caso tenha interesse em acessar a noticia completa. \cite{sobrinho2017}

\section{Screenly}
O \cite{screenly2017} usa o \textit{Raspberry Pi} e um programa próprio que deve ser instalado no equipamento para seu funcionamento. Na pagina de login, de acesso WEB, só é possível criar uma nova conta ou logar com uma existe, não havendo integração com Facebook ou Google para login automático. Um dos meios exibição das publicações é por meio de um Raspberry Pi com o uso de um software proprietário da Screenly instalado e uma televisão conectada a ele. Em nenhuma das versões o Screenly possui integração com as redes sociais. Em sua versão gratuita, ele é limitado a criar no máximo duas publicações para ficarem sendo exibidas sequencialmente, essas publicações podem ser imagens, links ou vídeos. Além do \textit{Raspberry Pi} é possível o uso de telas publicas para exibição, podendo ser de 1 em sua versão gratuita até 130 em sua versão mais cara.

\section{Xibo}
Outra solução encontrada na literatura corresponde ao \textit{software} Xibo, que trata-se de um sistema baseado em arquitetura cliente-servidor completo e flexível de sinalização digital que permite diversas customizações, na qual cada divulgação tem opção de estruturação das suas informações. Também suporta diferentes tipos de mídias como vídeos, imagens, texto, relógios, dados tabulares, etc. Possui gerenciador de conteúdo incluso (CMS) e possibilita que o servidor CMS e o módulo cliente estejam em dispositivos separados. No entanto, não possui integração com as redes sociais além de ser necessário que a cada divulgação a ser inserida tenha de ser estruturada como será a sua forma de apresentação \cite{xibo2017}.

\section{Comparativo}
A Tabela 1 faz um comparativo entre os sistemas citados acima, comparando algumas das funcionalidades consideradas importantes para sistemas que trabalham com a implantação de sinalização digital e \textit{maketing} digital, na qual seus elementos comparativos são descritos a seguir:
\begin{enumerate}[label=\Roman*)]
	\item Comprometimento com o propósito: o sistema em questão possibilita a veiculação de informações através de mecanismos de sinalização digital?
	\item Criação simples de divulgações: o operador possui facilidade de incluir novas divulgações com aspecto atrativo?
	\item Portabilidade: é possível visualizar a divulgação em diferentes dispositivos?
	\item Integração com redes sociais: o sistema integra-se nativamente de alguma forma com redes sociais, mesmo que de forma limitada?
	\item O conteúdo pode ser gerenciado em um dispositivo diferente ao que é criado, fortalecendo a descentralização e manutenção?
	\item A versão gratuita explora toda a capacidade do programa?
	\item Usa um sistema/dispositivo de fácil obtenção (Aplicativo próprio ou de uso comum)?
\end{enumerate}

\begin{table}[h!]
	\caption{Comparativo}
	\centering
	\begin{tabular}{|c|c|c|c|c|c|}
		\hline
		Quesito/Sistema & OOZO & MangoSigns & SID & Screenly & XIBO \\ \hline
		I 				& SIM  & SIM		& SIM & SIM 	 & SIM	\\ \hline
		II 				& SIM  & SIM 		& SIM & SIM 	 & SIM	\\ \hline
		III				& NÃO  & SIM 		& NÃO & NÃO 	 & NÃO	\\ \hline
		IV 				& SIM  & NÃO 		& SIM & NÃO 	 & NÃO	\\ \hline
		V 				& SIM  & SIM 		& SIM & SIM 	 & SIM	\\ \hline
		VI 				& NÃO  & NÃO 		& SIM & NÃO 	 & SIM	\\ \hline
		VII 			& SIM  & SIM 		& SIM & SIM 	 & SIM	\\ \hline
	\end{tabular}
\end{table}

\section{SIDv2}
Após todos os testes, nenhum das opções de software citados atendia aos objetivos desejados, que é a integração completa e em tempo real com as redes sociais, a gratuidade e o sistema de gestão acadêmica do IFB (SGA). Apesar de alguns deles possuírem a integração, ela não estava completa e nenhuma das opções gratuitas oferecia todas as funcionalidades desejadas, as que suportavam a integração era de forma bem limitada, sendo necessário uma assinatura do software.