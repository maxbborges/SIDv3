\chapter[Introdução]{Introdução}
Apesar de muito utilizada, a definição exata de mídia é difícil de ser explicada. De acordo com \citet{guazina2007}, o termo ``mídia'' é usado para representar os meios de comunicação em massa como jornais, televisão, rádio, cinema e \textit{Internet}.

Canais de televisão, outdoors ou até mesmo panfletos, em ambientes públicos ou privados, podem introduzir diversas mudanças comportamentais e comerciais. Por isso \citet{escobar2007} explica que a mídia é capaz de redefinir ``o modo como o homem se comunica e se relaciona com os semelhantes'', assim como \citet{hjarvard2012} coloca que ela pode modificar a relação e o comportamento humano, criando o conceito de midiatização, que caracteriza a influência que a mídia possui sobre as pessoas.

\citet{guazina2007} aponta que os meios de comunicação influenciam no processo de construção e formação das consciências. \citet{silva2007}, indica que a falta de acesso ou interesse por parte da população em buscar outras fontes, induzem no questionamento da veracidade das informações que são repassadas, mostrando o poder que a mídia pode ter sobre as pessoas.

Com o passar dos anos as divulgações de forma estática e tradicionais (revistas, jornais e canais de televisão) deixaram de ser os meios mais eficientes de se expor um conteúdo ou propaganda. Para \citet{meditsch2001}, o advento e popularização da \textit{Internet} trouxeram ameaças para esses meios, fazendo necessário a criação de novas formas de expor conteúdos, pensadas e desenvolvidas em conjunto com a \textit{Internet}.

Para \citet{escobar2007}, os atuais meios de veiculação de notícias (televisão e rádio) oferecem a possibilidade da informação ser obtida em tempo real, mas quando há a inclusão da \textit{Internet}, coloca-se a possibilidade de interação com as informações que são recebidas, onde os usuários atuam de forma simultânea, comentando ou opinando sobre aquele determinado assunto. Para \citet{deuze2002}, o surgimento da \textit{Internet} traz a possibilidade do público ``responder, interagir ou mesmo customizar certas histórias''. 

Diante disso, surgiram novas formas e ferramentas de \textit{marketing} ampliando os meios de disseminação de notícia, uma vez que ele é responsável por criar e melhorar as publicações que possuem o objetivo de influenciar as pessoas a adquirirem ou aderirem determinados produtos ou serviços, com o uso de diversos aparatos. Sendo assim, foram necessários que a publicidade e a tecnologia convergissem, definindo um novo conceito, o \textit{marketing} digital, que engloba novas tecnologias de comunicação e inclui as redes sociais como meio de exposição das propagandas.

A \textit{Internet} também resultou na popularização do uso das redes sociais para as diversas finalidades, incluindo a exposição de anúncios. \cite{seo2017} apontam que em uma pesquisa realizada pela \url{marketingcloud.com}, 90\% das compras efetuadas durante o período da pesquisa, foram influenciadas pelas redes sociais. O incremento destes meios, para \citet{escobar2007}, colocam a interatividade em evidência, pois por muitas vezes são usadas para ter um melhor contato com o usuário, então, a utilização de novas técnicas com o receptor, tornam a leitura menos monótona e com maior possibilidade de obter a atenção do usuário, de forma mais consistente.

Assim sendo, as mídias sociais obtiveram um importante papel para disseminação de informações ou conteúdos, tornando-se uma das ferramentas mais atraentes para divulgações. \citet{rosa2010}, aborda que crescentemente as redes sociais estão no cotidiano das pessoas. Estas são interessantes não apenas por serem um dos meios mais acessados atualmente, mas também possuem a facilidade de interação dos usuários com as notícias.


A evolução do \textit{marketing} trouxe consigo a exibição de propagandas, não só nas mídias sociais, mas também em dispositivos móveis, as pesquisas da \cite{emarketer2013} corroboram isso, relatando que os gastos com propagandas em dispositivos moveis são cada vez maiores. Além de permitir que se tenha receptores dos informes em várias localidades, além da possibilidade de interação com essas notícias que estão sendo transmitidas. Para \citet{santos2014}, a utilização do \textit{marketing} na \textit{Internet} favorece a criação do mesmo em ambiente digital.

Para ter maior abrangência nas publicações, apodera-se do conceito de sinalização digital, que para \citet{machado2010} dar-se-á com o uso de telas espalhadas por diversos pontos, com diferentes informações repassadas via \textit{Internet}. Com isso poderá haver receptores em diversos locais, independente de cidade, estado ou país. As telas ou televisores apresentam informações e propagandas de forma dinâmica, que podem ser gerenciadas remotamente de acordo com a necessidade. 

O uso de dispositivos móveis, concomitante com as telas, podem atrair um maior público, pois os dispositivos móveis, com o passar dos anos, estão sendo cada vez mais utilizados pelas pessoas. Estes dados podem ser confirmados por pesquisas feitas pela agência \cite{emarketer} no qual afirmam que até 2019 mais de 80\% das pessoas que acessam a \textit{Internet}, usarão o celular para acessa-la. Por isso, é crescente a utilização desses aparelhos como ferramenta para divulgação de informações, não somente pelo grande número de equipamentos, mas também pela integração com as redes sociais que esses oferecem. 

Para divulgação de notícias e informações, o Instituto Federal de Educação Ciência e Tecnologia de Brasília - \textit{Campus} Taguatinga (IFB), utiliza-se principalmente de suas páginas \textit{web} e Facebook. Para o uso desses meios, é necessário que os administradores façam publicações independentes para cada uma das páginas, descentralizando as publicações.

%Ademais, as únicas formas oficiais de contato entre professores e alunos é por meio do e-mail ou com uso de aplicativos externos à instituição como Moodle, Edmodo e Google Classroom. É necessário que cada estudante encaminhe o seu e-mail para que o professor possa entrar em contato, pois o atual sistema da instituição não dispõe de uma funcionalidade, na qual os professores possam obter o e-mail dos alunos automaticamente.

Para \citet{pinheiro2010}, a comunicação interna é de fato importante para o sucesso de uma organização. Em uma instituição educacional não é diferente, é interessante que os alunos e os professores sejam informados de futuros eventos, palestras e notícias. Os atuais veículos institucionais do IFB não possuem integração e atuam de forma independente.%, o que acabam degradando a qualidade e a disseminação dos informes.% 
Para \citet{santos2014}, a interatividade é significante pois tem como propósito ``estreitar o relacionamento com o público''.

De forma gráfica, com o uso do Sistema Integrado de Divulgação de Informações do IFB versão 3 (SIDv3) será possível exibir em painéis e televisores as notícias criadas. A comunidade poderá interagir com elas através de comentários na publicação do Facebook, eles serão apresentados nas telas juntamente com as notícias, possibilitando uma integração entre os dois meios de divulgação.  Além disso, o SIDv3 possui um aplicativo móvel o qual exibirá o mesmo conteúdo apresentado nos painéis e televisores e que também é capaz de realizar a comunicação entre professor e aluno, simulando uma integração com o Sistema de Gestão acadêmica (SGA).

% O Sistema Integrado de Divulgação de Informações do IFB versão 3 (SIDv3) oferece uma maior visibilidade das notícias, sendo possível, por meio de painéis espalhados pelo \textit{Campus}, uma melhor interação da comunidade com as notícias, apresentando nos painéis os comentários que foram publicados na mídia social, além de uma melhor forma de comunicação entre professor e aluno, oferecido pelo aplicativo móvel que simula um sistema de comunicação institucional, integrado ao Sistema de Gestão Acadêmica (SGA).

\section{Motivação}
\citet{bianchi2006} abordam que algo inédito e atual deve atrair a atenção do telespectador, sendo esse crucial para o sucesso das notícias que estão sendo exibidas. Quanto melhor essas forem disseminadas, maior a chance de sucesso. Nesse intuito, a interatividade e o dinamismo podem ser considerados modernos e o uso de ferramentas do cotidiano das pessoas, despertam e mantém a atenção do usuário, fazendo-o ter interesse em acompanhar e participar das noticias ou matérias.

Atualmente, o IFB utiliza o seu perfil na rede social Facebook, sua página oficial e os murais de cada \textit{Campus} para veicular notícias intrínsecas à instituição, sejam essas referentes a eventos ou institucionais. Para cada notícia é necessário que o administrador acesse as páginas separadamente e realize postagens independentes, além da necessidade de fixação de algumas dessas informações nos murais.

\citet{pinheiro2010}, cita a importância da comunicação interna, quando se tem um sistema de divulgação defasado, as informações podem não atingir o resultado esperado. Apresentar as notícias referentes ao IFB contando com um sistema mais interativo, que seja capaz de ter uma participação mais adequada do público com a notícia, como \citet{santos2014} aborda ser importante para o sucesso da informação.

Pensando nisso, um sistema que sirva de forma a centralizar as informações referentes à instituição pode ajudar no gerenciamento delas. Possibilitando criação, edição e exclusão das informações e que faça a integração dos meios de comunicação, viabilizando a troca do conteúdo em todos os pontos em que está sendo exibido, utilizando um único sistema. Propiciando uma possível interatividade do espectador com a notícia, disponibilizando o endereço para acesso completo à publicação no Facebook e exibindo os comentários feitos na mesma.

\section{Proposta}
Com uso da arquitetura cliente-servidor e tendo como base o Sistema Inteligente de Divulgação de Informações do IFG-Formosa, em sua versão 2 (SIDv2) implementado por \citet{sobrinho2017}, é proposto a elaboração da terceira versão (SIDv3). Com o uso dos conceitos de sinalização e \textit{marketing} digital, a proposta é implementar as melhoria na integração com a rede social Facebook e na apresentação de conteúdos para que o sistema faça união de distintos meios de apresentação de informes referentes ao IFB, possibilitando o gerenciamento completo de cada publicação criada pelo sistema.

O conteúdo criado, com auxílio do sistema, será vinculado à página do IFB no Facebook, sendo exibido nos painéis juntamente com os comentários realizados, devidamente moderados pelos administradores, em telas espalhadas em locais de maior movimento no \textit{Campus} Taguatinga do Instituto Federal de Brasília e nos dispositivos móveis de cada pessoa que possua o aplicativo instalado. 

A escolha da rede social Facebook é baseada nas pesquisas realizadas por \citet{muchardie2016}, na qual a apontam como uma das mais utilizadas, onde a maioria das pessoas que trabalham com \textit{marketing} digital a empregam para a exposição de conteúdos pela grande quantidade de usuários que essa detêm e também por possuir funcionalidades como compartilhamento fácil e suporte a textos com mais de 63000 caracteres, visto que o limite é superior em relação a outras redes sociais como Instagram e Twitter, na qual a capacidade é de aproximadamente 2200 e 140 caracteres, respectivamente.

O sistema visa proporcionar a integração dos meios usados atualmente para apresentação das informações referentes ao \textit{Campus}, além da inclusão de outros meios. Portanto, a ideia do SID é prover um meio que a comunidade possa interagir mais com as publicações acadêmicas e o administrador tenha uma maior facilidade de criação e edição das notícias, integrando vários serviços em um único.

% As pesquisas da \citet{emarketer} abordam que é crescente a utilização de celulares, portanto, um aplicativo para dispositivos móveis que exibe o mesmo conteúdo que são apresentados na rede social e nos televisores, disseminaria melhor as informações. Além disso, uma funcionalidade adicional estaria disponível apenas para estudantes e professores, essa ofereceria uma forma de contato entre os mesmos, observando que atualmente é realizada presencialmente, através de e-mails ou com uso de outros \textit{softwares} complementares.

Na versão para dispositivos móveis, além da apresentação das notícias, os professores e os alunos terão acesso a outra funcionalidade: o docente poderá enviar informações e avisos distintos para cada turma que ele leciona, enquanto os alunos poderão acessar cada mensagem enviada pelo professor para a turma em que ele está cadastrado.

A função de envio de mensagem é restrita e se dará através de um login, usando uma matrícula e senha fictícia cadastrados no bancos de dados que simula plataformas acadêmicas já existentes, onde não será possível o uso de dados reais por restrições de acesso a essas plataformas.

\section{Objetivos}
Este trabalho tem por objetivo principal, a partir de um sistema anterior, realizar a implementação melhoras e funcionalidades que faltaram para a versão \textit{Web} e  aperfeiçoamento da integração que o sistema realiza com o Facebook, melhorando a forma com que o sistema cria as publicações e implementando a recuperação de diversos dados presentes na rede social.

Por fim, será feito a criação de uma versão móvel que apresente as notícias criadas pelo sistema e os comentários publicados na rede social, além de simular a integração com um Sistema de Gestão acadêmica (SGA).

% Além a versão móvel, realizando a integração de diversos meios de disseminação de informações.

O intuito é aprimorar as formas com que o sistema apresenta e utiliza a rede social Facebook para prover interatividade com o usuário por meio de comentários.

\subsection{Objetivos Específicos}
	 \begin{itemize}
	\item Estudo dos \textit{frameworks} necessários para implementação e melhoria do sistema \textit{Web} e móvel.
	
	\item Estudo e detalhamento da documentação da API do Facebook para que seja realizado as melhorias necessárias e sirva como forma de possíveis consultas que visem novas implementações e usos da mesma.
	 	
	\item Disponibilização de uma API REST.
	\end{itemize}
	
\section{Metodologia}
A revisão de bibliografia será feita como meio de direcionamento do trabalho, onde serão usadas comparações entre ferramentas que apresentam o conceito de sinalização e \textit{marketing} digital, com o objetivo de avaliar as deficiências de cada uma delas, baseando-se nas necessidades do \textit{Campus}.

A análise servirá de forma a definir o que será necessário desenvolver ou alterar, para melhorar a maneira com que as informações são disseminadas, seguindo os conceitos que são considerados primordiais em um sistema de sinalização digital.

O estudo da documentação da Graph API e de suas ferramentas, tais como a Graph API Explorer, viabilizará a melhor integração do sistema com a rede social Facebook. Esses instrumentos serão utilizados para realização de testes práticos das diversas funcionalidades que a API dispõe, selecionando quais serão necessários à implementação, para que seja possível a recuperação e envio de dados e a melhoria no processo de \textit{login}.
	 
Tendo o SIDv2 como sistema base, com o auxílio das operacionalidades disponíveis na Graph API, será implementado melhorias e novas funcionalidades no sistema. As informações estarão apresentadas em dispositivos como televisores, painéis, páginas \textit{web} ou celulares e poderão ser alteradas Através de uma interface \textit{web}. Uma vez com posse das credencias de administrador. Após serem criadas ou modificadas, as publicações poderão ser transmitidas e acessadas pelos clientes em distintas plataformas ao mesmo tempo.

Todo o sistema, incluindo o móvel, seguirá o padrão de desenvolvimento ágil, com metodologia Scrum, sendo definido \textit{sprints} semanais, para a definição das funcionalidades a serem desenvolvidas ou melhoradas. 

\section{Organização do documento}
O documento esta dividido em \ref{consideracoes} capítulos. O Capítulo \ref{relacionados} aborda sobre aplicações que possuem conceitos que se assemelham com o SIDv3, expondo como essas funcionam, a fim de realizar um comparativo entre as funcionalidades apresentadas e as que são requeridas para o sistema.

O Capítulo \ref{cap:referencial} especifica conceitos e ferramentas que serão abordadas no decorrer do documento, com exceção da Graph API que é minuciada no Capítulo \ref{cap:api}, nesse é explicado as principais funcionalidades que essa oferece e exemplos da aplicação de cada uma delas.

As explicações detalhadas do sistema, a estrutura do SID, as particularidades de algumas interfaces e a maneira na qual é realizado suas ações em conjunto com a Graph API, são expostas no Capítulo \ref{cap:sid}.

No Capítulo \ref{resultados}, é abordado todo o resultado que foi obtido após a implementação das melhorias no SID, descrevendo as modificações realizadas e o comparando com os outros sistemas testados, relatando as dificuldades encontradas.

As considerações finais são expostas no Capítulo \ref{consideracoes}, denotando os benefícios de se usar o SID e sugerindo possíveis melhorias para futuras implementações.