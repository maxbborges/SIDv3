% Arquivo editável que gera o pdf
% Copyright (C) 2018  Daniel Saad Nogueira Nunes (daniel.nunes@ifb.edu.br)

% This program is free software: you can redistribute it and/or modify
% it under the terms of the GNU General Public License as published by
% the Free Software Foundation, either version 3 of the License, or
% (at your option) any later version.

% This program is distributed in the hope that it will be useful,
% but WITHOUT ANY WARRANTY; without even the implied warranty of
% MERCHANTABILITY or FITNESS FOR A PARTICULAR PURPOSE.  See the
% GNU General Public License for more details.

% You should have received a copy of the GNU General Public License
% along with this program.  If not, see <http://www.gnu.org/licenses/>.


\documentclass{aula-ifb}


% Insira os dados aqui
\author{Maxwell Borges Bezerra\\ 
\small{Orientador: prof. Daniel Saad}}
\title{Definição, Estudo e Implementação de um Sistema de Divulgação de Informações Integrado do IFB}
\subtitle{Monografia - Graduação em Bacharelado em Ciência da Computação}
\institute{Instituto Federal de Brasília, Câmpus Taguatinga}
\date{}



\begin{document}
\maketitle

\section{Sistema Integrado de Divulgações - SID}

\begin{frame}{Atuais meios}
\center
Página WEB\\
\url{http://www.ifb.edu.br/}
 
\vspace{20px}

 
Página do Facebook\\
\url{https://www.facebook.com/IFBrasilia}

\vspace{20px}

Panfletos

\vspace{20px}

Murais
\end{frame}

\begin{frame}{Quais são os problemas?}
\begin{itemize}
   \item Falta de integração entre os veículos de comunicação;
   \vspace{10px}
   \item Falta de interatividade com as notícias;
   \vspace{10px}
   \item Má disseminação da notícia;
   \vspace{10px}
   \item{Falta de uma aplicativo \textit{mobile}};
\end{itemize}	 
\end{frame}

\begin{frame}{Quais são as propostas?}
\begin{itemize}
   \item Melhorar a forma com que as notícias são expostas;
   \vspace{10px	}
   \item Melhorar a interatividade;
   \vspace{10px}
   \item Integrar 2 sistemas distintos;
   \vspace{10px}
   \item Melhorar a forma de comunicação entre professor e aluno;
   \vspace{10px}
   \item Apresentar uma forma de implantação;
\end{itemize}
\end{frame}

\begin{frame}{Portanto, os objetivos são...}

\end{frame}

\begin{frame}{Para isso foi necessário...}
\begin{itemize}
   \item Realizar uma revisão da bibliografia;
   \vspace{10px	}
   \item Estudar a documentação da Graph API e suas ferramentas;
   \vspace{10px}
   \item Realizar as modificações que forem necessárias e não implementadas na segunda versão do SID.
   \vspace{10px}
   \item Melhorar a forma de comunicação entre professor e aluno;
   \vspace{10px}
   \item Apresentar uma forma de implantação;
\end{itemize}
\end{frame}


\begin{frame}{Graph API}
Falar um pouco sobre as principais características de da API
\end{frame}

\begin{frame}{SID}
Falar sobre as funcionalidades do sistema
- Servidor
-Cliente
Aplicativo mobile
\end{frame}

\begin{frame}{Resultados}
Resultados - Comparação com os demais
\end{frame}

\begin{frame}{Graph API}
Conclusão e trabalhos futuros
\end{frame}

\end{document}